% !TeX encoding = UTF-8
% !TeX spellcheck = es_AR
%% Add the word "answers" to document class parameters in order to print the solutions.
\documentclass[12pt, addpoints]{../../common/epyl_exam_template}

\title{Práctica 2.2 - Lógica Proposicional}
\date{2018.1}

\begin{document}
\makeexamheader
\makeexamtitle
\examrule
\begin{questions}
  \question
    Identifique cuales de las siguientes son proposiciones y cuales no lo son.
    \begin{parts}
      \part Microsoft es la empresa detrás del sistema operativo Windows.
      \begin{solution} \fulltrue \end{solution}
      \part ¿Hay vida en Marte?
      \begin{solution} \fullfalse \end{solution}
      \part Las células eucariotas tienen núcleo y las procariotas no lo tienen.
      \begin{solution} \fulltrue \end{solution}
      \part ¡Eres un impostor!
      \begin{solution} \fullfalse \end{solution}
      \part ¿Es un celular inteligente?
      \begin{solution} \fullfalse \end{solution}
      \part No es cierto que haya empresas que beneficien a algunos políticos
      \begin{solution} \fulltrue \end{solution}
    \end{parts}

  \question
    Identifique las conectivas y las proposiciones involucradas en las
    siguientes oraciones:
    \begin{parts}
      \part Safari es un navegador y viene instalado en macOS.
      \begin{solution}
        Proposiciones:
        \begin{itemize}
          \item Safari es un navegador.
          \item Safari viene instalado en macOS.
        \end{itemize}

        Frase con conectivas:
        \begin{itemize}
          \item Safari es un navegador $\land$ Safari viene instalado en macOS.
        \end{itemize}
      \end{solution}

      \part Safari es un navegador, y no es cierto que viene instalado en Windows.
      \begin{solution}
        Proposiciones:
        \begin{itemize}
          \item Safari es un navegador.
          \item Safari viene instalado en Windows.
        \end{itemize}

        Frase con conectivas:
        \begin{itemize}
          \item Safari es un navegador $\land$ $\lnot$ Safari viene instalado en
          macOS.
        \end{itemize}
      \end{solution}

      \part Linux funciona en computadoras de escritorio, en dispositivos móviles
      o en otros dispositivos
      \begin{solution}
        Proposiciones:
        \begin{itemize}
          \item Linux funciona en computadoras de escritorio
          \item Linux funciona en dispositivos móviles
          \item Linux funciona en otros dispositivos
        \end{itemize}

        Frase con conectivas:
        \begin{itemize}
          \item Linux funciona en computadoras de escritorio $\lor$
          Linux funciona en dispositivos móviles $\lor$
          Linux funciona en otros dispositivos
        \end{itemize}
      \end{solution}

      \part GIMP no es ni un sistema operativo ni un navegador
      \begin{solution}
        Proposiciones:
        \begin{itemize}
          \item GIMP es un sistema operativo
          \item GIMP es un navegador
        \end{itemize}

        Frase con conectivas:
        \begin{itemize}
          \item $\lnot$ GIMP es un sistema operativo $\land$ $\lnot$
          GIMP es un navegador
        \end{itemize}
      \end{solution}

      \part Manjaro es una distribución de Linux basada en Arch y viene con
          el escritorio Gnome, el escritorio KDE u otros escritorio
        \begin{solution}
          Proposiciones:
          \begin{itemize}
            \item Manjaro es una distribución de Linux basada en Arch
            \item Manjaro viene con el escritorio Gnome
            \item Manjaro viene con el escritorio KDE
            \item Manjaro viene con otros escritorios
          \end{itemize}

          Frase con conectivas:
          \begin{itemize}
            \item Manjaro es una distribución de Linux basada en Arch $\land$
              (Manjaro viene con el escritorio Gnome $\lor$ Manjaro viene con el
              escritorio KDE $\lor$ Manjaro viene con otros escritorios)
          \end{itemize}
        \end{solution}

      \part No es cierto que Debian sea la distribución más vieja de Linux, pero
        Slackware si lo es.
        \begin{solution}
          Proposiciones:
          \begin{itemize}
            \item Debian es la distribución más vieja de Linux
            \item Slackware es la distribución más vieja de Linux
          \end{itemize}

          Frase con conectivas:
          \begin{itemize}
            \item $\lnot$ Debian es la distribución más vieja de Linux $\land$
            Slackware es la distribución más vieja de Linux
          \end{itemize}
        \end{solution}
      \end{parts}

  \question
    Dados los siguientes razonamientos, identifique los indicadores de premisa,
    los indicadores de conclusión, y estructure el razonamiento en premisas y
    conclusión.
    \begin{parts}
      \part Será Santino quien venga a la fiesta. Dado que a la fiesta
      iba a venir o bien Guadalupe o bien Santino. Pero Guadalupe no va
      a venir.
      \begin{solution}
        Indicador de premisa: ``Dado que''

        Premisas:
        \begin{itemize}
          \item A la fiesta iba a venir o bien Guadalupe o bien Santino.
          \item Guadalupe no va a venir.
        \end{itemize}

        Conclusión:
        \begin{itemize}
          \item Será Santino quien venga a la fiesta.
        \end{itemize}
      \end{solution}

      \part Si hay bananas o hay manzanas, entonces hay fruta. No hay manzanas.
        Pero hay fruta. Por lo tanto, hay bananas.
        \begin{solution}
          Indicador de conclusión: ``Por lo tanto''

          Premisas:
          \begin{itemize}
            \item Si hay bananas o hay manzanas, entonces hay fruta.
            \item No hay manzanas.
            \item Hay fruta.
          \end{itemize}

          Conclusión:
          \begin{itemize}
            \item Hay bananas.
          \end{itemize}
        \end{solution}

      \part La Tierra es plana o es redonda. Si la Tierra es redonda entonces no
        nos caeremos por el borde. En cambio, si la Tierra es plana, si nos
        caeremos por el borde. Pero la Tierra no es plana. En consecuencia, no nos
        caeremos por el borde.
        \begin{solution}
          Indicador de conclusión: ``En consecuencia''

          Premisas:
          \begin{itemize}
            \item La Tierra es plana o es redonda.
            \item Si la Tierra es redonda entonces no nos caeremos por el borde.
            \item Si la Tierra es plana, nos caeremos por el borde.
            \item La Tierra no es plana.
          \end{itemize}

          Conclusión:
          \begin{itemize}
            \item No nos caeremos por el borde.
          \end{itemize}
        \end{solution}

      \part El ascensor está en movimiento. Esto es así ya que si el ascensor no
        abre sus puertas es porque o bien está en movimiento, o bien se está preparando
        para moverse. Y la puerta está cerrada y el ascensor no se está preparando
        para moverse.
        \begin{solution}
          Indicador de premisas: ``Esto es así ya que''

          Premisas:
          \begin{itemize}
            \item Si el ascensor no abre sus puertas es porque o bien está en
              movimiento, o bien se está preparando para moverse.
            \item La puerta está cerrada y el ascensor no se está preparando para moverse.
          \end{itemize}

          Conclusión:
          \begin{itemize}
            \item El ascensor está en movimiento.
          \end{itemize}
        \end{solution}

        \part Si la celda actual está pintada de rojo y la celda siguiente a la
          derecha está pintada de negro, entonces el cabezal se moverá dos
          lugares a la de derecha. El cabezal no se movió dos lugares a la
          derecha. En consiguiente, o bien la celda actual no estaba pintada de
          rojo, o bien la celda siguiente a la derecha no estaba pintada de
          negro.
          \begin{solution}
            Indicador de conclusión: ``En consiguiente''

            Premisas:
            \begin{itemize}
              \item Si la celda actual está pintada de rojo y la celda siguiente a
                la derecha está pintada de negro, entonces el cabezal se moverá
                dos lugares a la de derecha.
              \item El cabezal no se movió dos lugares a la derecha.
            \end{itemize}

            Conclusión:
            \begin{itemize}
              \item O bien la celda actual no estaba pintada de rojo, o bien la
                celda siguiente a la derecha no estaba pintada de negro.
            \end{itemize}
          \end{solution}

        \part Los pasajeros no murieron. Por tanto, la U.S.S Enterprise
          descendió con éxito en la superficie. Ya qué, si la U.S.S. Enterprise
          no descendía con éxito en la superficie, entonces los pasajeros morirían.
          \begin{solution}
            Indicador de conclusión: ``Por tanto''\\
            Indicador de premisa: ``Ya qué''

            Premisas:
            \begin{itemize}
              \item Los pasajeros no murieron.
              \item Si la U.S.S. Enterprise no descendía con éxito en la superficie,
                entonces los pasajeros morirían.
            \end{itemize}

            Conclusión:
            \begin{itemize}
              \item La U.S.S Enterprise descendió con éxito en la superficie.
            \end{itemize}
          \end{solution}
      \end{parts}

    \question
      Dadas las siguientes estructuras de razonamientos, analizar si los
      mismos son válidos o inválidos.\footnote{Si le cuesta pensar en términos
        de las proposiciones ``Se cumple P'', ``Se cumple Q'', ``Se cumple R'',
        puede reemplazar las mismas por otras tres proposiciones que le resulten
        más intuitivas.}

      \newcommand\secump[1]{\text{Se cumple #1}}

      \begin{parts}
        \small
        \part $\secump{P} \land \secump{Q} \lseq \secump{P}$
        \begin{solution}
          \centerline{\begin{tabular}{c c | c || c }
            \hline
            Concl. && Premisa 1 & implic. \\
            $P$ & $Q$ & $P \land Q$ & $(P \land Q) \lthen P$\\
            \hline
            \true  & \true  & \true  & \true \\
            \true  & \false & \false & \true \\
            \false & \true  & \false & \true \\
            \false & \false & \false & \true \\
          \end{tabular}}
          Válido
        \end{solution}
        \part $\secump{P} \lor \secump{Q}, \lnot \secump{Q} \lseq \lnot \secump{P}$
        \begin{solution}
          \footnotesize
          \centerline{\begin{tabular}{c c | c | c | c | c || c}
            && Premisa 1 & Premisa 2 & & Concl. & implic. \\
            \hline
            $P$ & $Q$ & $P \lor Q$ & $\lnot Q$ & $(P \lor Q) \land (\lnot Q)$
              & $\lnot P$ & $((P \lor Q) \land (\lnot Q)) \lthen (\lnot P)$\\
            \hline
            \true  & \true  & \true  & \false & \false & \false & \true \\
            \true  & \false & \true  & \true  & \true  & \false & \false \\
            \false & \true  & \true  & \false & \false & \true  & \true \\
            \false & \false & \false & \true  & \false & \true  & \true \\
          \end{tabular}}
          \normalsize Inválido
        \end{solution}
        \part $\secump{P} \lor \secump{Q}, \lnot \secump{Q} \lseq \secump{P}$
        \begin{solution}
          \centerline{\begin{tabular}{c c | c | c | c || c}
            Concl. && Premisa 1 & Premisa 2 & & implic. \\
            \hline
            $P$ & $Q$ & $P \lor Q$ & $\lnot Q$ & $(P \lor Q) \land (\lnot Q)$
              & $((P \lor Q) \land (\lnot Q)) \lthen (P)$\\
            \hline
            \true  & \true  & \true  & \false & \false & \true \\
            \true  & \false & \true  & \true  & \true  & \true \\
            \false & \true  & \true  & \false & \false & \true \\
            \false & \false & \false & \true  & \false & \true \\
          \end{tabular}}
          Válido
        \end{solution}
        \part $\secump{P} \lthen \secump{Q}, \secump{P} \lseq \secump{Q}$
        \begin{solution}
          \centerline{\begin{tabular}{c c | c | c || c}
            Premisa 2 & Concl. & Premisa 1 & & implic. \\
            \hline
            $P$ & $Q$ & $P \lthen Q$ & $(P \lthen Q) \land (P)$ & $((P \lthen Q) \land (P)) \lthen Q$  \\
            \hline
            \true  & \true  & \true  & \true  & \true  \\
            \true  & \false & \false & \false & \true \\
            \false & \true  & \true  & \false & \true  \\
            \false & \false & \true  & \false & \true \\
          \end{tabular}}
          Válido
        \end{solution}
        \part $\lnot \secump{P} \lthen \secump{Q}, \lnot \secump{Q} \lseq \lnot \secump{P}$
        \begin{solution}
          \footnotesize
            \centerline{\begin{tabular}{c c | c | c | c | c || c}
            && Concl. & Premisa 1 & Premisa 2 & & implic.\\
            \hline
            $P$ & $Q$ & $\lnot P$ & $\lnot P \lthen Q$ & $\lnot Q$ &
              $(\lnot P \lthen Q) \land (\lnot Q)$ & $((\lnot P \lthen Q) \land (\lnot Q)) \lthen \lnot P$\\
            \hline
            \true  & \true  & \false & \true  & \false & \false & \true \\
            \true  & \false & \false & \true  & \true  & \true  & \true \\
            \false & \true  & \true  & \true  & \false & \false & \true \\
            \false & \false & \true  & \false & \true  & \false & \true \\
          \end{tabular}}
          \normalsize Inválido
        \end{solution}
        \part $\secump{P} \lthen \secump{Q}, \lnot \secump{Q} \lseq \lnot \secump{P}$
        \begin{solution}
          \footnotesize
          \centerline{\begin{tabular}{c c | c | c | c | c || c}
            && Concl. & Premisa 1 & Premisa 2 & & implic.\\
            \hline
            $P$ & $Q$ & $\lnot P$ & $P \lthen Q$ & $\lnot Q$ &
              $(P \lthen Q) \land (\lnot Q)$ & $((P \lthen Q) \land (\lnot Q)) \lthen \lnot P$\\
            \hline
            \true  & \true  & \false & \true  & \false & \false & \true \\
            \true  & \false & \false & \false & \true  & \false & \true \\
            \false & \true  & \true  & \true  & \false & \false & \true \\
            \false & \false & \true  & \true  & \true  & \true  & \true \\
          \end{tabular}}
          \normalsize Válido
        \end{solution}
        \part \footnotesize $(\secump{P} \lthen \secump{Q}), (\secump{Q} \lthen \secump{R}), \secump{P} \lseq \secump{R}$ \normalsize
        \begin{solution}
          \scriptsize
            \centerline{\begin{tabular}{c c c | c | c | c || c}
            Premisa 3 & Concl. & Premisa 1 & Premisa 2 & & & implic.\\
            \hline
            $P$ & $Q$ & $R$ & $P \lthen Q$ & $Q \lthen R$ & $A = (P \lthen Q) \land (Q \lthen R) \land (P)$ & $A \lthen R$\\
            \hline
            \true  & \true  & \true  & \true  & \true  & \true  & \true \\
            \true  & \true  & \false & \true  & \false & \false & \true \\
            \true  & \false & \true  & \false & \true  & \false & \true \\
            \true  & \false & \false & \false & \true  & \false & \true \\
            \false & \true  & \true  & \true  & \true  & \false & \true \\
            \false & \true  & \false & \true  & \false & \false & \true \\
            \false & \false & \true  & \true  & \true  & \false & \true \\
            \false & \false & \false & \true  & \true  & \false & \true \\
          \end{tabular}}
          \normalsize Válido
        \end{solution}
        \part \footnotesize $(\secump{P}~\lor~\secump{Q}) \lthen \secump{R}, \secump{P} \land \lnot \secump{Q} \lseq \secump{R}$ \normalsize
        \begin{solution}
          \scriptsize
          \centerline{\begin{tabular}{c c c c c | c | c | c || c}
            && Concl. &&& Premisa 1 & Premisa 2 & & implic. \\
            \hline
            $P$ & $Q$ & $R$ & $\lnot Q$ & $P \lor Q$ & $A = (P \lor Q) \lthen R$ & $B = P \land \lnot Q$ & $A \land B$ & $(A \land B) \lthen R$\\
            \hline
            \true  & \true  & \true  & \false & \true  & \true  & \true  & \true  & \true \\
            \true  & \true  & \false & \false & \true  & \false & \false & \false & \true \\
            \true  & \false & \true  & \true  & \true  & \true  & \true  & \true  & \true \\
            \true  & \false & \false & \true  & \true  & \false & \false & \false & \true \\
            \false & \true  & \true  & \false & \true  & \true  & \false & \false & \true \\
            \false & \true  & \false & \false & \true  & \false & \false & \false & \true \\
            \false & \false & \true  & \true  & \false & \true  & \false & \false & \true \\
            \false & \false & \false & \true  & \false & \true  & \false & \false & \true \\
          \end{tabular}}
          \normalsize Válido
        \end{solution}
      \end{parts}
  \normalsize

  \question
    Analice los siguientes razonamientos y extrapole la fórmula lógica de cada
    proposición. Luego, pruebe si el razonamiento es válido o inválido.
    \begin{parts}
      \part
        Si hay bananas o hay manzanas entonces hay fruta. No hay manzanas.
        Por tanto, no hay fruta.
        \begin{solution}
          p = Hay bananas\\
          q = Hay manzanas\\

          $p \lor q, \lnot q \lseq \lnot p$\\
          % TODO Los pibes se maman en esto porque no les dan los resultados.
          % Habría que agregar las tablas de verdad para cada uno para mostrar
          % que es válido o inválido.
          Es un razonamiento Inválido.
        \end{solution}
      \part 
        Si se comienza a tratar el calentamiento global se podrá detener a tiempo.
        Pero no se comienza a trata el calentamiento global. Por tanto, no se
        podrá detener a tiempo.
        \begin{solution}
          p = Se comienza a tratar el calentamiento global\\
          q = Se detendrá el calentamiento global a tiempo\\

          $p \lthen q, \lnot p \lseq \lnot q$\\
          % TODO Agregar tabla de verdad
          Es un razonamiento Inválido.
        \end{solution}
      \part 
        No es cierto que se deba detener el ascensor. Dado que, el ascensor
        se debe detener solo si está frente a la puerta de un piso. Y no es
        cierto que el ascensor esté frente a la puerta de un piso.
        \begin{solution}
          p = El ascensor está frente a la puerta de un piso\\
          q = Se debe detener el ascensor\\

          $p \liff q, \lnot p \lseq \lnot q$\\
          % TODO Agregar tabla de verdad
          Es un razonamiento Válido.
        \end{solution}
      \part
        Se han movilizado tropas aliadas al norte. Por lo tanto, hay ejércitos
        enemigos al norte. Ya qué si no hay ejércitos enemigos al norte entonces
        no es necesario movilizar tropas aliadas en esa dirección.
        \begin{solution}
          p = Se han movilizado tropas aliadas al norte\\
          q = Hay ejércitos enemigos al norte\\

          $\lnot q \lthen \lnot p, p \lseq q$\\
          % TODO Agregar tabla de verdad
          Es un razonamiento Válido.
        \end{solution}
      \part
        Si un software es libre, entonces tiene una licencia libre. Si un
        software tiene licencia libre entonces garantiza las cuatro libertades
        del software libre. Por lo tanto, si un software es libre, entonces
        garantiza las cuatro libertades del software libre.
        \begin{solution}
          p = El software es libre\\
          q = El software tiene una licencia libre\\
          r = El software garantiza las cuatro libertades del software libre\\

          $p \lthen q, q \lthen r \lseq p \lthen r$\\
          % TODO Agregar tabla de verdad
          Es un razonamiento Válido.
        \end{solution}
      \part
        La teoría de cuerdas une la gravedad con la mecánica cuántica, por tanto,
        es una teoría de como funciona nuestro universo. La teoría de cuerdas
        requiere de diez dimensiones para funcionar. En consecuencia, nuestro
        universo cuenta con diez dimensiones.
        \begin{solution}
          p = La teoría de cuerdas une la gravedad con la mecánica cuántica\\
          q = La teoría de cuerdas es una teoría de como funciona nuestro universo\\
          r = La teoría de cuerdas requiere de diez dimensiones para funcionar\\
          s = Nuestro universo cuenta con diez dimensiones.

          $p \lor q, r \lseq s$\\
          % TODO Agregar tabla de verdad
          Es un razonamiento Inválido.
          
          Se espera que no realicen la tabla de verdad llegado este punto, sino
          que les resulte natural comprender que, si en la conclusión tengo algo
          que no depende de las premisas, entonces es claramente un razonamiento
          inválido.
        \end{solution}
      \part
        Hay que verificar su existencia o hay que tener fe ciega en su existencia.
        Es posible verificar su existencia solo si se cuenta con el equipo adecuado.
        No se cuenta con el equipo adecuado. En consiguiente, hay que tener fe ciega
        en su existencia.
        \begin{solution}
          p = Hay que verificar su existencia\\
          q = Hay que tener fe ciega en su existencia\\
          r = Se cuenta con el equipo adecuado\\

          $p \lor q, p \liff r, \lnot r \lseq q$\\
          % TODO Agregar tabla de verdad
          Es un razonamiento Válido.
        \end{solution}
      \part 
        Si y solo si se cuenta con suficiente dinero se podrá construir el
        edificio. Si se ha vendido suficiente cantidad de mineral, entonces se
        contará con suficiente dinero. Por ende, se puede construir el edificio. 
        \begin{solution}
          p = Se cuenta con suficiente dinero\\
          q = Se podrá construir el edificio\\
          r = Se ha vendido suficiente cantidad de mineral\\

          $p \liff q, r \lthen p \lseq q$\\
          % TODO Agregar tabla de verdad
          Es un razonamiento Inválido.
        \end{solution}
    \end{parts}

  \question
    \begin{parts}
      \part
        ¿Será lo mismo la frase \colored{``Hay vida en Marte y hay vida en Ganimedes''}
        a la frase \colored{``Hay vida en Ganimedes y hay vida en Marte''}?
        \begin{solution}
          Si, son lo mismo. Esto nos indica que la propiedad conmutativa vale en la
          conjunción. Podemos probarlo mostrando que $(p \land q) \liff (q \land p)$
          es una tautología.
        \end{solution}
      \part
        ¿Será lo mismo la frase \colored{``Hay vida en Marte o hay vida en Ganimedes''}
        a la frase \colored{``Hay vida en Ganimedes o hay vida en Marte''}?
        \begin{solution}
          Si, son lo mismo. Esto nos indica que la propiedad conmutativa vale en la
          disyunción. Podemos probarlo mostrando que $(p \lor q) \liff (q \lor p)$
          es una tautología.
        \end{solution}
      \part
        ¿Será lo mismo la frase \colored{``Si hay vida en Marte entonces hay vida en Ganimedes''}
        a la frase \colored{``Si hay vida en Ganimedes entonces hay vida en Marte''}?
        \begin{solution}
          En este caso no son lo mismo, por tanto, la conmutatividad no aplica.
          Podemos ver que son claramente distintos probando $(p \lthen q) \liff (q \lthen p)$
          y viendo que efectivamente no da una tautología.
        \end{solution}
      \part
        ¿Será lo mismo la frase \colored{``No es cierto que no hay vida en Marte''}
        a la frase \colored{``Hay vida en Marte''}?
        \begin{solution}
          Si, son lo mismo. Esto nos muestra que la doble negación es equivalente
          a la frase sin negar. Lo probamos mediante $\lnot (\lnot p) \liff p$
          y viendo que es una tautología.
        \end{solution}
      \part
        ¿Será lo mismo la frase \colored{``No es cierto que, hay vida en Marte o hay vida en Ganimedes''}
        a la frase \colored{``No hay vida en Marte y no hay vida en Ganimedes''}?
        \begin{solution}
          Si, son lo mismo. Esto nos muestra que negar una disyunción, es lo mismo
          que negar ambas partes de la conjunción. Primer caso de De Morgan. Se
          puede probar mediante $\lnot (p \lor q) \liff (\lnot p \land \lnot q)$
          y viendo que es una tautología.
        \end{solution}
      \part
        ¿Será lo mismo la frase \colored{``No es cierto que, hay vida en Marte o hay vida en Ganimedes''}
        a la frase \colored{``No hay vida en Marte y no hay vida en Ganimedes''}?
        \begin{solution}
          Si, son lo mismo. Esto nos muestra que negar una disyunción, es lo mismo
          que negar ambas partes de la conjunción. Primer caso de De Morgan. Se
          puede probar mediante $\lnot (p \lor q) \liff (\lnot p \land \lnot q)$
          y viendo que es una tautología.
        \end{solution}
      \part
        ¿Será lo mismo la frase \colored{``No es cierto que, hay vida en Marte y hay vida en Ganimedes''}
        a la frase \colored{``No hay vida en Marte o no hay vida en Ganimedes''}?
        \begin{solution}
          Si, son lo mismo. Esto nos muestra que negar una conjunción, es lo mismo
          que negar ambas partes de la disyunción. Segundo caso de De Morgan. Se
          puede probar mediante $\lnot (p \land q) \liff (\lnot p \lor \lnot q)$
          y viendo que es una tautología.
        \end{solution}
    \end{parts}
\end{questions}
\end{document}
