% !TeX encoding = UTF-8
% !TeX spellcheck = es_AR
% arara: pdflatex
% arara: pdflatex
% arara: pdflatex
% arara: clean: {files: [01_conectivas.aux, 01_conectivas.out, 01_conectivas.log]}
% arara: clean: {files: [01_conectivas.fdb_latexmk, 01_conectivas.fls, 01_conectivas.vrb, 01_conectivas.synctex.gz]}

%% Add the word "answers" to document class parameters in order to print the solutions.
\documentclass[12pt, addpoints]{../../common/epyl_exam_template}

\title{Práctica 2.3 - Lógica de Predicados}
\date{2018.1}

\begin{document}
\makeexamheader
\makeexamtitle
\examrule
\begin{questions}
  \question
    Identifique en las siguientes oraciones cuales son los individuos de los que
    se habla, y cuales las propiedades que se menciona de los mismos. Escriba las
    propiedades utilizando parámetros.
    \begin{parts}
      \part La Tierra es un planeta.
      \begin{solution}
        constante: La Tierra\\
        propiedad: $x$ es un planeta
      \end{solution}
      \part Jorge está cansado.
      \begin{solution}
        constante: Jorge\\
        propiedad: $x$ está cansado
      \end{solution}
      \part Paula canta.
      \begin{solution}
        constante: Paula\\
        propiedad: $x$ canta
      \end{solution}
      \part El Chingolo es un pájaro.
      \begin{solution}
        constante: El Chingolo\\
        propiedad: $x$ es un pájaro
      \end{solution}
      \part Argentina es un país.
      \begin{solution}
        constante: Argentina\\
        propiedad: $x$ es un país
      \end{solution}
      \part Argentina es un equipo de fútbol.
      \begin{solution}
        constante: Argentina\\
        propiedad: $x$ es un equipo de fútbol
      \end{solution}
      \part El triángulo es una figura geométrica.
      \begin{solution}
        constante: El triangulo\\
        propiedad: $x$ es una figura geométrica
      \end{solution}
  \end{parts}

  \question
    Identifique en las siguientes oraciones cuales son los individuos de los que
    se habla, y cuales son las relaciones entre los mismos. Escriba las relaciones
    utilizando parámetros.
    \begin{parts}
      \part Pablo es hermano de Dario.
      \begin{solution}
        constantes: Pablo, Darío\\
        relación: $x$ es hermano de $y$
      \end{solution}
      \part Argentina juega contra Islandia.
      \begin{solution}
        constantes: Argentina, Islandia\\
        relación: $x$ juega contra $y$
      \end{solution}
      \part Maria es madre de Catalina.
      \begin{solution}
        constantes: María, Catalina\\
        relación: $x$ es hermano de $y$
      \end{solution}
      \part Pedro es padre de Catalina.
      \begin{solution}
        constantes: Pedro, Catalina\\
        relación: $x$ es padre de $y$
      \end{solution}
      \part Catalina es hermana de Felipe.
      \begin{solution}
        constantes: Catalina, Felipe\\
        relación: $x$ es hermana de $y$
      \end{solution}
    \end{parts}

  \question
    Dadas las siguientes propiedades y relaciones:
    \begin{itemize}
      \item $x$ es hombre
      \item $x$ es mayor de edad
      \item $x$ es argentino
      \item $x$ votó a $y$
    \end{itemize}
    Busque equivalencias para las siguientes propiedades utilizando las dadas y
    las conectivas que conoce.
    \begin{parts}
      \part $x$ es mujer
      \begin{solution}
        $x$ es mujer = $\lnot$ ($x$ es hombre)
      \end{solution}
      \part $x$ es una persona
      \begin{solution}
        $x$ es una persona = ($x$ es hombre) $\lor$ ($x$ es mujer)
      \end{solution}
      \part $x$ puede votar
      \begin{solution}
        $x$ puede votar =\\
        \quad ($x$ es una persona) $\land$ ($x$ es mayor de edad) $\land$ ($x$ es argentino)
      \end{solution}
      \part $x$ se votó a si mismo
      \begin{solution}
        $x$ se votó a si mismo = ($x$ puede votar) $\land$ ($x$ voto a $x$)
      \end{solution}
    \end{parts}

  \question
    Considere las siguientes expresiones que representan una famosa variación del
    juego piedra-papel-tijeras
    
    \begin{itemize}
      \item Las tijeras cortan al papel.
      \item El papel envuelve a la piedra.
      \item La piedra aplasta al lagarto.
      \item El lagarto envenena a Spock.
      \item Spock destruye las tijeras.
      \item Las tijeras decapitan al lagarto.
      \item El lagarto se come al papel.
      \item El papel desautoriza a Spock.
      \item Spock vaporiza la roca.
      \item La piedra aplasta las tijeras.
    \end{itemize}

    Tenga en cuenta que la expresión ``tijera corta al papel'' representa que
    la tijera vence al papel. Es decir, toda expresión, cualquiera sea, puede
    ser reformulada en término de, el \textbf{primer elemento vence al segundo}.

    Se pide complete la tablas a continuación para expresar quien vence a quien
    en dicho juego.

    \begin{tabular}{| l | c | c | c | c | c |}
      \hline
      $x$ vence a $y$ & Piedra & Papel & Tijera & Lagarto & Spock \\
      \hline
      Piedra  &&&&& \\
      \hline
      Papel   &&&&& \\
      \hline
      Tijera  &&&&& \\
      \hline
      Lagarto &&&&& \\
      \hline
      Spock   &&&&& \\
      \hline
    \end{tabular}

    \begin{solution}
      \begin{tabular}{| l | c | c | c | c | c |}
        \hline
        $x$ vence a $y$ & Piedra & Papel & Tijera & Lagarto & Spock \\
        \hline
        Piedra  & \false & \false & \true & \true & \false \\
        \hline
        Papel   & \true & \false & \false & \false & \true \\
        \hline
        Tijera  & \false & \true & \false & \true & \false \\
        \hline
        Lagarto & \false & \true & \false & \false & \true \\
        \hline
        Spock   & \true & \false & \true & \false & \false \\
        \hline
      \end{tabular}
    \end{solution}


  \question
    Pasar del lenguaje natural al lenguaje formal de la lógica de predicados
    las siguientes expresiones:
    \begin{parts}
      \part Catalina disfruta de correr
        \begin{solution}
          catalina = Catalina\\
          DisfrutaDeCorrer($x$) = $x$ disfruta de correr\\
          ~\\
          DisfrutaDeCorrer(catalina)
        \end{solution}
      \part Mario adora las monedas
        \begin{solution}
          mario = Mario\\
          AdoraLasMonedas($x$) = $x$ adora las monedas\\
          ~\\
          AdoraLasMonedas(mario)
        \end{solution}
      \part Buenos Aires es una provincia y está altamente poblada
        \begin{solution}
          ba = Buenos Aires\\
          EsProvincia($x$) = $x$ es una provincia\\
          AltamentePob($x$) = $x$ está altamente poblada\\
          ~\\
          EsProvincia(ba) $\land$ AltamentePob(ba)
        \end{solution}
      \part Está pintada de Rojo o está pintada de Azul
        \begin{solution}
          rojo = Rojo\\
          azul = Azul\\
          PintadaDe($x$) = está pintada de $x$\\
          ~\\
          PintadaDe(rojo) $\lor$ PintadaDe(azul)
        \end{solution}
      \part No es cierto que Rin Tin Tin sea un bulldog
        \begin{solution}
          rtt = Rin Tin Tin\\
          EsBulldog($x$) = $x$ es un bulldog\\
          ~\\
          $\lnot$ EsBulldog(rtt)
        \end{solution}
      \part
        Maria es amiga de Luis
        \begin{solution}
          maria = Maria\\
          luis = Luis\\
          Amiga($x$, $y$) = $x$ es amiga de $y$\\
          ~\\
          Amiga(maria, luis)
        \end{solution}
      \part
        Linux utiliza licencia GPL y FreeBSD utiliza licencia BSD
        \begin{solution}
          linux = Linux\\
          freebsd = FreeBSD\\
          gpl = Licencia GPL\\
          bsd = Licencia BSD\\
          UsaLicencia($x$, $y$) = $x$ utiliza la licencia $y$\\
          ~\\
          UsaLicencia(linux, gpl) $\land$ UsaLicencia(freebsd, bsd)
        \end{solution}
      \part
        No es cierto que Ubuntu utilice el núcleo illumos, pero si usa el núcleo Linux
        \begin{solution}
          linux = Linux\\
          illumos = illumos\\
          ubuntu = Ubuntu\\
          Nucleo($x$, $y$) = $x$ utiliza el núcleo $y$\\
          ~\\
          ($\lnot$ Nucleo(ubuntu, illumos)) $\land$ Nucleo(ubuntu, linux)
        \end{solution}
      \part
        Bill Gates, Steve Jobs y Larry Ellison son grandes empresarios del software.
        \begin{solution}
          gates = Bill Gates\\
          jobs = Steve Jobs\\
          ellison = Larry Ellison\\
          GranEmpresario($x$) = $x$ es un gran empresario de la industria del software\\
          ~\\
          GranEmpresario(gates) $\land$ GranEmpresario(jobs) $\land$ GranEmpresario(ellison)
        \end{solution}
      \part
        $a$ es más grande que $b$ y que $c$
        \begin{solution}
          a = $a$\\
          b = $b$\\
          c = $c$\\
          MasGrande($x$, $y$) = $x$ mas grande que $y$\\
          ~\\
          MasGrande(a, b) $\land$ MasGrande(a, c) \\
          o también\\
          a > b $\land$ a > c
        \end{solution}
      \part
        $a$ está entre $b$ y $c$
        \begin{solution}
          a = $a$\\
          b = $b$\\
          c = $c$\\
          Entre($x$, $y$, $z$) = $x$ esta entre $y$ y $z$\\
          ~\\
          Entre(a, b, c) $\lor$ Entre(a, c, b) \\
          o también\\
          b < a < c $\lor$ b > a > c
        \end{solution}
      \part
        Todos tienen mucho sueño
        \begin{solution}
          Sueño($x$) = $x$ tiene mucho sueño\\
          ~\\
          $\forall a.$ Sueño($a$) \\
          Interesante mostrar\\
          $\forall x.$ Sueño($x$)
        \end{solution}
      \part
        Alguien confía en Pedro
        \begin{solution}
          pedro = Pedro\\
          ConfiaEn($x$, $y$) = $x$ confía en $y$\\
          ~\\
          $\exists a.$ ConfiaEn($a$, pedro)
        \end{solution}
      \part
        Messi hace jugar a todos
        \begin{solution}
          messi = Messi\\
          HaceJugar($x$, $y$) = $x$ hace jugar a $y$\\
          ~\\
          $\forall a.$ HaceJugar(messi, $a$)
        \end{solution}
      \part
        Luis no quiere a nadie
        \begin{solution}
          luis = Luis\\
          QuiereA($x$, $y$) = $x$ quiere a $y$\\
          ~\\
          $\nexists a.$ QuiereA(luis, $a$)
        \end{solution}
      \part
        Alguien quiere a todos
        \begin{solution}
          QuiereA($x$, $y$) = $x$ quiere a $y$\\
          ~\\
          $\exists a.\forall b.$ QuiereA($a$, $b$)
        \end{solution}
      \part
        Todos quieren a alguien
        \begin{solution}
          QuiereA($x$, $y$) = $x$ quiere a $y$\\
          ~\\
          $\forall a.\exists b.$ QuiereA($a$, $b$)
        \end{solution}
    \end{parts}

    \question
      Dadas las constantes, funciones y predicados, que se muestran a continuación
      \begin{itemize}
        \item c = Carlos
        \item l = Luis
        \item a = Ana
        \item m = María
        \item H($x$) = $x$ es un hombre
        \item P($x$, $y$) = $x$ es progenitor biológico de $y$
      \end{itemize}
      se pide buscar la fórmula para las expresiones siguientes:
      \begin{parts}
        \part Carlos es padre de Ana
          \begin{solution}
            $H(c) \land P(c, a)$
          \end{solution}
        \part María es la madre de Carlos
          \begin{solution}
            $\lnot H(m) \land P(m, c)$
          \end{solution}
        \part Ana tiene una madre
          \begin{solution}
            $\exists x. \lnot H(x) \land P(x, a)$
          \end{solution}
        \part Luis es abuelo (materno o paterno) de Ana
          \begin{solution}
            $H(l) \land \exists x. P(l, x) \land P(x, a)$
          \end{solution}
        \part Todos tienen una madre
          \begin{solution}
            $\forall x. \exists y. \lnot H(y) \land P(y, x)$
          \end{solution}
        \part Hay alguien que tiene un hermano (o hermanastro)
          \begin{solution}
            $\exists x. \exists y. \exists z. P(z, x) \land P(z, y)$
          \end{solution}
        \part Nadie tiene dos padres
          \begin{solution}
            $\nexists x. \exists y. \exists z. P(y, x) \land P(z, x)$
          \end{solution}
        \part Nadie es progenitor de todos
          \begin{solution}
            $\nexists x. \forall y.P(x, y)$
          \end{solution}
      \end{parts}

  \question
    Las siguientes reglas rigen a todos los sistemas planetarios del universo.

    \begin{itemize}
      \item Todo elemento es o bien una estrella, o bien un planeta o bien un satélite.
      \item Si un elemento es un satélite entonces no es ni una estrella ni un planeta.
      \item Si un elemento es un planeta entonces no es ni una estrella ni un satélite.
      \item Si un elemento es una estrella entonces no es ni un planeta ni un satélite.
      \item Todo elemento que no sea una estrella orbíta a otro elemento.
      \item Todo elemento orbitado por planetas es una estrella.
      \item Todo elemento que orbíte un planeta es un satélite.
      \item Ningún elemento se orbíta a si mismo.
    \end{itemize}

    Sabiendo esto, y considerando todas las siguientes proposiciones como verdaderas,
    se pide complete las propiedades y relaciones de un sistema planetario descubierto
    en la galaxia M31.

    \begin{itemize}
      \item Existe un solo elemento que es una estrella.
      \item Todos los planetas orbítan alrededor de Apolo.
      \item Afrodita orbíta a Hermes o es un planeta.
      \item Gaia orbíta a Apolo.
      \item Selene orbíta a Gaia.
      \item Hermes no es un satélite ni una estrella.
      \item Timor no es un planeta.
      \item Nadie orbíta a Hermes.
      \item Existen dos elementos que orbítan a Ares.
      \item Metus es un satélite y no orbÍta ni a Gaia ni a Afrodita.
    \end{itemize}

    \centerline{\begin{tabular}{| l | c | c | c |}
      \hline
      & $x$ es un planeta & $x$ es un satélite & $x$ es una estrella \\
      \hline
      Hermes &&& \\
      \hline
      Afrodita &&& \\
      \hline
      Gaia &&& \\
      \hline
      Ares &&& \\
      \hline
      Selene &&& \\
      \hline
      Timor &&& \\
      \hline
      Metus &&& \\
      \hline
      Apolo &&& \\
      \hline
    \end{tabular}}

    \centerline{\begin{tabular}{| l | c | c | c | c | c | c | c | c |}
      \hline
      $x$ orbíta a $y$ & Hermes & Afrodita & Gaia & Ares & Selene & Timor & Metus & Apolo \\
      \hline
      Hermes &&&&&&&& \\
      \hline
      Afrodita &&&&&&&& \\
      \hline
      Gaia &&&&&&&& \\
      \hline
      Ares &&&&&&&& \\
      \hline
      Selene &&&&&&&& \\
      \hline
      Timor &&&&&&&& \\
      \hline
      Metus &&&&&&&& \\
      \hline
      Apolo &&&&&&&& \\
      \hline
    \end{tabular}}

    \begin{solution}
      \centerline{\begin{tabular}{| l | c | c | c |}
        \hline
        & $x$ es un planeta & $x$ es un satélite & $x$ es una estrella \\
        \hline
        Hermes & \true & \false & \false \\
        \hline
        Afrodita & \true & \false & \false \\
        \hline
        Gaia & \true & \false & \false \\
        \hline
        Ares & \true & \false & \false \\
        \hline
        Selene & \false & \true & \false \\
        \hline
        Timor & \false & \true & \false \\
        \hline
        Metus & \false & \true & \false \\
        \hline
        Apolo & \false & \false & \true \\
        \hline
      \end{tabular}}
  
      \centerline{\begin{tabular}{| l | c | c | c | c | c | c | c | c |}
        \hline
        $x$ orbíta a $y$ & Hermes & Afrodita & Gaia & Ares & Selene & Timor & Metus & Apolo \\
        \hline
        Hermes & \false & \false & \false & \false & \false & \false & \false & \true \\
        \hline
        Afrodita & \false & \false & \false & \false & \false & \false & \false & \true \\
        \hline
        Gaia & \false & \false & \false & \false & \false & \false & \false & \true \\
        \hline
        Ares & \false & \false & \false & \false & \false & \false & \false & \true \\
        \hline
        Selene & \false & \false & \true & \false & \false & \false & \false & \false \\
        \hline
        Timor & \false & \false & \false & \true & \false & \false & \false & \false \\
        \hline
        Metus & \false & \false & \false & \true & \false & \false & \false & \false \\
        \hline
        Apolo & \false & \false & \false & \false & \false & \false & \false & \false \\
        \hline
      \end{tabular}}
    \end{solution}

  \end{questions}
\end{document}
