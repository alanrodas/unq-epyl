% !TeX encoding = UTF-8
% !TeX spellcheck = es_AR
% arara: pdflatex
% arara: pdflatex
% arara: pdflatex
% arara: clean: {files: [01_conectivas.aux, 01_conectivas.out, 01_conectivas.log]}
% arara: clean: {files: [01_conectivas.fdb_latexmk, 01_conectivas.fls, 01_conectivas.vrb, 01_conectivas.synctex.gz]}

%% Add the word "answers" to document class parameters in order to print the solutions.
\documentclass[12pt, addpoints]{../../common/epyl_exam_template}

\title{Práctica 2.1 - Conectivas y Tablas de Verdad}
\date{2018.1}

\begin{document}
\makeexamheader
\makeexamtitle

\examrule

\begin{questions}

  \question
    Dados los siguientes razonamientos, indique si se tratan de razonamientos
    deductivos, razonamientos inductivos por analogía o razonamientos
    inductivos por enumeración.
    \begin{parts}
      \part~\\
        Dado que las versiones más nuevas de los navegadores soportan HTML5,
        y dado que el Microsoft Edge es la última versión del navegador de dicha
        empresa, se sigue que el Microsoft Edge soporta HTML5.
        \begin{solution} Razonamiento deductivo \end{solution}

      \part~\\
        Mi primo compró una computadora aquí y tenía instalado Windows 7.\\
        Mi hermano compró una computadora aquí y tenía instalado Windows 7.\\
        Mi tío compró una computadora aquí y tenía instalado Windows 7.\\
        Si compro una computadora aquí tendrá instalado Windows 7.
        \begin{solution} Razonamiento inductivo por analogía
          % TODO El ejemplo es bastante choto para indicar analogía, porque la
          % conclusion real es que toda computadora que se compre ahí va a
          % tener Windows 7. Hay que cambiar el ejercicio.
        \end{solution}

      \part~\\
        Maria se inscribió el jueves y solo tenía habilitadas las materias del CI.\\
        Pablo se inscribió el jueves y solo tenía habilitadas las materias del CI.\\
        Juan se inscribió el jueves y solo tenía habilitadas las materias del CI.\\
        Por lo tanto, todo alumno que se inscriba el jueves tiene solo habilitadas las materias del CI.
        \begin{solution} Razonamiento inductivo por enumeración \end{solution}
    \end{parts}

  \jump
  \question
    Considerando las siguientes preguntas:\\
    ~\\
    \begin{minipage}{0.46\textwidth}
      \begin{itemize}
        \item ¿Hay harina?
        \item ¿Hay manteca?
        \item ¿Hay aceite?
        \item ¿Hay agua?
      \end{itemize}
    \end{minipage}
    \begin{minipage}{0.46\textwidth}
      \begin{itemize}
        \item ¿Hay huevos?
        \item ¿Hay yerba?
        \item ¿Hay chocolate?
      \end{itemize}
    \end{minipage}
    ~\\
    Se le pide que exprese las preguntas a continuación utilizando dichas preguntas
    como básicas y utilizando conectivas lógicas:
    \begin{parts}
      \part \textbf{¿Hay para hacer una torta?} (Una torta requiere harina, huevos y manteca)
      \begin{solution} ¿Hay harina? y ¿Hay huevos? y ¿Hay manteca? \end{solution}
      \part \textbf{¿Hay para hacer huevos fritos?} (Requiere aceite y huevos)
      \begin{solution} ¿Hay huevos? y ¿Hay aceite? \end{solution}
      \part \textbf{¿Hay para hacer huevos duros?} (Requiere huevos y agua)
      \begin{solution} ¿Hay huevos? y ¿Hay agua? \end{solution}
      \part \textbf{¿Puedo almorzar huevos?} (Ya sean duros o fritos)
      \begin{solution}¿Hay para hacer huevos fritos? o ¿Hay para hacer huevos duros? \end{solution}
      \part \textbf{¿Hay para hacer una torta de chocolate?} (Lo mismo que una torta, pero además requiere chocolate)
      \begin{solution} ¿Hay para hacer una torta? y ¿Hay chocolate? \end{solution}
      \part \textbf{¿Solo se puede tomar té?} (Cuando no se puede tomar mate)
      \begin{solution}
        ¿Hay mate? = ¿Hay yerba? y ¿Hay agua?

        ¿Solo se puede tomar té? = ¿Es cierto que no hay mate?
      \end{solution}
      \part \textbf{¿No hay nada para el mate?} (Cuando se puede tomar mate pero no hay torta de ningún tipo)
      \begin{solution}
        ¿Hay mate? y ¿Es cierto que no hay para hacer una torta de chocolate? y ¿Es cierto que no hay para hacer una torta?

        \textbf{Aunque, como si no hay para una torta común, tampoco hay para una de chocolate, basta con:}

        ¿Hay mate? y ¿Es cierto que no hay para hacer una torta?
      \end{solution}
    \end{parts}
  
  \jump
  \question
    Considere ahora las siguientes preguntas y sus respuestas:
    \begin{itemize}
      \item ¿Plutón es un planeta? \fullfalse
      \item ¿Marte es un planeta? \fulltrue
      \item ¿Deimos es un satélite? \fulltrue
      \item ¿Ganímedes es un satélite? \fulltrue
      \item ¿Eros es un satélite? \fullfalse
    \end{itemize}
    ~\\
    Se pide exprese las siguientes preguntas utilizando las preguntas anteriores
    y las conectivas vistas, y determine el valor de verdad de cada una de ellas.
    \begin{parts}
      \part ¿Son Marte y Plutón planetas?
      \begin{solution} \fullfalse \end{solution}
      \part Es Marte un planeta o es Plutón un planeta?
      \begin{solution} \fulltrue \end{solution}
      \part ¿Es cierto que Marte es un planeta y Plutón no lo es?
      \begin{solution} \fulltrue \end{solution}
      \part ¿Es cierto que Ganímedes y Eros son satélites?
      \begin{solution} \fullfalse \end{solution}
      \part ¿Es cierto que Eros no es un satélite, pero Deimos si lo es?
      \begin{solution} \fulltrue \end{solution}
      \part ¿Es Deimos un satélite o es cierto qué Marte es un planeta?
      \begin{solution} \fulltrue \end{solution}
    \end{parts}

  \jump
  \question
    Sabiendo que las siguientes preguntas evalúan todas a \fulltrue\\
    ~\\
    \begin{itemize}
      \item ¿La palmera está creciendo torcida? y ¿Es cierto que el árbol no dio paltas este año?
      \item ¿Es cierto que no hay flores en el cantero? o ¿El árbol dio paltas este año?
    \end{itemize}
    ~\\
    Se pide que responda las siguientes preguntas
    \begin{parts}
      \part ¿El árbol dio paltas este año?
      \part ¿Hay flores en el cantero?
      \part ¿La palmera está creciendo torcida?
    \end{parts}
    \begin{solution}
      Si la primer frase es verdadera, como hay un \textbf{y} en el medio,
      ambas partes deben ser verdaderas. O sea que sabemos que las respuestas
      a estas dos preguntas son verdaderas:
      \begin{itemize}
        \item ¿La palmera está creciendo torcida?
        \item ¿Es cierto que el árbol no dio paltas este año?
      \end{itemize}

      Por lo que, eliminando la negación de la segunda pregunta averiguamos que:
      
      \begin{enumerate}
        \item ¿El árbol dio paltas este año? \fullfalse
        \item ¿Hay flores en el cantero?
        \item ¿La palmera está creciendo torcida? \fulltrue
      \end{enumerate}

      Analizamos ahora la segunda pregunta. Como hay un \textbf{o}, sabemos que
      alguna de las dos partes es verdadera. Pero ya sabemos que
      \textbf{¿El árbol dio paltas este año?} es \fullfalse, por lo que la que debe
      ser verdadera es la otra parte. Eliminamos la negación y obtenemos:

      \begin{enumerate}
        \item ¿El árbol dio paltas este año? \fullfalse
        \item ¿Hay flores en el cantero? \fullfalse
        \item ¿La palmera está creciendo torcida? \fulltrue
      \end{enumerate}
    \end{solution}

  \jump
  \question
    Considere las siguientes preguntas:
    \begin{itemize}
      \item ¿Se cumple P?
      \item ¿Se cumple Q?
      \item ¿Se cumple R?
    \end{itemize}
    Se pide analice que valuaciones dan \fulltrue y cuales \fullfalse en las
    siguientes preguntas. Determine en qué casos se está hablando de una
    tautología, en casos de una contradicción y en cuáles de contingencias.\footnote{
      Si pensar en si se cumple P, Q y R lo marea, remplace esas condiciones por
      cosas que le resulten más familiares, como ``¿Se cumple que el lobo vive en el bosque?''.
      La gracia del ejercicio es que la pregunta no es realmente importante, sino
      lo que podemos obtener de ellas.
    }
    \begin{parts}
      \part ¿Se cumple P? y ¿No se cumple P?
      \begin{solution}
        \begin{tabular}{c | c || c}
          ¿Se cumple P? & ¿No se cumple P? & ¿Se cumple P? y ¿No se cumple P? \\
          \hline
          &&\\
          \fulltrue & \fullfalse & \fullfalse\\
          \fullfalse & \fulltrue & \fullfalse\\
        \end{tabular}
        \\Se trata de una contradicción
      \end{solution}
      \part ¿Se cumple P? o ¿No se cumple P?
      \begin{solution}
        \begin{tabular}{c | c || c}
          ¿Se cumple P? & ¿No se cumple P? & ¿Se cumple P? o ¿No se cumple P? \\
          \hline
          &&\\
          \fulltrue & \fullfalse & \fulltrue\\
          \fullfalse & \fulltrue & \fulltrue\\
        \end{tabular}
        \\Se trata de una tautología
      \end{solution}
      \part ¿Se cumple P? y ¿Se cumple Q?
      \begin{solution}
        \begin{tabular}{c | c || c}
          ¿P? & ¿Q? & ¿P y Q? \\
          \hline
          &&\\
          \fulltrue  & \fulltrue  & \fulltrue  \\
          \fulltrue  & \fullfalse & \fullfalse \\
          \fullfalse & \fulltrue  & \fullfalse \\
          \fullfalse & \fullfalse & \fullfalse \\
        \end{tabular}
        \\Se trata de una contingencia
      \end{solution}
      \part ¿Se cumple P? y ¿Se cumple P o se cumple Q?
      \begin{solution}
        \begin{tabular}{c | c | c || c}
          ¿P? & ¿Q? & ¿P o Q? & ¿P? y ¿P o Q?\\
          \hline
          &&\\
          \fulltrue  & \fulltrue  & \fulltrue  & \fulltrue  \\
          \fulltrue  & \fullfalse & \fulltrue  & \fullfalse \\
          \fullfalse & \fulltrue  & \fulltrue  & \fullfalse \\
          \fullfalse & \fullfalse & \fullfalse & \fullfalse \\
        \end{tabular}
        \\Se trata de una contingencia, que es igual a la anterior.
      \end{solution}
      \part ¿Se cumple P? o ¿Se cumple Q?
      \begin{solution}
        \begin{tabular}{c | c || c}
          ¿P? & ¿Q? & ¿P o Q? \\
          \hline
          &&\\
          \fulltrue  & \fulltrue  & \fulltrue  \\
          \fulltrue  & \fullfalse & \fulltrue  \\
          \fullfalse & \fulltrue  & \fulltrue  \\
          \fullfalse & \fullfalse & \fullfalse \\
        \end{tabular}
        \\Se trata de una contingencia
      \end{solution}
      \part ¿No se cumple P? o ¿Se cumple Q y se cumple P?
      \begin{solution}
        \begin{tabular}{c | c | c || c}
          ¿P? & ¿Q? & ¿Q y P? & ¿P? o ¿Q y P? \\
          \hline
          &&\\
          \fulltrue  & \fulltrue  & \fulltrue  & \fulltrue  \\
          \fulltrue  & \fullfalse & \fullfalse & \fulltrue  \\
          \fullfalse & \fulltrue  & \fullfalse & \fulltrue  \\
          \fullfalse & \fullfalse & \fullfalse & \fullfalse  \\
        \end{tabular}
        \\Se trata de una contingencia, nuevamente igual a la anterior
      \end{solution}
      \part ¿Se cumple P? y ¿Se cumple Q o se cumple R?
      \begin{solution}
        \begin{tabular}{c | c | c | c || c}
          ¿P? & ¿Q? & ¿R? & ¿P y Q? & ¿P y Q? o ¿R? \\
          \hline
          &&&\\
          \true  & \true  & \true  & \true  & \true  \\
          \true  & \true  & \false & \true  & \true  \\
          \true  & \false & \true  & \false & \true  \\
          \true  & \false & \false & \false & \false \\
          \false & \true  & \true  & \false & \true  \\
          \false & \true  & \false & \false & \false \\
          \false & \false & \true  & \false & \true  \\
          \false & \false & \false & \false & \false \\
        \end{tabular}
        \\Se trata de una contingencia, nuevamente igual a la anterior
      \end{solution}
    \end{parts}

  \jump
  \question
    Considerando las siguientes preguntas:\\
    ~\\
    \begin{minipage}{0.46\textwidth}
      \begin{itemize}
        \item ¿Hay ejercito enemigo al norte?
        \item ¿Hay ejercito enemigo al este?
        \item ¿Hay ejercito enemigo al sur?
        \item ¿Hay ejercito enemigo al oeste?
      \end{itemize}
    \end{minipage}
    \begin{minipage}{0.46\textwidth}
      \begin{itemize}
        \item ¿Hay ejercito aliado al norte?
        \item ¿Hay ejercito aliado al este?
        \item ¿Hay ejercito aliado al sur?
        \item ¿Hay ejercito aliado al oeste?
      \end{itemize}
    \end{minipage}
    \jump
    Se le pide que exprese las preguntas a continuación utilizando dichas preguntas
    como básicas y utilizando conectivas lógicas. Puede desarrollar preguntas
    auxiliares que le ayuden a simplificar las preguntas a resolver.
    \begin{parts}
      \part \textbf{¿Se está amenazado?} (Cuando hay ejercito enemigo en alguna dirección)
      \begin{solution}
        ¿Hay ejercito enemigo al norte? o\\
        ¿Hay ejercito enemigo al este? o\\
        ¿Hay ejercito enemigo al sur? o\\
        ¿Hay ejercito enemigo al oeste?
      \end{solution}
      \part \textbf{¿Se está libre de peligro?} (Cuando no hay ejércitos enemigos en ninguna dirección)
      \begin{solution}
        ¿Es cierto que no hay ejercito enemigo al norte? y\\
        ¿Es cierto que no hay ejercito enemigo al este? y\\
        ¿Es cierto que no hay ejercito enemigo al sur? y\\
        ¿Es cierto que no hay ejercito enemigo al oeste?

        \textbf{Aunque estar libre de peligro, es lo mismo que no estar amenazado, por lo
        que basta con:}

        ¿Es cierto que no se está amenazado?
      \end{solution}
      \part \textbf{¿Se tiene apoyo?} (Cuando hay algún ejercito aliado en alguna dirección)
      \begin{solution}
        ¿Hay ejercito aliado al norte? o\\
        ¿Hay ejercito aliado al este? o\\
        ¿Hay ejercito aliado al sur? o\\
        ¿Hay ejercito aliado al oeste?
      \end{solution}
      \part \textbf{¿Se está hasta las manos?} (Cuando no hay apoyo y se está amenazado)
      \begin{solution}
        ¿Es cierto que no se tiene apoyo? y ¿Se está amenazado?
      \end{solution}
      \part \textbf{¿Se puede neutralizar alguna amenaza?} (Cuando hay un ejercito enemigo en alguna dirección, pero también hay un ejercito aliado allí)
      \begin{solution}
        ¿Se puede neutralizar al norte? =\\
        \quad ¿Hay ejercito enemigo al norte? y ¿Hay ejercito aliado al norte?\\
        ¿Se puede neutralizar al este? =\\
        \quad ¿Hay ejercito enemigo al este? y ¿Hay ejercito aliado al este?\\
        ¿Se puede neutralizar al sur? =\\
        \quad ¿Hay ejercito enemigo al sur? y ¿Hay ejercito aliado al sur?\\
        ¿Se puede neutralizar al oeste? =\\
        \quad ¿Hay ejercito enemigo al oeste? y ¿Hay ejercito aliado al oeste?

        ¿Se puede neutralizar al norte? o ¿Se puede neutralizar al este? o ¿Se puede neutralizar al sur? o ¿Se puede neutralizar al oeste?
      \end{solution}
      \part \textbf{¿Se puede neutralizar todas las amenazas?} (Cuando, si hay un ejercito enemigo en alguna dirección, necesariamente también hay un ejercito aliado allí)
      \begin{solution}
        ¿Se puede neutralizar al norte? y\\
        ¿Se puede neutralizar al este? y\\
        ¿Se puede neutralizar al sur? y\\
        ¿Se puede neutralizar al oeste?
      \end{solution}
    \end{parts}

  \question
    Dadas las siguientes preguntas y sus respuestas:\\
    ~\\
    \begin{itemize}
      \item ¿Es Willy Willy un Toki Toki? \fulltrue
      \item ¿Es Buggy Buggy un Chiri Biri? \fullfalse
      \item ¿Es Cuqui Cuqui un Tele Tubi? \fullfalse
      \item ¿Es Mochi Mochi un Tutsi Tutsi? \fulltrue
    \end{itemize}
    ~\\
    Responda si las siguientes preguntas son verdaderas o falsas \footnote{
      Nuevamente, olvide las preguntas en si, y si le cuesta pensar en
      Willy Willys y Mochi Mochis, remplace las preguntas por algunas que
      le resulten más cómodas.
    }
    \begin{parts}
      \part ¿Es cierto que Willy Willy no es un Toki Toki? o ¿Es cierto que Cuqui Cuqui no es un Tele Tubi?
      \begin{solution} \fulltrue \end{solution}
      \part ¿Es cierto que Buggy Buggy no es un Chiri Biri? y ¿Es Mochi Mochi un Tutsi Tutsi y Willy Willy un Toki Toki? 
      \begin{solution} \fulltrue \end{solution}
      \part ¿Es cierto que no es cierto que Willy Willy no es un Toki Toki?
      \begin{solution} \fulltrue \end{solution}
      \part ¿Es cierto que Willy Willy no es un Toki Toki o bien Cuqui Cuqui es un Tele Tubi?
      \begin{solution} \fullfalse \end{solution}
    \end{parts}
  
  \jump
  \question
    Se ha encontrado vida en otro planeta, y se ha decidido nombrar a los animales
    encontrados como ``Woofle'', ``Brlfks'' y ``Morlock''.\\
    Cada animal tiene sus características distintivas (pueden ser grandes, chicos,
    con o sin pelo, acuáticos o terrestres, etc.).\\
    Si las respuestas a todas las preguntas siguientes son \fulltrue, enumere que
    características tiene cada animal. Para ello, realice la tabla de verdad de
    cada pregunta y analice las respuestas a las preguntas base en las valuaciones
    verdaderas.
    ~\\
    \begin{parts}
      \part ¿El Woofle tiene pelo? y ¿El Woofle no es acuático?
      \part ¿El Morlock es terrestre y es grande? y ¿El Brlfks es acuático?
      \part ¿El Brlfks es pequeño? o ¿El Morlock es pequeño?
      \part ¿El Woofle tiene pelo y el Brlfks no? o ¿El Brlfks no tiene pelo y el Morlock tiene pelo?
      \part ¿El Woofle no es grande o el Brlfks es grande? y ¿El Morlock no es pequeño?
    \end{parts}
    \begin{solution}
      La resolución es similar al ejercicio de la palta, las flores y la palmera,
      pero ahora hay muchos más casos a considerar. Lo ideal es plantear el ejercicio
      con una tabla de la forma:
      
      \begin{tabular}{| l | c | c | c |}
        \hline
          Animal & Pelo & Terrestre & Grande \\
        \hline
          Brlfks &&& \\
        \hline
          Morlock &&& \\
        \hline
          Woofle &&& \\
        \hline
      \end{tabular}

      Vamos analizando luego cada oración y marcamos con verdadero o falso donde
      corresponda. Por ejemplo, de la primera desprendemos que el Woofle tiene pelo,
      y no es acuático (o sea, es terrestre), pues hay un \textbf{y} separando
      ambas preguntas, por lo que ambas deben ser \fulltrue.

      La tabla nos queda:

      \begin{tabular}{| l | c | c | c |}
        \hline
          Animal & Pelo & Terrestre & Grande \\
        \hline
          Brlfks &&& \\
        \hline
          Morlock &&& \\
        \hline
          Woofle & \fulltrue & \fulltrue & \\
        \hline
      \end{tabular}

      Seguimos analizando las otras preguntas y completando la tabla con la
      misma idea. Eventualmente llegaremos a que:

      Morlock: no tiene pelo, es terrestre y grande\\
      Woofle: tiene pelo, es terrestre y pequeño\\
      Brlfks: no tiene pelo, es acuático y pequeño
    \end{solution}

\end{questions}
\end{document}
