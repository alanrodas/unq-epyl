% !TeX encoding = UTF-8
% !TeX spellcheck = es_AR
%% Add the word "answers" to document class parameters in order to print the solutions.
\documentclass[10pt, addpoints]{../../../common/epyl_exam_template}

\title{Primer Parcial - Lógica}
\date{2018.1}
\professor{Alan Rodas Bonjour}
\timelimit{2 hs}
\topic{Lógica}
\instance{Parcialito}

\renewcommand\examlogo{\includegraphics[scale=0.07]{../../../common/unq_logo.jpg}}

\begin{document}
\makeexamheader
\makeexamtitle
\makeexamextradata

\begin{questions}
  \question
  Considerando las siguientes proposiciones como base:\\
  ~\\
    \begin{itemize}
      \item En 1215 el Imperio Mongól llegó a su máxima expansión
      \item En 1290 el Imperio Mongól llegó a su máxima expansión
      \item Gengis es padre de Tolui
      \item Tolui es padre de Kublai
      \item Gengis fue un Gran Khan del Imperio Mongól
      \item Kublai fue un Gran Khan del Imperio Mongól
      \item Gengis gobernó desde 1206 al 1227
      \item Kublai gobernó desde el 1260 al 1294
      \item Gengis abolió la esclavitud en el Imperio Mongól
      \item Kublai abolió la esclavitud en el Imperio Mongól
    \end{itemize}
  ~\\
  Se le pide que exprese las expresiones a continuación en base a las anteriores:
  \begin{parts}
      \part \textbf{Kublai es nieto de Gengis}
      \begin{solution}
        Gengis es padre de Tolui $\land$\\
        Tolui es padre de Kublai
      \end{solution}
      \part \textbf{Kublai llevó el imperio a su máxima expansión}
      \begin{solution}
        Kublai fue un Gran Khan del Imperio Mongól $\land$\\
        Kublai gobernó desde el 1260 al 1294 $\land$\\
        En 1290 el Imperio Mongól llegó a su máxima expansión
      \end{solution}
      \part \textbf{Un Gran Khan abolió la esclavitud en el Imperio Mongól}
      \begin{solution}
        (Gengis fue un Gran Khan del Imperio Mongól $\land$ Gengis abolió la esclavitud el Imperio Mongól)\\
        $\lor$\\
        (Kublai fue un Gran Khan del Imperio Mongól $\land$ Kublai abolió la esclavitud en el Imperio Mongól)\\
      \end{solution}
      \part \textbf{Kublai tuvo un abuelo fracasado} (pues no llevo al imperio a su máxima expansión)
      \begin{solution}
        Kublai es nieto de Gengis $\land$\\
        Gengis gobernó desde 1206 al 1227 $\land$\\
        $\lnot$ En 1215 el Imperio Mongól llegó a su máxima expansión
      \end{solution}
  \end{parts}

  \jump

  \question
  Pase los siguientes razonamientos del lenguaje natural al formal de la lógica proposicional.
  \begin{parts}
    \part~\\
    O bien no chequeamos todas las condiciones o bien no respetamos la sintaxis.
    Esto es así dado que si respetábamos la sintaxis y chequeábamos todas las
    condiciones, entonces el código se ejecutaría correctamente. Sin embargo, el
    código no se ejecutó correctamente.
    \begin{solution}
        IP = Por lo tanto\\
        p = El código se ejecutó correctamente\\
        q = Chequeamos todas las condiciones\\
        r = Respetamos la sintaxis\\
        $(q \land r) \lthen p, \lnot p \lseq (\lnot q) \lor (\lnot r)$
    \end{solution}

    \part~\\
    Deberíamos confiar en el líder del equipo si y solo si el líder da
    señales de que el proyecto será un éxito. El líder solo dará señales
    de que el proyecto será un éxito si hay suficiente dinero. Pero no hay
    suficiente dinero. Por lo que no deberíamos confiar en el líder.
    \begin{solution}
        IC = \\
        p = Deberíamos confiar en el líder\\
        q = El proyecto será un éxito\\
        r = Hay suficiente dinero\\
        $p \liff q, q \liff r, \lnot r \lseq p$
    \end{solution}
  \end{parts}
  
  \jump
  \question
  Dadas las formulas de los siguientes razonamientos, se pide que pruebe si son
  válidos o inválidos.
  \begin{parts}
      \part $p \lthen q, q \lthen r, \lnot p \lseq \lnot r$
      \begin{solution} \fullfalse \end{solution}
      \part $(\lnot p \land \lnot q) \lthen \lnot r, r \lseq p \lor q$
      \begin{solution} \fulltrue \end{solution}
  \end{parts}
  \jump
  \question
  Sabiendo que las siguientes expresiones sobre estados del mundo evalúan
  todas a \fulltrue, se pide que complete las tablas a continuación:
  ~\\
  \begin{itemize}
    \item Todos los estados son capitalistas o comunistas.
    \item Nadie que sea comunista es capitalista.
    \item Nadie que sea capitalista es comunista.
    \item Todos los capitalistas quieren destruir a los comunistas.
    \item Dracma quiere destruir a los capitalistas.
    \item O Creta es comunista o Aerugo es comunista.
    \item Si Aerugo es comunista, entonces Xing lo es.
    \item Todos los comunistas quieren destruir a Aerugo.
    \item Amestris se quiere destruir a si misma.
    \item Nadie más quiere destruir a nadie.
  \end{itemize}
  ~\\
  \begin{tabular}{| c | c | c | c |}
    \hline
      & $x$ es comunista & $x$ es capitalista \\
    \hline
    Xing     & \false &        \\
    \hline
    Amestris &        & \false \\
    \hline
    Creta    &        &        \\
    \hline
    Aerugo   &        &        \\
    \hline
    Dracma   & \true  &        \\
    \hline
  \end{tabular}

  \begin{tabular}{| c | c | c | c | c | c |}
    \hline
$x$ quieren destruir a $y$ & Xing & Amestris & Creta & Aerugo & Dracma \\
    \hline
    Xing     &       &       &       &      &       \\
    \hline
    Amestris &       &       &       &      &       \\
    \hline
    Creta    &       &       &       &      &       \\
    \hline
    Aerugo   &       &       &       &      &       \\
    \hline
    Dracma   &       &       &       &      &       \\
    \hline
  \end{tabular}
  \begin{solution}
    \begin{tabular}{| c | c | c | c |}
      \hline
        & $x$ es comunista & $x$ es capitalista \\
      \hline
      Xing     & \false & \true  \\
      \hline
      Amestris & \true  & \false \\
      \hline
      Creta    & \true  & \false \\
      \hline
      Aerugo   & \false & \true  \\
      \hline
      Dracma   & \true  & \false \\
      \hline
    \end{tabular}

    \begin{tabular}{| c | c | c | c | c | c |}
      \hline
  $x$ quieren destruir a $y$ & Xing & Amestris & Creta & Aerugo & Dracma \\
      \hline
      Xing     & \false & \true  & \true & \false  & \true \\
      \hline
      Amestris & \true  & \false & \true & \false  & \false \\
      \hline
      Creta    & \false & \false & \false  & \true & \false \\
      \hline
      Aerugo   & \false & \true  & \true & \false  & \true  \\
      \hline
      Dracma   & \true & \false  & \false  & \true & \false \\
      \hline
    \end{tabular}
  \end{solution}

  \question
  Los conjuntos son un objeto matemático que consiste en una colección
  de elementos de algún tipo. Por ejemplo, podemos pensar en un conjunto
  de números (ej. $\{1,2,3,4\}$, es el conjunto que incluye a los números
  1, 2, 3 y 4, y ninguno más). Un conjunto, puede estar vacío (es un
  conjunto que no contiene ningún elemento).

  Considere los conjuntos \textbf{$U$} y \textbf{$C$}, \textbf{$J$} y los predicados \textbf{$x$ es un conjunto},
  \textbf{$x$ es un número} y \textbf{$x$ está incluido en $y$}.

  Se pide que desarrolle el diccionario que considere conveniente, y elabore
  fórmulas para expresar las siguientes afirmaciones.
  \begin{parts}
    \part El conjunto $C$ es vacío.
    \begin{solution}
      $c = C$\\
      $u = U$\\
      $j = J$\\
      $C(x) = x$ es un conjunto\\
      $R(x) = x$ es un número\\
      $I(x, y) = x$ está incluido en $y$\\
      ~\\
      $\nexists z.I(z, c)$
    \end{solution}
    \part El conjunto $U$ contiene a todos los números.
    \begin{solution}
      $\forall z.R(z) \lthen I(z, u)$
    \end{solution}
    \part El conjunto $J$ tiene a todos los números negativos.
    \begin{solution}
      $N(x) = x < 0$\\
      ~\\
      $\forall z.R(x) \land N(x) \lthen I(z, j)$
    \end{solution}
    \part Hay al menos un conjunto que se contiene a si mismo.
    \begin{solution}
      $\exists a.C(a) \land I(a, a)$
    \end{solution}
  \end{parts}

\end{questions}
\end{document}
    