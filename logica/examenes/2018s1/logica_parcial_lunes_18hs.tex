% !TeX encoding = UTF-8
% !TeX spellcheck = es_AR
%% Add the word "answers" to document class parameters in order to print the solutions.
\documentclass[10pt, addpoints]{../../../common/epyl_exam_template}

\title{Primer Parcial - Lógica}
\date{2018.1}
\professor{Alan Rodas Bonjour}
\timelimit{2 hs}
\topic{Lógica}
\instance{Parcialito}

\renewcommand\examlogo{\includegraphics[scale=0.07]{../../../common/unq_logo.jpg}}

\begin{document}
\makeexamheader
\makeexamtitle
\makeexamextradata

\begin{questions}
  \question
  Considerando las siguientes proposiciones como base:\\
  ~\\
    \begin{itemize}
      \item Gengis fue un Gran Khan del Imperio Mongól
      \item Kublai fue un Gran Khan del Imperio Mongól
      \item Gengis gobernó desde 1206 al 1227
      \item Kublai gobernó desde el 1260 al 1294
      \item Tolui es hijo de Gengis
      \item Kublai es hijo de Tolui
      \item En 1290 el Imperio Mongól llegó a su máxima expansión
      \item Gengis abolió la esclavitud en el Imperio Mongól
      \item Kublai abolió la esclavitud en el Imperio Mongól
    \end{itemize}
  ~\\
  Se le pide que exprese las expresiones a continuación en base a las anteriores:
  \begin{parts}
      \part \textbf{Kublai es nieto de Gengis}
      \begin{solution}
        Kublai es hijo de Tolui $\land$\\
        Tolui es hijo de Gengis
      \end{solution}
      \part \textbf{Kublai llevó el imperio a su máxima expansión}
      \begin{solution}
        Kublai fue un Gran Khan del Imperio Mongól $\land$\\
        Kublai gobernó desde el 1260 al 1294 $\land$\\
        En 1290 el Imperio Mongól llegó a su máxima expansión
      \end{solution}
      \part \textbf{Un Gran Khan abolió la esclavitud en el Imperio Mongól}
      \begin{solution}
        (Gengis fue un Gran Khan del Imperio Mongól $\land$ Gengis abolió la esclavitud el Imperio Mongól)\\
        $\lor$\\
        (Kublai fue un Gran Khan del Imperio Mongól $\land$ Kublai abolió la esclavitud en el Imperio Mongól)\\
      \end{solution}
      \part \textbf{Hubo esclavos en el Imperio Mongol desde 1206 a 1294}
      \begin{solution}
        Gengis gobernó desde 1206 al 1227 $\land$\\
        Kublai gobernó desde el 1260 al 1294 $\land$\\
        $\lnot$ Un Gran Khan abolió la esclavitud en el Imperio Mongól
      \end{solution}
  \end{parts}

  \jump

  \question
  Pase los siguientes razonamientos del lenguaje natural al formal de la lógica proposicional.
  \begin{parts}
    \part~\\
    Solo nos convertiremos en unicornio si los inversionistas ponen más dinero.
    Los inversionistas solo pondrán dinero si conseguimos más usuarios.
    Pero no podemos conseguir más usuarios. Por lo tanto, no nos convertiremos
    en unicornio.
    \begin{solution}
        IC = Por lo tanto\\
        p = Nos convertiremos en unicornio\\
        q = Los inversionistas ponen más dinero\\
        r = Conseguimos más usuarios\\
        $p \liff q, q \liff r, \lnot r \lseq \lnot p$
    \end{solution}

    \part~\\
    La lista está vacía. Dado que si la lista no estuviera vacía, el
    código habría entrado al condicional y ejecutado el comando
    para imprimir. Y no ejecutó el comando para imprimir.
    \begin{solution}
        IP = Dado que\\
        p = La lista está vacía\\
        q = El código entra al condicional\\
        r = Se ejecutó el comando para imprimir\\
        $\lnot p \lthen (q \land r), \lnot r \lseq p$
    \end{solution}
  \end{parts}
  
  \jump
  \question
  Dadas las formulas de los siguientes razonamientos, se pide que pruebe si son
  razonamientos válidos o inválidos.
  \begin{parts}
      \part $p \lthen q, q \lthen p \lseq q \liff p$
      \begin{solution} \fulltrue \end{solution}
      \part $(p \land q) \lthen \lnot r, r \lseq (\lnot p) \lor (\lnot q)$
      \begin{solution} \fulltrue \end{solution}
  \end{parts}
  \jump
  \question
  Sabiendo que las siguientes expresiones sobre estados del mundo evalúan
  todas a \fulltrue, se pide que complete las tablas a continuación:
  ~\\
  \begin{itemize}
    \item Todos los estados son capitalistas o comunistas.
    \item Nadie que sea comunista es capitalista.
    \item Nadie que sea capitalista es comunista.
    \item Todos los capitalistas quieren destruir a los comunistas.
    \item Aerugo quiere destruir a los capitalistas.
    \item O Creta es comunista o Dracma es comunista.
    \item Si Dracma es comunista, entonces Amestris lo es.
    \item Todos los comunistas quieren destruir a Dracma.
    \item Xing se quiere destruir a si misma.
    \item Nadie más quiere destruir a nadie.
  \end{itemize}
  ~\\
  \begin{tabular}{| c | c | c | c |}
    \hline
      & $x$ es comunista & $x$ es capitalista \\
    \hline
    Amestris       & \false &        \\
    \hline
    Xing       &        & \false \\
    \hline
    Dracma &        &        \\
    \hline
    Creta &        &        \\
    \hline
    Aerugo       & \true  &        \\
    \hline
  \end{tabular}

  \begin{tabular}{| c | c | c | c | c | c |}
    \hline
$x$ quieren destruir a $y$ & Amestris & Xing & Dracma & Creta & Aerugo \\
    \hline
    Amestris       &       &       &       &      &       \\
    \hline
    Xing       &       &       &       &      &       \\
    \hline
    Dracma &       &       &       &      &       \\
    \hline
    Creta &       &       &       &      &       \\
    \hline
    Aerugo       &       &       &       &      &       \\
    \hline
  \end{tabular}
  \begin{solution}
    \begin{tabular}{| c | c | c | c |}
      \hline
        & $x$ es comunista & $x$ es capitalista \\
      \hline
      Amestris       & \false & \true  \\
      \hline
      Xing       & \true  & \false \\
      \hline
      Dracma & \false & \true  \\
      \hline
      Creta & \true  & \false \\
      \hline
      Aerugo       & \true  & \false \\
      \hline
    \end{tabular}

    \begin{tabular}{| c | c | c | c | c | c |}
      \hline
  $x$ quieren destruir a $y$ & Amestris & Xing & Dracma & Creta & Aerugo \\
      \hline
      Amestris & \false & \true  & \false & \true  & \true \\
      \hline
      Xing     & \true  & \false & \false & \true  & \false \\
      \hline
      Dracma   & \false & \true  & \false & \true  & \true  \\
      \hline
      Creta    & \false & \false & \true  & \false & \false \\
      \hline
      Aerugo   & \true & \false  & \true  & \false & \false \\
      \hline
    \end{tabular}
  \end{solution}

  \question
  Los conjuntos son un objeto matemático que consiste en una colección
  de elementos de algún tipo. Por ejemplo, podemos pensar en un conjunto
  de números (ej. $\{1,2,3,4\}$, es el conjunto que incluye a los números
  1, 2, 3 y 4, y ninguno más). Un conjunto, puede estar vacío (es un
  conjunto que no contiene ningún elemento).

  Considere los conjuntos \textbf{$U$} y \textbf{$C$} y los predicados \textbf{$x$ es un conjunto},
  \textbf{$x$ es un número} y \textbf{$x$ está incluido en $y$}.

  Se pide que desarrolle el diccionario que considere conveniente, y elabore
  fórmulas para expresar las siguientes afirmaciones.
  \begin{parts}
    \part El conjunto $C$ no es vacío.
    \begin{solution}
      $c = C$\\
      $u = U$\\
      $C(x) = x$ es un conjunto\\
      $R(x) = x$ es un número\\
      $I(x, y) = x$ está incluido en $y$\\
      ~\\
      $\exists z.I(z, c)$
    \end{solution}
    \part El conjunto $U$ contiene a todos los números.
    \begin{solution}
      $\forall z.R(z) \lthen I(z, u)$
    \end{solution}
    \part El conjunto C tiene a todos los números negativos.
    \begin{solution}
      $N(x) = x < 0$\\
      ~\\
      $\forall z.R(x) \land N(x) \lthen I(z, u)$
    \end{solution}
    \part Ningún conjunto se contiene a si mismo.
    \begin{solution}
      $\nexists a.C(a) \land I(a, a)$
    \end{solution}
  \end{parts}

\end{questions}
\end{document}
    