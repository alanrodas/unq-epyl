% !TeX encoding = UTF-8
% !TeX spellcheck = es_AR
%% Add the word "answers" to document class parameters in order to print the solutions.
\documentclass[10pt, addpoints]{../../common/epyl_exam_template}

\title{Parcialito Lógica}
\date{2018.1}
\professor{Alan Rodas Bonjour}
\timelimit{2 hs}
\topic{Lógica}
\instance{Parcialito}

\begin{document}
\makeexamheader
\makeexamtitle
\makeexamextradata

\begin{questions}
  \question
  Considerando las siguientes proposiciones como base:
  ~\\
  \begin{itemize}
    \item El paquete pesa más de 10 Kg
    \item El paquete pesa menos de 10 Kg
    \item El paquete mide más de 1 metro
    \item El paquete mide menos de 1 metro
  \end{itemize}
  ~\\
  Se le pide que exprese las expresiones a continuación en base a las anteriores:
  \begin{parts}
      \part \textbf{El paquete paga arancel especial} (Los paquetes que pesan mucho o que son muy largos pagan un arancel especial)
      \begin{solution} El paquete pesa más de 10 Kg $\lor$ El paquete mide más de 1 metro \end{solution}
      \part \textbf{El paquete se entrega en domicilio} (Si el paquete pesa menos de 10 Kg y mide menos de un metro)
      \begin{solution} El paquete pesa menos de 10 Kg $\land$ El paquete mide menos de 1 metro \end{solution}
      \part \textbf{Se retira en aduana} (Cualquier paquete que no se pueda entregar a domicilio y que pague un arancel especial)
      \begin{solution}$\lnot$ El paquete se entrega en domicilio $\land$ El paquete paga arancel especial \end{solution}
      \part \textbf{Se retira en sucursal} (Cualquier paquete que no pague arancel especial y pese más de 10 Kg)
      \begin{solution} $\lnot$ El paquete paga arancel especial $land$ El paquete pesa más de 10 Kg \end{solution}
  \end{parts}

  \jump

  \question
  Dados los siguientes razonamientos, identifique las indicadores de conclusión
  o de premisa que encuentra, especifique cuales son las premisas, cual es la
  conclusión, y pase a lenguaje formal de la lógica proposicional indicando
  claramente el diccionario y las conectivas para cada proposición.
  \begin{parts}
    \part~\\
    Si hubiera tenido una computadora de pequeño y me hubieran enseñado a
    programar en ese momento, entonces este curso me sería trivial. Pero no me
    enseñaron a programar de pequeño. Es por eso que este curso no me es trivial.
    \begin{solution}
        IC = Es por eso que\\
        p = Hubiera tenido una computadora de pequeño\\
        q = Me hubieran enseñado a programar de pequeño\\
        r = El curso me es trivial\\
        $(p \land q) \lthen r, \lnot q \lseq \lnot r$
    \end{solution}

    \part~\\
    La mesa no es adecuada, ya que una mesa es adecuada si y solo si tiene lugar para ocho
    personas o bien puede soportar mucho peso. Pero esta mesa ni tiene lugar para ocho
    personas ni soporta mucho peso.
    \begin{solution}
        p = La mesa tiene lugar para ocho personas\\
        q = La mesa soporta mucho peso\\
        r = La mesa es adecuada\\
        $(p \lor q) \liff r, \lnot p \land \lnot q \lseq \lnot r$
    \end{solution}
  \end{parts}
  
  \jump
  \question
  Dadas las formulas de los siguientes razonamientos, se pide que pruebe si son
  razonamientos válidos o inválidos.
  \begin{parts}
      \part $\lnot p \lthen q, \lnot p \lseq q$
      \begin{solution}
        \begin{tabular}{ c c | c | c | c || c}
          & Concl. & Premisa 2 & Premisa 1 & & Implic.\\
          \hline
          $p$ & $q$ & $\lnot p$ & $\lnot p \lthen q$ &
            $(\lnot p \lthen q) \land (\lnot p)$ & $((\lnot p \lthen q) \land (\lnot p)) \lthen q$ \\
          \hline
          \true  & \true  & \false & \true  & \false & \true  \\
          \true  & \false & \false & \true  & \false & \true  \\
          \false & \true  & \true  & \true  & \true  & \true  \\
          \false & \false & \true  & \false & \false & \true  \\
        \end{tabular}

        Es un razonamiento VÁLIDO.
      \end{solution}
      \part $p \lthen q, \lnot p \lseq \lnot q$
      \begin{solution}
        \begin{tabular}{ c c | c | c | c | c || c}
          && Premisa 1 & Premisa 2 & & Concl. & Implic.\\
          \hline
          $p$ & $q$ & $p \lthen q$ & $\lnot p$ & $(p \lthen q) \land (\lnot p)$
            & $\lnot q$ & $((p \lthen q) \land (\lnot p)) \lthen (\lnot q)$ \\
          \hline
          \true  & \true  & \true  & \false & \false & \false & \true  \\
          \true  & \false & \false & \false & \false & \true  & \true  \\
          \false & \true  & \true  & \true  & \true  & \false & \false \\
          \false & \false & \true  & \true  & \true  & \true  & \true  \\
        \end{tabular}

        Es un razonamiento INVÁLIDO.
      \end{solution}
      \part $(p \land q) \lor r, \lnot p \lseq r$
      \begin{solution}
        \begin{tabular}{ c c c c | c | c | c || c}
          && Concl. && Premisa 1 & Premisa 2 & & Implic.\\
          \hline
          $p$ & $q$ & $r$ & $p \land q$ & $(p \land q) \lor r$ & $\lnot p$
            & $A = ((p \land q) \lor r) \land (\lnot p)$ & $A \lthen r$ \\
          \hline
          \true  & \true  & \true  & \true  & \true  & \false & \false & \true \\
          \true  & \true  & \false & \true  & \true  & \false & \false & \true \\
          \true  & \false & \true  & \false & \true  & \false & \false & \true \\
          \true  & \false & \false & \false & \false & \false & \false & \true \\
          \false & \true  & \true  & \false & \true  & \true  & \true  & \true \\
          \false & \true  & \false & \false & \false & \true  & \false & \true \\
          \false & \false & \true  & \false & \true  & \true  & \true  & \true \\
          \false & \false & \false & \false & \false & \true  & \false & \true \\
        \end{tabular}
        
        Es un razonamiento VÁLIDO.
      \end{solution}
      \part $(p \land q) \lthen r, \lnot r \lseq (\lnot p) \lor (\lnot q)$
      \begin{solution}
        \tiny
        \begin{tabular}{ c c c c c c | c | c | c | c || c}
          &&&&&& Premisa 1 & Premisa 2 & & Concl. & Implic.\\
          \hline
          $p$ & $q$ & $r$ & $\lnot p$ & $\lnot q$ & $p \land q$ & $A = (p \land q) \lthen r$ & $\lnot r$
            & $B = A \land (\lnot r)$ & $C = (\lnot p) \lor (\lnot q)$ & $B \lthen C$ \\
          \hline
          \true  & \true  & \true  & \false & \false & \true  & \true  & \false & \false & \false & \true \\
          \true  & \true  & \false & \false & \false & \true  & \false & \true  & \false & \false & \true \\
          \true  & \false & \true  & \false & \true  & \false & \true  & \false & \false & \true  & \true \\
          \true  & \false & \false & \false & \true  & \false & \true  & \true  & \true  & \true  & \true \\
          \false & \true  & \true  & \true  & \false & \false & \true  & \false & \false & \true  & \true \\
          \false & \true  & \false & \true  & \false & \false & \true  & \true  & \true  & \true  & \true \\
          \false & \false & \true  & \true  & \true  & \false & \true  & \false & \false & \true  & \true \\
          \false & \false & \false & \true  & \true  & \false & \true  & \true  & \true  & \true  & \true \\
        \end{tabular}
        \normalsize
        
        Es un razonamiento VÁLIDO.
      \end{solution}
  \end{parts}
  \jump
  \question
  Sabiendo que las siguientes expresiones evalúan todas a \fulltrue, se pide
  que complete las tablas a continuación:
  ~\\
  \begin{itemize}
    \item Todos son o bien atléticos o bien inteligentes.
    \item Nadie que sea atlético es inteligente.
    \item Nadie que sea inteligente es atlético.
    \item Todos los que son atléticos son buenos en los deportes.
    \item Algunas personas inteligentes son buenos en los deportes.
    \item Mario no es bueno en los deportes.
    \item Todos los inteligentes aman a los demás inteligentes.
    \item Todos los atléticos aman a los demás atléticos.
    \item Toad ama a Luigi.
    \item Aquellos que son buenos en los deportes, aman a Luigi.
    \item Mario se ama a si mismo.
    \item Nadie más ama a nadie.
  \end{itemize}
  ~\\
  \begin{tabular}{| c | c | c | c |}
    \hline
      & $x$ es atlético & $x$ es inteligente & $x$ es bueno en los deportes \\
    \hline
    Mario & \false &        &        \\
    \hline
    Luigi &        & \false &        \\
    \hline
    Peach & \false &        &        \\
    \hline
    Toad  &        &        & \false \\
    \hline
    Yoshi & \true  &        &        \\
    \hline
    Daisy & \true  &        &        \\
    \hline
  \end{tabular}
  
  \begin{tabular}{| c | c | c | c | c | c | c |}
    \hline
$x$ ama a $y$ & Mario & Luigi & Peach & Toad & Yoshi & Daisy \\
    \hline
    Mario     &       &       &       &      &       &       \\
    \hline
    Luigi     &       &       &       &      &       &       \\
    \hline
    Peach     &       &       &       &      &       &       \\
    \hline
    Toad      &       &       &       &      &       &       \\
    \hline
    Yoshi     &       &       &       &      &       &       \\
    \hline
    Daisy     &       &       &       &      &       &       \\
    \hline
  \end{tabular}
  \begin{solution}
    \begin{tabular}{| c | c | c | c |}
      \hline
        & $x$ es atlético & $x$ es inteligente & $x$ es bueno en los deportes \\
      \hline
      Mario & \false & \true  & \false \\
      \hline
      Luigi & \true  & \false & \true  \\
      \hline
      Peach & \false & \true  & \true  \\
      \hline
      Toad  & \false & \true  & \false \\
      \hline
      Yoshi & \true  & \false & \true  \\
      \hline
      Daisy & \true  & \false & \true  \\
      \hline
    \end{tabular}
    
    \begin{tabular}{| c | c | c | c | c | c | c |}
      \hline
  $x$ ama a $y$ & Mario  & Luigi  & Peach  & Toad   & Yoshi  & Daisy  \\
      \hline
      Mario     & \true  & \false & \true  & \true  & \false & \false \\
      \hline
      Luigi     & \false & \true  & \false & \false & \true  & \true  \\
      \hline
      Peach     & \true  & \true  & \false & \true  & \false & \false \\
      \hline
      Toad      & \true  & \true  & \true  & \false & \false & \false \\
      \hline
      Yoshi     & \false & \true  & \false & \false & \false & \true  \\
      \hline
      Daisy     & \false & \true  & \false & \false & \true  & \false \\
      \hline
    \end{tabular}
  \end{solution}

  \jump
  \question
  Considere $a$, $b$ números naturales. Se pide exprese en términos
  lógicos las siguientes expresiones, definiendo los elementos del diccionario
  que crea convenientes para hacerlo.
  \begin{parts}
    \part Ningún número es menor que $a$
    \begin{solution}
      Constantes: $a$, $b$ \\
      Predicados:\\
      \quad Menor($x$, $y$) = $x$ es menor a $y$ (o en términos matemáticos $x < y$)\\
      ~\\
      $\nexists z. Menor(z, a)$

      \textbf{También se puede expresar de forma lógico matemática como:}

      $\nexists z. z < a$
    \end{solution}
    \part Existe un número tal que es más grande que $a$ y más chico que $b$
    \begin{solution}
      Constantes: $a$ \\
      Predicados:\\
      \quad Menor($x$, $y$) = $x$ es menor a $y$ (o en términos matemáticos $x < y$)\\
      \quad Mayor($x$, $y$) = $x$ es mayor a $y$ (o en términos matemáticos $x > y$)\\
      ~\\
      $\exists z. Mayor(z, a) \land Menor(z, b)$

      \textbf{También se puede usar solo Menor}

      $\exists z. Menor(a, z) \land Menor(z, b)$

      \textbf{También se puede expresar de forma lógico matemática como:}

      $\exists z. a < z \land z < b$
    \end{solution}
    \part Todo número más grande que $b$ es más grande que $a$
    \begin{solution}
      Constantes: $a$, $b$ \\
      Predicados:\\
      \quad Menor($x$, $y$) = $x$ es menor a $y$ (o en términos matemáticos $x < y$)\\
      \quad Mayor($x$, $y$) = $x$ es mayor a $y$ (o en términos matemáticos $x > y$)\\
      ~\\
      $\forall z. Mayor(z, b) \lthen Mayor(z, a)$

      \textbf{También se puede expresar de forma lógico matemática como:}

      $\forall z. z > b \lthen z > a$
    \end{solution}
    \part Todo número al que se le reste $a$ es igual a si mismo.
    \begin{solution}
      Constantes: $a$ \\
      Funciones:\\
      \quad resta($x$, $y$) = $x$ restado en $y$ (o en términos matemáticos $x - y$)
      Predicados:\\
      \quad Iguales($x$, $y$) = $x$ es igual a $y$ (o en términos matemáticos $x = y$)\\
      ~\\
      $\forall z. Iguales(resta(z, a), z)$

      \textbf{También se puede expresar de forma lógico matemática como:}

      $\forall z. z - a = z$
    \end{solution}
  \end{parts}

\end{questions}
\end{document}
        