
\section{Valores de verdad (Booleanos).}
\toc[currentsection,currentsubsection]

%%%%%%%%%%%%%%%%%%%%%%%%%%%%%%%%%%%%%%%%%%%%%%%%%%%%%

\begin{frame}{Valores de verdad}
  Como ya vimos la clase pasada, la lógica nos va a permitir hacer preguntas
  que se responden con si o no.
  \jump
  Otras acepciones de \colored{``si''} podrían ser \colored{``cierto''},
  \colored{``afirmativo''} o \colored{``verdadero''}
  \jump
  Mientras tanto, para \colored{``no''} podrían ser \colored{``mentira''},
  \colored{``negativo''} o \colored{``falso''}.
  \jump
  \bolder{Lo importante es que la respuesta es binaria, o una u otra, y no hay una
  tercera posibilidad}.
  \jump
  Más aún, \bolder{las respuestas son dicotómicas}. Es decir, no es posible que una
  respuesta sea ``si'' y ``no'' al mismo tiempo, y una es complementaria a la
  otra.
\end{frame}

%%%%%%%%%%%%%%%%%%%%%%%%%%%%%%%%%%%%%%%%%%%%%%%%%%%%%

\begin{frame}{Booleanos}
  A este tipo de respuestas se las conoce como \bolder{valores de verdad}, pues
  nos dicen si algo es verdadero o no. En general se resumen las respuestas
  afirmativas como ``Verdadero'' y las negativas como ``Falso''.
  \jump
  También se les llama \bolder{booleanos}. Se llaman así en honor a George Boole, un matemático y lógico británico
  quien estudió en profundidad las respuestas lógicas. Desarrolló un álgebra
  basada en la lógica (álgebra de Boole), la cual marca los fundamentos de la
  aritmética computacional moderna. Es considerado uno de los fundadores de las
  ciencias de la computación.
  \jump
  \bolder{Un booleano entonces, puede tener dos valores, los cuales son
  complementarios, ``Verdadero'' o ``Falso''}
\end{frame}

%%%%%%%%%%%%%%%%%%%%%%%%%%%%%%%%%%%%%%%%%%%%%%%%%%%%%

\begin{frame}{Respuestas a preguntas lógicas}
  Volvamos a las preguntas de nuestra verdulería mística. Las respuestas que nos
  dará el verdulero son entonces, respuestas booleanas.
  \jump
  \bolder{Cualquier pregunta que hagamos en la lógica, tendrá una respuesta booleana}
  \jump
  Así, si retomamos las preguntas que hicimos, podemos asignarles como respuestas
  ``\fulltrue'' (a aquellas que se respondieron con ``Si'') o ``\fullfalse'' (A
  aquellas que se respondieron con ``No'')
  \jump
  A lo largo de las diapositivas vamos a utilizar \true como ``\fulltrue'' y
  \false como ``\fullfalse''.
\end{frame}

%%%%%%%%%%%%%%%%%%%%%%%%%%%%%%%%%%%%%%%%%%%%%%%%%%%%%
%%%%%%%%%%%%%%%%%%%%%%%%%%%%%%%%%%%%%%%%%%%%%%%%%%%%%

\section{Preguntas usando la lógica.}
\toc[currentsection,currentsubsection]

%%%%%%%%%%%%%%%%%%%%%%%%%%%%%%%%%%%%%%%%%%%%%%%%%%%%%

\begin{frame}{Haciendo preguntas lógicas}
  Volviendo al ejemplo de la verdulería, imaginemos que hacemos las siguientes
  preguntas y obtenemos las respuestas que se muestran.
  \jump
  \begin{itemize}
    \item ¿Hay bananas? \true
    \item ¿Hay manzanas? \true
    \item ¿Hay naranjas? \false
    \item ¿Hay limones? \true
  \end{itemize}
  \jump
  Ahora bien, volvamos a formular nuestras queridas preguntas
  \colored{``¿Hay para una ensalada de frutas?''} y
  \colored{``¿Hay para preparar un bizcochuelo?''}.
  Recordemos, para que haya para una ensalada de frutas, necesitamos que hayan
  bananas, manzanas y naranjas. Por su parte, para poder preparar un bizcochuelo
  necesitamos que haya, o bien naranjas, o bien limones.
\end{frame}

%%%%%%%%%%%%%%%%%%%%%%%%%%%%%%%%%%%%%%%%%%%%%%%%%%%%%

\begin{frame}{Haciendo preguntas lógicas - Cont}
  Habíamos visto como:
  \jump
  \bolder{¿Hay para una ensalada de frutas? = ¿Hay bananas, naranjas y manzanas?}
  \jump
  La segunda pregunta, por la redacción que le damos en español por cuestiones
  de forma y estilo.
  \jump
  En realidad, se están preguntando tres cosas distintas en una sola pregunta:
  \begin{enumerate}
    \item El hecho de haber bananas
    \item El hecho de haber naranjas
    \item El hecho de haber manzanas
  \end{enumerate}
  \jump
  Por eso vamos a optar y preferir una redacción más redundante y que no suena
  necesariamente bien en español, pero que deja claro cuales son las partes que
  componen a nuestra pregunta.
\end{frame}

%%%%%%%%%%%%%%%%%%%%%%%%%%%%%%%%%%%%%%%%%%%%%%%%%%%%%

\begin{frame}{Haciendo preguntas lógicas - Cont}
  Así vamos a decir que \colored{``¿Hay para una ensalada de frutas?''} es lo
  mismo que preguntar \colored{``¿hay bananas? y ¿hay naranjas y ¿hay manzanas?''}
  \jump
  De forma similar, y como ya habíamos visto, la pregunta
  \colored{``¿Hay para preparar un bizcochuelo?''}, sería lo mismo que preguntar
  \colored{``¿hay naranjas o  limones?''}, lo cual con nuestro esquema de
  redacción es equivalente a \colored{``¿hay naranjas? o ¿hay limones?''}
  \jump
  Analicemos un poco cuales serán las respuestas a estas preguntas.
\end{frame}

%%%%%%%%%%%%%%%%%%%%%%%%%%%%%%%%%%%%%%%%%%%%%%%%%%%%%

\begin{frame}{Respuestas a preguntas compuestas}
  La respuesta a \colored{``¿Hay para una ensalada de frutas?''} debería ser
  \fullfalse, pues carecemos de uno de los ingredientes claves para poder hacer
  la ensalada, las naranjas, a pesar de que tengamos los otros dos ingredientes,
  las manzanas y las bananas.
  \jump
  La respuesta a \colored{``¿Hay para preparar un bizcochuelo?''} debería ser
  \fulltrue, pues a pesar de que no tenemos naranjas, podríamos usar limones.
  \jump
  Veamos un último ejemplo:
\end{frame}

%%%%%%%%%%%%%%%%%%%%%%%%%%%%%%%%%%%%%%%%%%%%%%%%%%%%%

\begin{frame}{Haciendo preguntas lógicas 2}
  A Ana le encanta lo que cocina su madre, quien le ha preparado un bizcochuelo.
  Sin embargo, lo que más le gusta a Ana es la ensalada de fruta. Por tanto, le
  ha pedido a su madre que solo le prepare bizcochuelo si le es imposible prepararle
  ensalada de frutas.
  \jump
  Ana, desconfiando de su madre, recurre al verdulero para preguntarle acerca
  de las frutas disponibles y ver que a su madre le haya sido imposible prepararle
  la ensalada de frutas que ella tanto desea.
  \jump
  Así, Ana no desea preguntar \colored{``¿Hay para una ensalada de frutas?''},
  sino más bien, lo opuesto, \colored{``¿Es cierto que no hay para una ensalada de frutas?''}
\end{frame}

%%%%%%%%%%%%%%%%%%%%%%%%%%%%%%%%%%%%%%%%%%%%%%%%%%%%%

\begin{frame}{Haciendo preguntas lógicas 2 - Cont}
  Pero, pensemos que significa en términos de lo que conocemos
  \colored{``¿Es cierto que no hay para una ensalada de frutas?''}
  \jump
  Para que no haya para ensalada, tendría que haber faltante de alguna de las
  frutas que requerimos. Es decir algunas de las siguientes tres preguntas,
  tendría que ser verdadera.
  \begin{itemize}
    \item \colored{¿Es cierto que no hay bananas?}
    \item \colored{¿Es cierto que no hay manzanas?}
    \item \colored{¿Es cierto que no hay naranjas?}
  \end{itemize}
  \jump
  Pero, ¿Sabemos las respuestas a esas preguntas?
\end{frame}

%%%%%%%%%%%%%%%%%%%%%%%%%%%%%%%%%%%%%%%%%%%%%%%%%%%%%

\begin{frame}{Haciendo preguntas lógicas 2 - Cont}
  Si lo pensamos, \colored{``¿Es cierto que no hay naranjas?''} solo
  puede ser \fulltrue cuando la respuesta a \colored{``¿hay naranjas?''}
  es \fullfalse.
  \jump
  De haber naranjas, es decir, la respuesta a \colored{``¿hay naranjas?''} es
  \fulltrue, entonces la respuesta a \colored{``¿Es cierto que no hay naranjas?''}
  será claramente \fullfalse.
  \jump
  Es decir, las preguntas \colored{``¿hay naranjas?''} y
  \colored{``¿Es cierto que no hay naranjas?''} son preguntas complementarias.
  El valor de una, depende de la otra, y son contrarios.
  \jump
  No hay forma de que el valor de ambas sean \fulltrue al mismo tiempo, o
  \fullfalse al mismo tiempo.
\end{frame}

%%%%%%%%%%%%%%%%%%%%%%%%%%%%%%%%%%%%%%%%%%%%%%%%%%%%%

\begin{frame}{Haciendo preguntas más complejas}
  Qué pasa si queremos hacer un bizcochuelo de banana. Sencillo, necesitamos
  poder hacer un bizcochuelo, pero además agregarle bananas. Pensemos entonces
  en la pregunta:
  \jump
  \colored{¿Hay para preparar un bizcochuelo de banana?}
  \jump
  ¿A qué sería equivalente dicha pregunta? Bien, sabemos que primero debemos
  saber si es posible hacer un bizcochuelo, por lo que la respuesta a
  la pregunta \colored{``¿Hay para preparar un bizcochuelo?''} debería ser
  \fulltrue. También tiene que haber bananas, por lo que la respuesta a la
  pregunta \colored{``¿Hay bananas?''} también debería ser \fulltrue.
\end{frame}

%%%%%%%%%%%%%%%%%%%%%%%%%%%%%%%%%%%%%%%%%%%%%%%%%%%%%

\begin{frame}{Haciendo preguntas más complejas}
  Podemos entonces encontrar que la conectiva que debe unir dichas dos preguntas
  es ``y'', ya que ambas partes deben cumplirse. Así, encontramos la equivalencia
  \jump
  \centerline{\colored{¿Hay para preparar un bizcochuelo de banana?}}
  \centerline{\colored{=}}
  \centerline{\colored{¿Hay para preparar un bizcochuelo? y ¿Hay bananas?}}
  \jump
  Ya definimos previamente que significa que haya para preparar un bizcochuelo
  en términos de preguntas más simples. Por tanto, no nos hace falta definirlo
  nuevamente.
  \jump
  La gracia de esta parte de la lógica va a radicar precisamente en qué
  \colored{si ya definimos una pregunta en terminos de preguntas básicas,
  podemos reutilizar ahora esta pregunta como si fuera una pregunta básica}.
\end{frame}

%%%%%%%%%%%%%%%%%%%%%%%%%%%%%%%%%%%%%%%%%%%%%%%%%%%%%

\begin{frame}{Haciendo preguntas más complejas}
  Esto último es vital para un programador.
  \jump
  Lo que queremos es poder definir preguntas en torno a otras preguntas, las
  cuales, pueden a su vez estar en torno a otras preguntas.
\end{frame}

%%%%%%%%%%%%%%%%%%%%%%%%%%%%%%%%%%%%%%%%%%%%%%%%%%%%%
%%%%%%%%%%%%%%%%%%%%%%%%%%%%%%%%%%%%%%%%%%%%%%%%%%%%%

\section{Formalismos lógicos}
\toc[currentsection,currentsubsection]

%%%%%%%%%%%%%%%%%%%%%%%%%%%%%%%%%%%%%%%%%%%%%%%%%%%%%

\begin{frame}{Lo que importa es la forma}
  Volvamos al ejemplo de la ensalada de frutas. La respuesta a
  \colored{``¿hay bananas? y ¿hay naranjas? y ¿hay manzanas?''} será \fulltrue
  solo cuando las respuestas individuales de \colored{``¿hay bananas?''},
  \colored{``¿hay naranjas?''} y \colored{``¿hay manzanas?''} son todas \fulltrue.
  \jump
  Supongamos ahora que en lugar de una verdulería estamos en un supermercado,
  y queremos saber si tenemos los elementos necesarios para una fiesta infantil
  (chizitos, papitas y palitos).
  \jump
  Así podríamos desarrollar la pregunta \colored{``¿hay elementos para fiesta infantil?''}
  como un análogo a la pregunta \colored{``¿hay chizitos? y ¿hay papitas? y ¿hay palitos?''}.
  \jump
  La respuesta a dicha pregunta será \fulltrue cuando las respuestas a las partes
  individuales (\colored{¿hay chizitos?}, \colored{¿hay papitas?} y
  \colored{¿hay palitos?}) sean todas \fulltrue.
\end{frame}

%%%%%%%%%%%%%%%%%%%%%%%%%%%%%%%%%%%%%%%%%%%%%%%%%%%%%

\begin{frame}{Lo que importa es la forma}
  Podríamos decir que las preguntas \colored{``¿hay bananas? y ¿hay naranjas? y ¿hay manzanas?''}
  y \colored{``¿hay chizitos? y ¿hay papitas? y ¿hay palitos?''} ambas tienen la misma
  \bolder{forma}.
  \jump
  Para la lógica es indistinto si estamos en la verdulería o en el supermercado,
  pues lo que nos va a dar son herramientas para trabajar con la forma de las
  preguntas, y no con su contenido.
  \jump
  \bolder{Por este motivo decímos que la lógica es una ciencia formal. Porque
  estudia las formas.}
\end{frame}

%%%%%%%%%%%%%%%%%%%%%%%%%%%%%%%%%%%%%%%%%%%%%%%%%%%%%

\begin{frame}{Lo que importa es la forma}
  Así, preguntas compuestas distintas que tengan la misma forma, tendrán
  respuestas que dependen de sus partes más sencilla de la misma forma
  (En el ejemplo anterior, todas las partes deben ser \fulltrue).
  \jump
  Así, un formalismo lógico nos va a decir como deben ser las formas y que
  reglas valen dentro de ellas, sin hablarnos nunca acerca del contenido de
  lo que estamos preguntando.
\end{frame}

%%%%%%%%%%%%%%%%%%%%%%%%%%%%%%%%%%%%%%%%%%%%%%%%%%%%%
%%%%%%%%%%%%%%%%%%%%%%%%%%%%%%%%%%%%%%%%%%%%%%%%%%%%%

\subsection{Conectivas.}
\toc[currentsection,currentsubsection]

%%%%%%%%%%%%%%%%%%%%%%%%%%%%%%%%%%%%%%%%%%%%%%%%%%%%%

\begin{frame}{Conectivas}
  Como vimos, al armar preguntas a partir de otras más simples, debemos unirlas
  con alguna palabra. Vimos dos: ``y'' y ``o''.
  \jump
  Pero, ¿cuando usamos una y cuando la otra? Y más importante, ¿Por qué?
  \jump
  Para ello, los formalismos lógicos definen el concepto de \bolder{conectiva},
  también llamado \bolder{conector lógico}. Una conectiva es una palabra que,
  de alguna forma, va a unir dos preguntas, o va a modificar la pregunta
  original de alguna forma.
  \jump
  Veremos entonces las distintas conectivas desde un punto intuitivo primero,
  para luego ver como los formalismos lógicos trabajan con dichas conectivas.
\end{frame}

%%%%%%%%%%%%%%%%%%%%%%%%%%%%%%%%%%%%%%%%%%%%%%%%%%%%%

\begin{frame}{Conjunción}
  La \bolder{conjunción} es el nombre formal que recibe unir preguntas con un
  \colored{``y''}.
  \jump
  \bolder{Cuando unimos preguntas con un ``y'' es porque esperamos que las
  preguntas sencillas que unimos tengan todas \fulltrue como respuesta para
  que la respuesta a la pregunta compuesta sea \fulltrue.}
  \jump
  Un ejemplo es el de la ensalada de frutas. La respuesta a
  \colored{``¿hay bananas? y ¿hay naranjas? y ¿hay manzanas?''} será \fulltrue
  solo cuando las respuestas individuales de \colored{``¿hay bananas?''},
  \colored{``¿hay naranjas?''} y \colored{``¿hay manzanas?''} son todas \fulltrue.
\end{frame}

%%%%%%%%%%%%%%%%%%%%%%%%%%%%%%%%%%%%%%%%%%%%%%%%%%%%%

\begin{frame}{Conjunción}
  Imaginemos dos preguntas cualquiera, llamémoslas ``$p$'' y ``$q$'' (`$a$'' y ``$b$''
  es indistinto).
  \jump
  Lo que nos dicen los formalismos lógicos acerca de la conjunción, es que si
  elaboramos una pregunta compuesta a partir de ``$p$'' y ``$q$'' uniendo ambas
  preguntas con ``y'', entonces, la respuesta a esa pregunta compuesta será
  \fulltrue solo cuando las respuestas a ``$p$'' y a ``$q$'' son \fulltrue.
  En cualquier otro caso, la respuesta a la pregunta compuesta será \fullfalse.
  \jump
  Eso queda expresado en la siguiente tabla:
  \jump
  \centerline{\begin{tabular}{c c | c}
    $p$ & $q$ & $p$ ``y'' $q$ \\
    \hline
    \true & \true & \true \\
    \true & \false & \false \\
    \false & \true & \false \\
    \false & \false & \false \\
  \end{tabular}}
\end{frame}

%%%%%%%%%%%%%%%%%%%%%%%%%%%%%%%%%%%%%%%%%%%%%%%%%%%%%

\begin{frame}{Conjunción}
  ¿Esto significa que solo podemos unir dos preguntas más chicas?
  \jump
  Bueno, si y no. Lo que nos dice es que si queremos unir más términos, en
  realidad tendremos que asociar en partes.
  \jump
  Por ejemplo, si se quiere una pregunta que sea \colored{$p$ ``y'' $q$ ``y'' $r$}
  entonces lo que se debe hacer son dos preguntas \colored{$p$ ``y'' $q$} por
  un lado, y luego una pregunta que involucre esta última con $r$:
  \colored{($(p$ ``y'' $q)$ ``y'' $r$}
  \jump
  El concepto es idéntico a lo que hacemos en matemática con la suma o la
  multiplicación, las cuales están definidas para dos elementos, pero eso no
  implica que no podamos escribir ``$3+5+7$''.
  \jump
  Dentro de un momento veremos bien que significa ``resolver'' en
  términos lógicos.
\end{frame}

%%%%%%%%%%%%%%%%%%%%%%%%%%%%%%%%%%%%%%%%%%%%%%%%%%%%%

\begin{frame}{Disyunción}
  Otra de las formas de unir las preguntas es mediante un ``o'', lo que se
  conoce como \bolder{disyunción}.
  \jump
  La disyunción nos dice que para que una pregunta compuesta sea \fulltrue basta
  con que alguna de las preguntas más sencillas que las componen sea \fulltrue.
  \jump
  Tal es el ejemplo del bizcochuelo. Si hay naranjas y limones, podemos elegir
  cualquiera de los dos para hacer el bizcochuelo, si solo hay naranjas,
  utilizaremos naranjas, y si solo hay limones, usaremos limones. Solo en el caso
  en que no haya ni naranjas ni limones nos será imposible preparar el bizcochuelo.
\end{frame}

%%%%%%%%%%%%%%%%%%%%%%%%%%%%%%%%%%%%%%%%%%%%%%%%%%%%%

\begin{frame}{Disyunción}
  Volvamos a suponer dos preguntas cualquiera ``$p$'' y ``$q$''.
  \jump
  Lo que nos dicen los formalismos lógicos acerca de la disyunción, es que si
  elaboramos una pregunta compuesta a partir de ``$p$'' y ``$q$'' uniendo ambas
  preguntas con ``o'', entonces, la respuesta a esa pregunta compuesta será
  \fulltrue cuando la respuestas a ``$p$'' sea y a ``$q$'' sean ambas \fulltrue,
  o cuando la respuesta a ``$p$'' sea \fulltrue y a ``$q$'' sea \fullfalse, o
  cuando a ``$p$'' sea \fullfalse y a ``$q$'' sea \fulltrue.
  En el caso, en que tanto ``$p$'' como ``$q$'' sean \fullfalse la respuesta a
  la pregunta compuesta será \fullfalse.
  \jump
  Esto se expresa en la siguiente tabla:
  \jump
  \centerline{\begin{tabular}{c c | c}
    $p$ & $q$ & $p$ ``o'' $q$ \\
    \hline
    \true & \true & \true \\
    \true & \false & \true \\
    \false & \true & \true \\
    \false & \false & \false \\
  \end{tabular}}
\end{frame}

%%%%%%%%%%%%%%%%%%%%%%%%%%%%%%%%%%%%%%%%%%%%%%%%%%%%%

\begin{frame}{Negación}
  La \bolder{negación} es la única conectiva que trabaja sobre solo una
  pregunta.
  \jump
  Usamos la negación cuando queremos que una pregunta sea \fulltrue, cuando
  la pregunta original sobre la que actúa sea \fullfalse. Es decir, cuando
  queremos el complemento de la pregunta original.
  \jump
  Tal es el caso de la pregunta \colored{``¿Es cierto que no hay naranjas?''}.
  En esa pregunta lo que nos interesa preguntar es por el complemento de 
  \colored{``¿hay naranjas?''}
\end{frame}

%%%%%%%%%%%%%%%%%%%%%%%%%%%%%%%%%%%%%%%%%%%%%%%%%%%%%

\begin{frame}{Negación}
  Así, si suponemos una pregunta cualquiera ``$p$'', vamos a decir que
  la conectiva de negación aplicada a esa pregunta (``no'' $p$), va a darnos el
  complemento de $p$.
  \jump
  Es decir, si ``$p$'' es \fulltrue, entonces ``no $p$'' será \fullfalse. En
  cambio si ``$p$'' es \fullfalse, entonces ``no $p$'' será \fulltrue.
  \jump
  Esto está dado por el siguiente cuadro:
  \jump
  \centerline{\begin{tabular}{c | c}
    $p$ & ``no'' $p$ \\
    \hline
    \true & \false \\
    \false & \true
  \end{tabular}}
\end{frame}

%%%%%%%%%%%%%%%%%%%%%%%%%%%%%%%%%%%%%%%%%%%%%%%%%%%%%

\begin{frame}{Conectivas}
  Cada conectiva está formalizada con un símbolo (Así como la suma está
  formalizada con el signo ``+'')
  \jump
  Así, a la conjunción se le asigna el signo $\land$, a la disyunción el signo
  $\lor$ y a la negación el signo $\lnot$.
  \jump
  Para resumir entonces, vimos tres conectivas, con las siguientes tablas:
  \jump
  \begin{columns}
    \begin{column}{0.3\textwidth}
      \centerline{\bolder{Conjunción}}
      \centerline{\begin{tabular}{c c | c}
        $p$ & $q$ & $p \land q$ \\
        \hline
        \true & \true & \true \\
        \true & \false & \false \\
        \false & \true & \false \\
        \false & \false & \false \\
      \end{tabular}}
    \end{column}
    \begin{column}{0.3\textwidth}
      \centerline{\bolder{Disyunción}}
      \centerline{\begin{tabular}{c c | c}
        $p$ & $q$ & $p \lor q$ \\
        \hline
        \true & \true & \true \\
        \true & \false & \true \\
        \false & \true & \true \\
        \false & \false & \false \\
      \end{tabular}}
    \end{column}
    \begin{column}{0.3\textwidth}
      \centerline{\bolder{Negación}}
      \centerline{\begin{tabular}{c | c}
        $p$ & ``no'' $p$ \\
        \hline
        \true & \false \\
        \false & \true
      \end{tabular}}
    \end{column}
  \end{columns}
\end{frame}

%%%%%%%%%%%%%%%%%%%%%%%%%%%%%%%%%%%%%%%%%%%%%%%%%%%%%
%%%%%%%%%%%%%%%%%%%%%%%%%%%%%%%%%%%%%%%%%%%%%%%%%%%%%

\subsection{Tablas de Verdad.}
\toc[currentsection,currentsubsection]

%%%%%%%%%%%%%%%%%%%%%%%%%%%%%%%%%%%%%%%%%%%%%%%%%%%%%

\begin{frame}{¿Cuando algo es verdadero?}
  Pensemos un momento. ¿Como sabemos cuando una pregunta compuesta es
  \fulltrue y cuando es \fullfalse?
  \jump
  La realidad es que hasta el momento solamente lo sabemos de forma intuitiva,
  y esto es sencillo con preguntas que tienen pocas conectivas y pocas partes,
  pero puede volverse muy complejo muy rápidamente.
  \jump
  Pensemos por ejemplo en una persona que va a la verdulería para saber si
  puede preparar alguno de varios platillos. Para ello elabora la siguiente
  pregunta compuesta:
  \jump
  \colored{¿hay papas? y ¿hay batatas?, o ¿hay papas? y ¿hay zanahorias?, o
  ¿es cierto que no hay papas? y ¿hay puerro?}
  \jump
  La pregunta en si es bastante ambigua, pero saber cuando es verdadero o falso
  es bastante complejo.
\end{frame}

%%%%%%%%%%%%%%%%%%%%%%%%%%%%%%%%%%%%%%%%%%%%%%%%%%%%%

\begin{frame}{¿Cuando algo es verdadero?}
  Por suerte la lógica nos da una forma bastante eficaz de saber cuando
  una pregunta compuesta es \fulltrue y cuando es \fullfalse.
  \jump
  El método se conoce como \bolder{ánalisis mediante tabla de verdad}, y consiste
  en elaborar una tabla a partir de las preguntas más básicas. Así, se puede
  obtener que valores deben tener las mismas para que la pregunta compuesta
  sea \fulltrue.
  \jump
  Así como en matemática una cuenta compleja se resuelve paso a paso solucionando
  cada una de las operaciones en un orden determinado, en las tablas de verdad
  la resolución se realiza paso a paso, y estará dada por las tablas de las
  conectivas.
\end{frame}

%%%%%%%%%%%%%%%%%%%%%%%%%%%%%%%%%%%%%%%%%%%%%%%%%%%%%

\begin{frame}{Tabla de verdad - Ejemplo}
  Intentemos entonces elaborar una tabla de verdad para el ejemplo que vimos:
  \colored{¿hay papas? y ¿hay batatas?, o ¿hay papas? y ¿hay zanahorias?, o
  ¿es cierto que no hay papas? y ¿hay puerro?}.
  \jump
  El primer paso consiste en encontrar las preguntas sencillas que componen esa
  pregunta compuesta.
\end{frame}

%%%%%%%%%%%%%%%%%%%%%%%%%%%%%%%%%%%%%%%%%%%%%%%%%%%%%

\begin{frame}{Tabla de verdad - Ejemplo}
  Intentemos entonces elaborar una tabla de verdad para el ejemplo que vimos:
  \colored{¿hay papas? y ¿hay batatas?, o ¿hay papas? y ¿hay zanahorias?, o
  ¿es cierto que no hay papas? y ¿hay batatas?}.
  \jump
  El primer paso consiste en encontrar las preguntas sencillas que componen esa
  pregunta compuesta. Tenemos:
  \begin{itemize}
    \item ¿hay papas?
    \item ¿hay batatas?
    \item ¿hay zanahoria?
  \end{itemize}
  \jump
  Note que ``¿es cierto que no hay papas?'' es una pregunta que se resuelve
  sabiendo ``¿hay papas?'', pues es su complemento.
\end{frame}

%%%%%%%%%%%%%%%%%%%%%%%%%%%%%%%%%%%%%%%%%%%%%%%%%%%%%

\begin{frame}{Tabla de verdad - Ejemplo Cont}
  Una vez que tenemos esas preguntas, nos será necesario formalizar la
  pregunta en términos de nuestras conectivas de forma clara. Es decir,
  colocando paréntesis para separar claramente las cosas, y replanteando
  algunas preguntas en términos de complementos.
  \jump
  Nos quedaría de la siguiente forma:
  \jump
  \colored{(¿hay papas? $\land$ ¿hay batatas?) $\lor$ (¿hay papas? $\land$ ¿hay zanahorias?) $\lor$
  (($\lnot$ ¿hay papas?) $\land$ ¿hay batatas?)}.
  \jump
  Ahora si podremos armar la tabla.
\end{frame}

%%%%%%%%%%%%%%%%%%%%%%%%%%%%%%%%%%%%%%%%%%%%%%%%%%%%%

\begin{frame}{Tabla de verdad - Ejemplo Cont}
  Comenzamos por colocar una columna por cada pregunta que tenemos que contestar.
  \jump
  \begin{tabular}{c  c  c}
    ¿hay papas? & ¿hay batatas? & ¿hay zanahorias? \\
    \hline
  \end{tabular}
\end{frame}

%%%%%%%%%%%%%%%%%%%%%%%%%%%%%%%%%%%%%%%%%%%%%%%%%%%%%

\begin{frame}{Tabla de verdad - Ejemplo Cont}
  Cada una de esas preguntas se responde, o bien con \fulltrue o bien con \fullfalse.
  A su vez, puede darse que haya papas y batatas, pero no zanahorias. O que haya
  papas, pero no haya batatas ni zanahorias. A priori no sabemos como nos
  responderá el verdulero, y por eso tenemos que analizar todos los posibles casos.
  \jump
  Para cada caso, completamos con una nueva fila, colocando los valores de las
  respuestas a cada una de las preguntas que tendrían en dicho caso.
  \jump
  En la primer fila se muestra el caso de que todas las respuestas sean \fulltrue,
  y en la segunda, el caso en donde solo falta zanahoria.
  \jump
  \centerline{\begin{tabular}{c  c  c}
    ¿hay papas? & ¿hay batatas? & ¿hay zanahorias? \\
    \hline
    \fulltrue & \fulltrue & \fulltrue \\
    \fulltrue & \fulltrue & \fullfalse
  \end{tabular}
  }
\end{frame}

%%%%%%%%%%%%%%%%%%%%%%%%%%%%%%%%%%%%%%%%%%%%%%%%%%%%%

\begin{frame}{Tabla de verdad - Ejemplo Cont}
  Completamos con los casos restantes.
  \jump
  La cantidad de casos es $2^n$ donde $n$ es la cantidad de preguntas sencillas
  a responder por el verdulero. En este caso son 3, y por tanto habrá $2^3 = 8$
  posibles casos.
  \jump
  \centerline{\begin{tabular}{c  c  c}
    ¿hay papas? & ¿hay batatas? & ¿hay zanahorias? \\
    \hline
    \fulltrue & \fulltrue & \fulltrue \\
    \fulltrue & \fulltrue & \fullfalse \\
    \fulltrue & \fullfalse & \fulltrue \\
    \fulltrue & \fullfalse & \fullfalse \\
    \fullfalse & \fulltrue & \fulltrue \\
    \fullfalse & \fulltrue & \fullfalse \\
    \fullfalse & \fullfalse & \fulltrue \\
    \fullfalse & \fullfalse & \fullfalse
  \end{tabular}
  }
\end{frame}

%%%%%%%%%%%%%%%%%%%%%%%%%%%%%%%%%%%%%%%%%%%%%%%%%%%%%

\begin{frame}{Tabla de verdad - Ejemplo Cont}
  Ahora empezamos a agregar columnas para cada una de las preguntas compuestas
  a resolver. En el primer paréntesis por ejemplo, debemos responder
  ``(¿hay papas? $\land$ ¿hay batatas?)''.
  \jump
  A su vez, completamos las distintas
  filas teniendo en cuenta las respuestas que toman cada una de los componentes de
  esa pregunta en dicha fila y la conectiva utilizada.
  \jump
  Así, la primera y segunda fila se completarán con \fulltrue, pues hay tanto
  papas como batatas, pero esto no ocurrirá en ninguna otra fila, por lo que el
  resto lo completaremos con \fullfalse. Esto es lo que dice la tabla de la
  conjunción.
\end{frame}

%%%%%%%%%%%%%%%%%%%%%%%%%%%%%%%%%%%%%%%%%%%%%%%%%%%%%

\begin{frame}{Tabla de verdad - Ejemplo Cont}
  \Tiny{
  \centerline{\begin{tabular}{c  c  c | c }
    ¿hay papas? & ¿hay batatas? & ¿hay zanahorias? & ¿hay papas? $\land$ ¿hay batatas?)\\
    \hline
    \fulltrue & \fulltrue & \fulltrue & \fulltrue \\
    \fulltrue & \fulltrue & \fullfalse & \fulltrue \\
    \fulltrue & \fullfalse & \fulltrue & \fullfalse \\
    \fulltrue & \fullfalse & \fullfalse & \fullfalse \\
    \fullfalse & \fulltrue & \fulltrue & \fullfalse \\
    \fullfalse & \fulltrue & \fullfalse & \fullfalse \\
    \fullfalse & \fullfalse & \fulltrue & \fullfalse \\
    \fullfalse & \fullfalse & \fullfalse & \fullfalse \\
  \end{tabular}
  }}
\end{frame}

%%%%%%%%%%%%%%%%%%%%%%%%%%%%%%%%%%%%%%%%%%%%%%%%%%%%%

\begin{frame}{Tabla de verdad - Ejemplo Cont}
  Seguimos el proceso para cada paréntesis. En este punto vemos que vamos a obtener
  una tabla bastante grande, por lo que es conveniente simplificar la tabla.
  \jump
  Llamaremos ``$p$'' a ``¿hay papas?'', ``$b$'' a ``¿hay batatas?'' y ``$z$'' a
  ``¿hay zanahoria?'' para que ocupen menos lugar. Completamos luego con los
  paréntesis que siguen.
  \jump
  \centerline{\begin{tabular}{c  c  c | c | c | c}
    $p$ & $b$ & $z$ & $p \land b$ & $p \land z$ & $\lnot p$\\
    \hline
    \true  & \true  & \true  & \true  & \true  & \false \\
    \true  & \true  & \false & \true  & \false & \false \\
    \true  & \false & \true  & \false & \true  & \false \\
    \true  & \false & \false & \false & \false & \false \\
    \false & \true  & \true  & \false & \false & \true \\
    \false & \true  & \false & \false & \false & \true \\
    \false & \false & \true  & \false & \false & \true \\
    \false & \false & \false & \false & \false & \true \\
  \end{tabular}
  }
\end{frame}

\begin{frame}{Tabla de verdad - Ejemplo Cont}
  En este punto ya tenemos que agregar columnas que usan los valores de las
  columnas que calculamos anteriormente para resolverse, como es el caso de 
  ``($\lnot$ ¿hay papas?) $\land$ ¿hay batatas?''. El proceso es idéntico,
  solo que realizamos la tabla mirando la columna de ``$\lnot p$'' y la de
  ``$b$''.
  \jump
  \centerline{\begin{tabular}{c  c  c | c | c | c | c | c }
    $p$ & $b$ & $z$ & $p \land b$ & $p \land z$ & $\lnot p$ & $(\lnot p) \land b $\\
    \hline
    \true  & \true  & \true  & \true  & \true  & \false & \false \\
    \true  & \true  & \false & \true  & \false & \false & \false \\
    \true  & \false & \true  & \false & \true  & \false & \false \\
    \true  & \false & \false & \false & \false & \false & \false \\
    \false & \true  & \true  & \false & \false & \true  & \true \\
    \false & \true  & \false & \false & \false & \true  & \true \\
    \false & \false & \true  & \false & \false & \true  & \false \\
    \false & \false & \false & \false & \false & \true  & \false \\
  \end{tabular}
  }
\end{frame}

%%%%%%%%%%%%%%%%%%%%%%%%%%%%%%%%%%%%%%%%%%%%%%%%%%%%%

\begin{frame}{Tabla de verdad - Ejemplo Cont}
  Por ultimo resolvemos las disyunciones por partes, de derecha a izquierda, pues
  no hay paréntesis para desambiguar allí. Siempre usando la columna de los
  resultados que fuimos obteniendo y completando según las tablas de las
  conectivas.
  \jump
  \Tiny{
  \centerline{\begin{tabular}{c  c  c | c | c | c | c | c || c }
    $p$ & $b$ & $z$ & $p \land b$ & $p \land z$ & $\lnot p$ & $(\lnot p) \land b $ & $(p \land b) \lor (p \land z)$ &  $((p \land b) \lor (p \land z)) \lor ((\lnot p) \land b)$\\
    \hline
    \true  & \true  & \true  & \true  & \true  & \false & \false & \true  & \true  \\
    \true  & \true  & \false & \true  & \false & \false & \false & \true  & \true  \\
    \true  & \false & \true  & \false & \true  & \false & \false & \true  & \true  \\
    \true  & \false & \false & \false & \false & \false & \false & \false & \false \\
    \false & \true  & \true  & \false & \false & \true  & \true  & \false & \true  \\
    \false & \true  & \false & \false & \false & \true  & \true  & \false & \true  \\
    \false & \false & \true  & \false & \false & \true  & \false & \false & \false \\
    \false & \false & \false & \false & \false & \true  & \false & \false & \false \\
  \end{tabular}
  }}
\end{frame}

%%%%%%%%%%%%%%%%%%%%%%%%%%%%%%%%%%%%%%%%%%%%%%%%%%%%%

\begin{frame}{Tabla de verdad - Ejemplo Cont}
  Una vez que tenemos la tabla completa, podemos descartar los resultados
  intermedios, pues no los utilizaremos.
  \jump
  \centerline{\begin{tabular}{c  c  c || c }
    $p$ & $b$ & $z$ & $(p \land b) \lor (p \land z) \lor ((\lnot p) \land b)$\\
    \hline
    \true  & \true  & \true  & \true  \\
    \true  & \true  & \false & \true  \\
    \true  & \false & \true  & \true  \\
    \true  & \false & \false & \false \\
    \false & \true  & \true  & \true  \\
    \false & \true  & \false & \true  \\
    \false & \false & \true  & \false \\
    \false & \false & \false & \false \\
  \end{tabular}
  }
\end{frame}

%%%%%%%%%%%%%%%%%%%%%%%%%%%%%%%%%%%%%%%%%%%%%%%%%%%%%

\begin{frame}{Analizando el resultado de una tabla de verdad}
  Si miramos las filas resultantes de una tabla de verdad, podremos ver en que
  casos la respuesta será \fulltrue de forma un poco más sencilla.
  \jump
  Por ejemplo, podemos apreciar que basta con que haya papas y alguna otra
  verdura (batatas o zanahorias) o que haya batatas y no papas, independientemente
  de si hay zanahorias o no.
  \jump
  No solo nos permite entender mejor los casos de verdad y de falsedad de la
  pregunta, sino que nos va a permitir comprender mejor la naturaleza de la
  misma, y por tanto formular nuevas preguntas.
\end{frame}

%%%%%%%%%%%%%%%%%%%%%%%%%%%%%%%%%%%%%%%%%%%%%%%%%%%%%
%%%%%%%%%%%%%%%%%%%%%%%%%%%%%%%%%%%%%%%%%%%%%%%%%%%%%

\subsection{Valuaciones.}
\toc[currentsection,currentsubsection]

%%%%%%%%%%%%%%%%%%%%%%%%%%%%%%%%%%%%%%%%%%%%%%%%%%%%%

\begin{frame}{Valuaciones}
  Una \bolder{valuación} consiste en la asignación de valores de verdad a cada
  una de las preguntas. Al hacer esto estamos indicando que el universo es de
  una manera particular (estamos imaginando la forma en la que nos respondería
  el verdulero)
  \jump
  No siempre podemos saber como es el universo, pero podemos analizar todas las
  posibles valuaciones y encontrar cosas interesantes.
  \jump
  Cada fila de una tabla de verdad se corresponde con una valuación.
\end{frame}

%%%%%%%%%%%%%%%%%%%%%%%%%%%%%%%%%%%%%%%%%%%%%%%%%%%%%

\begin{frame}{Tautologías}
  Imaginemos la siguiente pregunta compuesta:
  \jump
  \colored{¿hay naranjas? o ¿es cierto que no hay naranjas?}
  \jump
  La forma lógica de esta pregunta es:
  \jump
  \colored{¿hay naranjas? $\lor$ $\lnot$ ¿hay naranjas?}
  \jump
  Analicemos las posibles valuaciones para dicha fórmula.
\end{frame}

%%%%%%%%%%%%%%%%%%%%%%%%%%%%%%%%%%%%%%%%%%%%%%%%%%%%%

\begin{frame}{Tautologías}
  Solo hay una pregunta, \colored{``¿hay naranjas?''}. Esa pregunta puede
  tener dos posibles respuestas, \fulltrue o \fullfalse.
  \jump
  Veamos mediante la tabla de verdad, que pasa con la pregunta compuesta en
  cada caso:
  \jump
  \centerline{
    \begin{tabular}{c | c || c}
      ¿hay naranjas? & $\lnot$ ¿hay naranjas? & ¿hay naranjas? $\lor$ $\lnot$ ¿hay naranjas? \\
      \hline
      \fulltrue & \fullfalse & \fulltrue \\
      \fullfalse & \fulltrue & \fulltrue \\
    \end{tabular}
  }
\end{frame}

%%%%%%%%%%%%%%%%%%%%%%%%%%%%%%%%%%%%%%%%%%%%%%%%%%%%%

\begin{frame}{Tautologías}
  \bolder{Decimos que una pregunta es una tautología, si todas las valuaciones
  posibles dan siempre \fulltrue}.
  \jump
  Note que una pregunta simple, nunca es una tautología, pues puede ser
  respondida tanto con \fulltrue como con \fullfalse.
\end{frame}

%%%%%%%%%%%%%%%%%%%%%%%%%%%%%%%%%%%%%%%%%%%%%%%%%%%%%

\begin{frame}{Contradicciones}
  Imaginemos ahora la siguiente pregunta compuesta:
  \jump
  \colored{¿hay naranjas? y ¿es cierto que no hay naranjas?}
  \jump
  La forma lógica de esta pregunta es:
  \jump
  \colored{¿hay naranjas? $\land$ $\lnot$ ¿hay naranjas?}
  \jump
  Analicemos las posibles valuaciones para dicha fórmula.
\end{frame}

%%%%%%%%%%%%%%%%%%%%%%%%%%%%%%%%%%%%%%%%%%%%%%%%%%%%%

\begin{frame}{Contradicciones}
  Nuevamente, solo hay una pregunta, \colored{``¿hay naranjas?''}. Esa pregunta
  puede tener dos posibles respuestas, \fulltrue o \fullfalse.
  \jump
  Veamos mediante la tabla de verdad, que pasa con la pregunta compuesta en
  cada caso:
  \jump
  \centerline{
    \begin{tabular}{c | c || c}
      ¿hay naranjas? & $\lnot$ ¿hay naranjas? & ¿hay naranjas? $\land$ $\lnot$ ¿hay naranjas? \\
      \hline
      \fulltrue & \fullfalse & \fullfalse \\
      \fullfalse & \fulltrue & \fullfalse \\
    \end{tabular}
  }
\end{frame}

%%%%%%%%%%%%%%%%%%%%%%%%%%%%%%%%%%%%%%%%%%%%%%%%%%%%%

\begin{frame}{Contradicciones}
  \bolder{Decimos que una pregunta es una contradicción, si todas las valuaciones
  posibles dan siempre \fullfalse}.
  \jump
  Nuevamente, note que una pregunta simple, nunca es una contradicción, pues
  puede ser respondida tanto con \fulltrue como con \fullfalse.
\end{frame}

%%%%%%%%%%%%%%%%%%%%%%%%%%%%%%%%%%%%%%%%%%%%%%%%%%%%%

\begin{frame}{Contingencias}
  \bolder{Decimos que una pregunta es una contingencia, si algunas de las
  valuaciones posibles dan \fulltrue y otras dan \fullfalse}.
\end{frame}
