
\begin{frame}{Donde la metáfora se queda corta}
  Hasta ahora vimos como la lógica nos ayuda a responder preguntas.
  \jump
  De hecho, vimos como, si tuviéramos verdulero místico al cual poder
  preguntarle cosas, podríamos hacer varias preguntas a partir de preguntas más
  simples.
  \jump
  Pero en la vida real, no vamos a tener ningún verdulero que nos responda
  nuestras preguntas, ni solo vamos a querer preguntar sobre verduras.
  \jump
  En general, tampoco vamos a preguntar cosas al aire, sino que vamos a hacer
  afirmaciones sobre el mundo, las cuales, a priori, pueden ser verdaderas o
  falsas, y que deberán ser sometidas a contrastación empírica.
\end{frame}

%%%%%%%%%%%%%%%%%%%%%%%%%%%%%%%%%%%%%%%%%%%%%%%%%%%%%

\begin{frame}{Ejemplos de afirmaciones}
  Podríamos pensar en el siguiente ejemplo:
  \jump
  \colored{La Tierra es plana}
  \jump
  Consideremos entonces dicha afirmación. ¿Será \fulltrue o \fullfalse?
  \jump
  Hoy, cualquier persona en su sano juicio (o sea, no los que creen cualquier
  cosa que ven en Youtube) diría que esa afirmación es válida.
  O sea, que si fuera una pregunta, \colored{``¿La Tierra es plana?''},
  entonces, algún ser místico (¿Dios?) nos respondería \fullfalse.
  \jump
  Pero si hubiéramos hecho la pregunta hace 1500 o 2000 años, probablemente
  nuestra intuición y conocimiento nos habrían dicho que la respuesta obvia era
  \fullfalse.
\end{frame}

%%%%%%%%%%%%%%%%%%%%%%%%%%%%%%%%%%%%%%%%%%%%%%%%%%%%%

\begin{frame}{Ejemplos de afirmaciones}
  Podríamos pensar también en
  \jump
  \colored{La Tierra gira alrededor del sol}
  \jump
  \colored{La enfermedades son causadas por animales microscópicos}
  \jump
  Todas estas afirmaciones fueron consideradas falsas, pero luego pasaron a
  considerarse verdaderas cuando el estado de la ciencia y la técnica fue
  avanzando.
  \jump
  Podríamos considerar que son, de alguna forma, preguntas encubiertas.
\end{frame}

%%%%%%%%%%%%%%%%%%%%%%%%%%%%%%%%%%%%%%%%%%%%%%%%%%%%%

\begin{frame}{Sobre las afirmaciones}
  A la lógica, en tanto se usa para analizar la realidad, no le interesan las
  preguntas dirigidas a un ser místico, sino el análisis de las afirmaciones
  que hacemos, y como una serie de afirmaciones se relacionan con otras.
  \jump
  Así, nuestra analogía de la lógica como algo que nos responde preguntas llega
  hasta este punto.
  \jump
  Eso no significa que los conceptos que vimos hasta ahora no nos sirvan, pero
  vamos a ver como estos encajan en el formalismo lógico más sencillo.
\end{frame}

%%%%%%%%%%%%%%%%%%%%%%%%%%%%%%%%%%%%%%%%%%%%%%%%%%%%%
%%%%%%%%%%%%%%%%%%%%%%%%%%%%%%%%%%%%%%%%%%%%%%%%%%%%%

\section{Introducción a la lógica proposicional.}
\toc[currentsection,currentsubsection]

%%%%%%%%%%%%%%%%%%%%%%%%%%%%%%%%%%%%%%%%%%%%%%%%%%%%%

\begin{frame}{Lógica proposicional}
  \begin{block}{Lógica proposicional}
    La \bolder{lógica proposicional} o \bolder{lógica de orden cero} es la
    rama de la lógica matemática que estudia proposiciones, los métodos de
    vincularlas mediante conectores lógicos y las relaciones y propiedades
    que se derivan de esos procedimientos.
  \end{block}
\end{frame}

%%%%%%%%%%%%%%%%%%%%%%%%%%%%%%%%%%%%%%%%%%%%%%%%%%%%%
%%%%%%%%%%%%%%%%%%%%%%%%%%%%%%%%%%%%%%%%%%%%%%%%%%%%%

\subsection{Proposiciones}

%%%%%%%%%%%%%%%%%%%%%%%%%%%%%%%%%%%%%%%%%%%%%%%%%%%%%

\begin{frame}{Proposiciones}
  \begin{block}{Proposición}
    Entidad atómica de la lógica proposicional, portadora de valor de verdad
  \end{block}
  \jump
  Una proposición es una oración que afirma o cuenta algo, sobre lo que podemos
  decir que es cierto o no lo es.
  \jump
  Es, podríamos decir, la posibilidad de que eso se cumpla.
  \jump
  Las proposiciones, usan la \bolder{función informativa} del lenguaje
  (también llamada a veces \bolder{descriptiva} o \bolder{aseverativa}).
\end{frame}

%%%%%%%%%%%%%%%%%%%%%%%%%%%%%%%%%%%%%%%%%%%%%%%%%%%%%

\begin{frame}{Funciones del lenguaje}
  Puede ser útil pues repasar las funciones del lenguaje para tener más claro
  que cosas son proposiciones y que cosas no lo son:
  \jump
  \small{
  \centerline{\begin{tabular}{|l | p{2.5cm} | p{6cm}|}
    \hline
    Imperativo & ¡Ven a verme!	&	Le damos una orden o instrucción a otra persona. \\
    \hline
    Exclamativo & ¡Viva la libertad! & Expresamos una emoción o un deseo. \\
    \hline
    Interrogativo & ¿Está lloviendo? & Solicitamos información sobre un evento o situación. \\
    \hline
    Informativo & El trabajo es muy complicado. & Transmitimos información que puede ser falsa o verdadera. En este caso el trabajo puede ser cierto que sea complicado, o puede ser falso y el trabajo en realidad ser sencillo. \\
    \hline
  \end{tabular}
  }}
  \jump
  \bolder{Solo la función informativa corresponde a una proposición.}
\end{frame}

%%%%%%%%%%%%%%%%%%%%%%%%%%%%%%%%%%%%%%%%%%%%%%%%%%%%%

\begin{frame}{Ejemplos de proposiciones}
  A continuación hay algunos ejemplos de proposiciones.
  Determinemos, según nuestro criterio, cuales son \fulltrue y
  cuales son \fullfalse.
  \begin{itemize}
    \item La Tierra es plana
    \item Está lloviendo en este lugar
    \item La UNQ está en Bernal
    \item Argentina ganó el mundial de 1978
    \item Los perros tienen cuatro patas (salvo amputaciones)
  \end{itemize}
\end{frame}

%%%%%%%%%%%%%%%%%%%%%%%%%%%%%%%%%%%%%%%%%%%%%%%%%%%%%

\begin{frame}{Ejemplos de proposiciones - Cont}
  Algunos ejemplos en donde en donde para poder responder las
  proposiciones necesitamos información que, a priori no
  conocemos.
  \begin{itemize}
    \item Todas las células eucariotas tienen pared celular
    \item El transistor más pequeño del mundo mide 14nm
    \item Argentina ganará el mundial FIFA Rusia 2018
    \item Toyota es uno de los mayores fabricantes de autos del mundo
  \end{itemize}
\end{frame}

%%%%%%%%%%%%%%%%%%%%%%%%%%%%%%%%%%%%%%%%%%%%%%%%%%%%%
%%%%%%%%%%%%%%%%%%%%%%%%%%%%%%%%%%%%%%%%%%%%%%%%%%%%%

\subsection{Uso de conectivas en proposiciones}

%%%%%%%%%%%%%%%%%%%%%%%%%%%%%%%%%%%%%%%%%%%%%%%%%%%%%

\begin{frame}{Oraciones compuestas}
  Hay casos en donde las oraciones son un poco más complejas. Pensemos el
  siguiente ejemplo:
  \jump
  \colored{El avión se estrelló en la cordillera o realizó un aterrizaje de emergencia}
  \jump
  Si miramos la oración, veremos que, en realidad, hay dos partes bien
  demarcadas. Es decir, si esto fuera una pregunta, sería una pregunta
  compuesta de dos partes: \colored{``¿El avión se estrelló en la cordillera?''}
  y \colored{``¿El avión realizó un aterrizaje de emergencia?''}.
\end{frame}

%%%%%%%%%%%%%%%%%%%%%%%%%%%%%%%%%%%%%%%%%%%%%%%%%%%%%

\begin{frame}{Oraciones compuestas - Cont}
  El truco está en que la oración contiene una conectiva, ``o'', que nos indica
  que la misma puede separarse en dos partes.
  \jump
  El avión se estrelló en la cordillera \colored{o} realizó un aterrizaje de emergencia
  \jump
  Esto nos indica que \bolder{la oración está formada por dos proposiciones}.
  \jump
  \bolder{El valor de verdad de la oración, dependerá entonces del valor
  de verdad de las proposiciones y de la conectiva que las une} (de
  la misma forma que sucedía cuando uníamos preguntas)
  \jump
  \bolder{Las proposiciones (tantas como se quiera) pueden unirse en formas tan
  complejas como sea necesario, usando las conectivas de conjunción, disyunción y negación}
\end{frame}

%%%%%%%%%%%%%%%%%%%%%%%%%%%%%%%%%%%%%%%%%%%%%%%%%%%%%

\begin{frame}{Proposiciones compuestas}
  Llamamos a las proposiciones que no tienen conectivas
  \bolder{proposiciones atómicas} y a las que tienen conectivas y están
  formadas por otras proposiciones \bolder{proposiciones compuestas}.
  \jump
  Una proposiciones compuesta puede estar formada a su vez por otras
  proposiciones compuestas, o por proposiciones simples.
  \jump
  Eventualmente, siempre debe haber una proposición simple.
\end{frame}

%%%%%%%%%%%%%%%%%%%%%%%%%%%%%%%%%%%%%%%%%%%%%%%%%%%%%

\begin{frame}{Conectivas en el lenguaje natural}
  Veamos ahora el siguiente caso
  \jump
  \colored{Tal vez el avión se estrelló en la cordillera , tal vez realizó un aterrizaje de emergencia.}
  \jump
  Si analizamos la oración, veremos que no cambia el significado con respecto a
  la anterior, aunque tiene una diferencia en su redacción.
  \jump
  \bolder{En el lenguaje natural las conectivas pueden tomar diversas formas.}
  \jump
  Algunas resultan bastante obvias, otras a las que estamos menos habituados,
  resultan al principio complicadas de interpretar correctamente.
  \jump
  A continuación mostramos algunos ejemplos de formas que pueden tomar diversas
  conectivas:
\end{frame}

%%%%%%%%%%%%%%%%%%%%%%%%%%%%%%%%%%%%%%%%%%%%%%%%%%%%%

\begin{frame}{Conectivas en el lenguaje natural}
  \begin{tabular}{| l | p{8cm} |}
    \hline
    Conjunción & y, también, además, adicionalmente, en adición, ahora,
                incluso, inclusive, así mismo, de igual forma, del mismo modo,
                igualmente, sin embargo, no obstante, pero, pese ..., empero,
                aunque, aun así, a pesar de que, tanto como, al igual que,
                por otra parte, más, mas, ``no... ni...'' (el ``ni'' también es una negación),
                aparte, así mismo, ``por otro lado'' \\
    \hline
    Disyunción &  o, ``tal vez ..., tal vez ...'', de pronto, de pronto también,
                  aunque de pronto, puede, aunque puede \\
    \hline
    Negación & no, no es cierto que, no es verdad que, ni (también indica conjunción) \\
    \hline
  \end{tabular}
\end{frame}

%%%%%%%%%%%%%%%%%%%%%%%%%%%%%%%%%%%%%%%%%%%%%%%%%%%%%

\begin{frame}{Análisis del lenguaje natural}
  Pensemos ahora la siguiente oración:
  \jump
  \colored{El avión se estrelló en la cordillera o realizó un aterrizaje de emergencia y
  se encuentra incomunicado.}
  \jump
  Como primer paso, intentemos buscar las conectivas que hay en dicha oración,
  e identifiquemos las afirmaciones por partes.
\end{frame}

%%%%%%%%%%%%%%%%%%%%%%%%%%%%%%%%%%%%%%%%%%%%%%%%%%%%%

\begin{frame}{Análisis del lenguaje natural - Cont}
  Tenemos aquí 2 conectivas, ``o'' e ``y ''.
  \jump
  El avión se estrelló en la cordillera \colored{o} realizó un aterrizaje de
  emergencia \colored{y} se encuentra incomunicado.
  \jump
  Esto divide la oración en tres partes:
  \begin{enumerate}
    \item El avión se estrelló en la cordillera
    \item El avión realizó un aterrizaje de emergencia
    \item El avión se encuentra incomunicado
  \end{enumerate}
  \jump
  \small{Notece que la oración no dice explicitamente ``El avión se encuentra incomunicado'',
  sino solamente ``se encuentra incomunicado''. Pero ahí hay un sujeto tácito.
  Si pensamos quien se encuentra incomunicado, claramente refiere al avión.
  \jump
  Al momento de realizar el análisis, es conveniente transcribir cada proposición,
  entera colocando claramente el sujeto y el predicado.
  }
\end{frame}

%%%%%%%%%%%%%%%%%%%%%%%%%%%%%%%%%%%%%%%%%%%%%%%%%%%%%

\begin{frame}{Análisis del lenguaje natural - Cont}
  Ahora bien, tenemos una oración que tendrá un valor de verdad, y que está
  compuesta por tres proposiciones más pequeñas y dos conectivas, una conjunción
  y una disyunción.
  \jump
  Pero, ¿Cuál es la formula que corresponde a dicha oración?
  \jump
  Para simplificar la fórmula asignemos una letra a cada oración:
  \begin{itemize}
    \item e = El avión se estrelló en la cordillera
    \item a = El avión realizó un aterrizaje de emergencia
    \item i = El avión se encuentra incomunicado
  \end{itemize}
  \jump
  De esta forma, cuando nombremos a ``e'' en la fórmula, sabremos que se
  corresponde con la proposición ``El avión se estrelló en la cordillera''.
\end{frame}

%%%%%%%%%%%%%%%%%%%%%%%%%%%%%%%%%%%%%%%%%%%%%%%%%%%%%

\begin{frame}{Análisis del lenguaje natural - Cont}
  Tenemos entonces que la formula es:
  \jump
  \centerline{\bolder{$e \lor a \land i$}}
  \jump
  Pero, como ya vimos la clase pasada, las conectivas actúan entre dos preguntas,
  en este caso, entre dos proposiciones. Para desambiguar, necesitamos colocar
  paréntesis. Tenemos entonces dos posibilidades:
  \begin{columns}
    \begin{column}{0.5\textwidth}
      \centerline{\bolder{$(e \lor a) \land i$}}
    \end{column}
    \begin{column}{0.5\textwidth}
      \centerline{\bolder{$e \lor (a \land i)$}}
    \end{column}
  \end{columns}
\end{frame}

%%%%%%%%%%%%%%%%%%%%%%%%%%%%%%%%%%%%%%%%%%%%%%%%%%%%%

\begin{frame}{Análisis del lenguaje natural - Cont}
  ¿Será lo mismo una que la otra?. Para analizarlo, basta con ver la tabla de
  verdad, y ver que las valuaciones que hacen que la primer fórmula sea \fulltrue,
  no son necesariamente las mismas que hacen que la segunda lo sea.
  \jump
  \begin{columns}
    \begin{column}{0.5\textwidth}
      \centerline{
        \begin{tabular}{c c c | c || c}
          $e$ & $a$ & $i$ & $e \lor a$ & $(e \lor a) \land i$ \\
          \hline
          \true  & \true  & \true  & \true  & \true  \\
          \true  & \true  & \false & \true  & \false \\
          \true  & \false & \true  & \true  & \true  \\
          \true  & \false & \false & \true  & \false \\
          \false & \true  & \true  & \true  & \true  \\
          \false & \true  & \false & \true  & \false \\
          \false & \false & \true  & \false & \false \\
          \false & \false & \false & \false & \false \\
        \end{tabular}
      }
    \end{column}
    \begin{column}{0.5\textwidth}
      \centerline{
        \begin{tabular}{c c c | c || c}
          $e$ & $a$ & $i$ & $a \land i$ & $e \lor (a \land i)$ \\
          \hline
          \true  & \true  & \true  & \true  & \true  \\
          \true  & \true  & \false & \false & \true  \\
          \true  & \false & \true  & \false & \true  \\
          \true  & \false & \false & \false & \true  \\
          \false & \true  & \true  & \true  & \true  \\
          \false & \true  & \false & \false & \false \\
          \false & \false & \true  & \false & \false \\
          \false & \false & \false & \false & \false \\
        \end{tabular}
      }
    \end{column}
  \end{columns}
\end{frame}

%%%%%%%%%%%%%%%%%%%%%%%%%%%%%%%%%%%%%%%%%%%%%%%%%%%%%

\begin{frame}{Análisis del lenguaje natural}
  Entonces, ¿Qué fórmula corresponde a la frase? ¿Cuál usamos?
  \jump
  Para muchas disciplinas que utilizan la lógica como ciencia de apoyo poder
  analizar correctamente una oración es de vital importancia.
  \jump
  Imaginemos por ejemplo un abogado que intenta probar la inocencia de su
  cliente, o un sociólogo intentando analizar un discurso político.
  \jump
  Por ese motivo las ciencias empíricas y formales intentan evitar el uso de
  lenguaje ambiguo dentro de sus libros, e intentar dejar muy en claro qué
  fórmula es la que corresponde a un texto, mediante la misma redacción.
\end{frame}

%%%%%%%%%%%%%%%%%%%%%%%%%%%%%%%%%%%%%%%%%%%%%%%%%%%%%

\begin{frame}{Proposiciones ocultas}
  Más aún, tomemos el siguiente ejemplo:
  \jump
  \colored{La luz está prendida o la luz está apagada}
  \jump
  A simple vista, claramente hay dos proposiciones y una conectiva, ``o''.
  \begin{enumerate}
    \item La luz está prendida
    \item La luz está apagada
  \end{enumerate}
  \jump
  Sin embargo, conociendo la naturaleza de las luces, sabemos que una luz tiene
  solo dos estados, prendida o apagada (olvidemos las luces con atenuación). Si
  una luz no está prendida, entonces está apagada, y si no está apagada, entonces
  está prendida.
  \jump
  Es decir, nuestras dos proposiciones son complementarias, por lo tanto, una
  expresarse en términos de la otra utilizando la conectiva de negación.
\end{frame}

%%%%%%%%%%%%%%%%%%%%%%%%%%%%%%%%%%%%%%%%%%%%%%%%%%%%%

\begin{frame}{Proposiciones ocultas}
  Así, la frase anterior puede reescribirse como:
  \jump
  \colored{La luz está prendida o la luz no está prendida}
  \jump
  Ahora si, vemos claramente que hay una sola proposición (``La luz está prendida'')
  y dos conectivas, ``o'' y ``no''.
  \jump
  La lógica nos permite razonar mejor acerca de lo que dice el lenguaje natural,
  pues este es muy amplio y ambiguo. Sin embargo, analizar el lenguaje natural
  requiere mucha práctica y observación cuidadosa sobre lo que estamos diciendo.
\end{frame}

%%%%%%%%%%%%%%%%%%%%%%%%%%%%%%%%%%%%%%%%%%%%%%%%%%%%%
%%%%%%%%%%%%%%%%%%%%%%%%%%%%%%%%%%%%%%%%%%%%%%%%%%%%%

\section{Razonamientos.}
\toc[currentsection,currentsubsection]

%%%%%%%%%%%%%%%%%%%%%%%%%%%%%%%%%%%%%%%%%%%%%%%%%%%%%

\begin{frame}{Oraciones que se relacionan}
  Hasta ahora venimos analizando oraciones, que puede darse el caso que sean
  proposiciones, u oraciones compuestas por proposiciones y conectivas.
  \jump
  La gracia de la lógica es poder comprender como varias oraciones se
  interrelacionan, y poder extraer información útil de dichas relaciones.
  \jump
  El pasar la oración a una fórmula, no es el objetivo final, sino que es
  una parte del proceso que nos va a permitir expresar y analizar las oraciones
  y sus relaciones.
  \jump
  \bolder{El objetivo final es poder elaborar y analizar razonamientos}.
\end{frame}

%%%%%%%%%%%%%%%%%%%%%%%%%%%%%%%%%%%%%%%%%%%%%%%%%%%%%

\begin{frame}{Razonamientos}
  \begin{block}{Razonamiento}
    Un razonamiento es la actividad de la mente que permite inferir
    necesariamente una conclusión a partir de una serie de premisas.
  \end{block}
  \jump
  Las \bolder{premisas} no son más que \bolder{proposiciones} con las que
  contamos a priori como información que \bolder{asumimos como verdadera}.
  \jump
  Una \bolder{conclusión} es simplemente una \bolder{nueva proposición}, la
  cual, \bolder{si el razonamiento es correcto y las premisas eran efectivamente
  verdaderas, será verdadera}.
\end{frame}

%%%%%%%%%%%%%%%%%%%%%%%%%%%%%%%%%%%%%%%%%%%%%%%%%%%%%
%%%%%%%%%%%%%%%%%%%%%%%%%%%%%%%%%%%%%%%%%%%%%%%%%%%%%

\subsection{Razonamientos en el lenguaje natural}

%%%%%%%%%%%%%%%%%%%%%%%%%%%%%%%%%%%%%%%%%%%%%%%%%%%%%

\begin{frame}{Razonamientos en el lenguaje natural}
  Al momento de hablar, elaboramos muchos razonamientos. Expresamos posibilidades
  en base a hipótesis que poseemos previamente. Pensemos en el siguiente ejemplo:
  \jump
  \colored{La Tierra es plana o es redonda. La Tierra no es plana. Por lo tanto,
  la Tierra es redonda.}
  \jump
  Tenemos nuestra hipótesis, ``La tierra es plana o redonda'', no hay otra
  posibilidad. Como llegamos a dicha hipótesis no es relevante.
  \jump
  Tenemos un conocimiento adicional, ``La tierra no es plana''. Tampoco sabemos
  como arribamos a eso, tal vez como resultado de un experimento. Nuevamente,
  no importa.
  \jump
  Elaboramos una conclusión en base a la hipótesis y al conocimiento adicional,
  ``La tierra es redonda''.
  \jump
  Pero, ¿Cómo sabemos que es un razonamiento? Porque encontramos un indicador
  de conclusión.
\end{frame}

%%%%%%%%%%%%%%%%%%%%%%%%%%%%%%%%%%%%%%%%%%%%%%%%%%%%%

\begin{frame}{Indicadores de conclusión}
  Un indicador de conclusión es una palabra (o conjunto de palabras) que indica
  que la oración que viene a continuación corresponde a la conclusión de una
  serie de premisas que fueron expresadas anteriormente.
  \jump
  Algunos ejemplos son:
  \begin{columns}
    \begin{column}{0.5\textwidth}
      \begin{itemize}
        \item Por lo tanto
        \item En consecuencia
        \item Se concluye que
        \item Se deduce
      \end{itemize}
    \end{column}
    \begin{column}{0.5\textwidth}
      \begin{itemize}
        \item Por ende
        \item Luego
        \item Entonces
        \item Así pues
      \end{itemize}
    \end{column}
  \end{columns}
  \jump
  Nuevamente, el lenguaje natural es muy amplio, y por tanto hay muchas
  posibilidades, poder identificar claramente cuando se habla de una conclusión
  es muy útil para gente que estudia las ciencias sociales. A nosotros nos va
  a interesar, pero más nos va a importar ver que los razonamientos tengan sentido.
\end{frame}

%%%%%%%%%%%%%%%%%%%%%%%%%%%%%%%%%%%%%%%%%%%%%%%%%%%%%

\begin{frame}{Razonamientos en el lenguaje natural}
  Volviendo al ejemplo, tenemos tres oraciones, de las cuales,
  dos son premisas, y una es conclusión.
  \jump
  \begin{itemize}
    \item La Tierra es plana o es redonda.
    \item La Tierra no es plana.
    \item la Tierra es redonda.
  \end{itemize}
  \jump
  Si analizamos el conjunto de oraciones, y buscamos las conectivas
  podremos ver que en realidad hay solo dos proposiciones base:
  ``La Tierra es plana'' y ``La Tierra es redonda''. Luego, las
  oraciones son proposiciones compuestas con estas dos y conectivas.
  \jump
  Extrapolemos la fórmula
\end{frame}

%%%%%%%%%%%%%%%%%%%%%%%%%%%%%%%%%%%%%%%%%%%%%%%%%%%%%

\begin{frame}{Razonamientos en el lenguaje más formal}
  Nos quedarían las siguientes tres oraciones:
  \begin{itemize}
    \item La Tierra es plana $\lor$ La Tierra es redonda.
    \item $\lnot$ La Tierra es plana.
    \item La Tierra es redonda.
  \end{itemize}
  \jump
  \bolder{Los razonamientos pueden tener muchas premisas, pero siempre hay
  una única conclusión}.
  \jump
  \bolder{Por tanto, en un razonamiento solo puede haber un único indicador
  de conclusión}.
\end{frame}

%%%%%%%%%%%%%%%%%%%%%%%%%%%%%%%%%%%%%%%%%%%%%%%%%%%%%

\begin{frame}{Razonamientos en el lenguaje más formal}
  Los razonamientos suelen escribirse en lenguaje formal colocando cada
  premisa en un renglón aparte, y la conclusión como última línea, separando
  las premisas de la conclusión por una línea horizontal.
  \jump
  \colored{
    \begin{lreasoning}
      \lpremise{La Tierra es plana $\lor$ La Tierra es redonda}
      \lpremise{$\lnot$ La Tierra es plana}
      \lconclusion{La Tierra es redonda}
    \end{lreasoning}
  }
  \jump
  Otra forma de escribir lo mismo es:
  \jump
  \colored{La Tierra es plana $\lor$ La Tierra es redonda, $\lnot$ La Tierra es
    plana $\lseq$ La Tierra es redonda}
\end{frame}

%%%%%%%%%%%%%%%%%%%%%%%%%%%%%%%%%%%%%%%%%%%%%%%%%%%%%

\begin{frame}{Otro formato de razonamientos}
  Veamos ahora el siguiente razonamiento:
  \jump
  \colored{Sabemos que la Tierra es redonda. Esto es así porque la Tierra es
  redonda o plana. Y sabemos que la Tierra no es plana}
  \jump
  Si pensamos lo que nos cuenta la oración, veremos que es lo mismo que lo que
  nos decía la anterior. Dada la información de que la Tierra no es plana, y
  nuestra hipótesis, podemos saber que la Tierra es redonda.
  \jump
  Pero aquí no hay un indicador de conclusión. De hecho, la conclusión está
  antes que el resto de la información. En este caso tenemos \bolder{indicadores
  de premisa}
\end{frame}

%%%%%%%%%%%%%%%%%%%%%%%%%%%%%%%%%%%%%%%%%%%%%%%%%%%%%

\begin{frame}{Indicadores de premisa}
  Un indicador de premisa es una palabra (o conjunto de palabras) que indica
  que la oración o conjunto de oraciones que viene a continuación corresponde
  a premisas de una conclusión que fue expresada anteriormente.
  \jump
  Algunos ejemplos son:
  \begin{columns}
    \begin{column}{0.5\textwidth}
      \begin{itemize}
        \item Dado qué
        \item Ya qué
        \item Esto es así porque
        \item Porque
      \end{itemize}
    \end{column}
    \begin{column}{0.5\textwidth}
      \begin{itemize}
        \item Esto se sigue de 
        \item En vista de que 
        \item Pues
      \end{itemize}
    \end{column}
  \end{columns}
  \jump
  Además, observemos como en este último ejemplo, hay partes de la oración que
  no añaden valor relevante, sino que simplemente brindan estructura y estilo,
  como es el caso de ``Sabemos que''.
\end{frame}

%%%%%%%%%%%%%%%%%%%%%%%%%%%%%%%%%%%%%%%%%%%%%%%%%%%%%

\begin{frame}{Cuando vale un razonamiento}
  A simple vista pareciera que el razonamiento anterior tiene mucho sentido,
  por lo tanto, esperamos que valga.
  \jump
  ¿Pero que significa ``que valga''?
  \jump
  Como ya dijimos, \bolder{un razonamiento vale cuando, si las premisas son \fulltrue,
  entonces la conclusión necesariamente es \fulltrue}. En otras palabras,
  decimos que \bolder{las premisas implican lógicamente la conclusión}.
\end{frame}

%%%%%%%%%%%%%%%%%%%%%%%%%%%%%%%%%%%%%%%%%%%%%%%%%%%%%
%%%%%%%%%%%%%%%%%%%%%%%%%%%%%%%%%%%%%%%%%%%%%%%%%%%%%

\subsection{Implicación}

%%%%%%%%%%%%%%%%%%%%%%%%%%%%%%%%%%%%%%%%%%%%%%%%%%%%%

\begin{frame}{Implicación}
  Para analizar si una serie de premisas \bolder{implica lógicamente}
  la conclusión, primero tenemos que agregar una nueva conectiva, la
  \bolder{implicación}.
  \jump
  La implicación une dos proposiciones, llamadas \bolder{antecedente} y
  \bolder{consecuente}.
  \jump
  La implicación lo que dice es qué, de cumplirse el antecedente,
  el consecuente debe también cumplirse si o si para que la implicación
  sea cierta. Es decir, si el antecedente es \fulltrue, entonces el
  consecuente también debe ser \fulltrue.
  \jump
  La implicación no nos dice que pasa con el consecuente en el caso de que
  no se cumpla el antecedente. De ser el antecedente \fullfalse, el consecuente
  puede valer tanto \fullfalse como \fulltrue.
\end{frame}

%%%%%%%%%%%%%%%%%%%%%%%%%%%%%%%%%%%%%%%%%%%%%%%%%%%%%

\begin{frame}{Implicación - Cont}
  Asumamos entonces dos proposiciones cualquiera, $p$ y $q$, donde $p$ será
  el antecedente y $q$ será el consecuente.
  \jump
  El signo de implicación es $\lthen$, y diremos que $p$ implica lógicamente a
  $q$ escribiendo ``$p \lthen q$''. El signo suele leerse coloquialmente como
  ``entonces'' (es decir, leemos ``p entonces q'').
  \jump
  A la implicación le corresponde la siguiente tabla de verdad:
  \centerline{
    \begin{tabular}{c c | c}
      $p$ & $q$ & $p \lthen q$ \\
      \hline
      \true  & \true  & \true \\
      \true  & \false & \false \\
      \false & \true  & \true \\
      \false & \false & \true \\
    \end{tabular}
  }
  \jump
  El caso en que el antecedente es falso suele prestarse a confusión.
  Recordemos que la implicación solo me dice que pasa si el antecedente se
  cumple, no en caso contrario.
\end{frame}

%%%%%%%%%%%%%%%%%%%%%%%%%%%%%%%%%%%%%%%%%%%%%%%%%%%%%
%%%%%%%%%%%%%%%%%%%%%%%%%%%%%%%%%%%%%%%%%%%%%%%%%%%%%

\subsection{Prueba de validez de un razonamiento}

%%%%%%%%%%%%%%%%%%%%%%%%%%%%%%%%%%%%%%%%%%%%%%%%%%%%%

\begin{frame}{Probando si un razonamiento tiene sentido}
  Ahora si podemos probar si un razonamiento es válido o no. Podemos hacer
  esto mediante el proceso de tabla de verdad.
  \jump
  Para ello tenemos que tomar cada fórmula correspondiente a una premisa, y
  unirlas a todas en una única fórmula, utilizando conjunciones. Debemos recordar
  de colocar la fórmula que corresponde a cada premisa entre paréntesis para evitar
  ambigüedades.
  \jump
  Así en el ejemplo anterior tenemos dos premisas:
  \begin{itemize}
    \item La Tierra es plana $\lor$ La Tierra es redonda.
    \item $\lnot$ La Tierra es plana.
  \end{itemize}
  \jump
  Si las unimos nos queda la siguiente fórmula:\\
  \colored{(La Tierra es plana $\lor$ La Tierra es redonda) $\land$ ($\lnot$ La Tierra es plana)}
\end{frame}

%%%%%%%%%%%%%%%%%%%%%%%%%%%%%%%%%%%%%%%%%%%%%%%%%%%%%

\begin{frame}{Probando si un razonamiento tiene sentido}
  Ahora, esta nueva fórmula deberá implicar lógicamente a la conclusión. Ponemos
  tanto a todas las premisas, como a la conclusión entre paréntesis y armamos una
  única formula, donde las premisas implican a la conclusión.
  \jump
  \colored{((La Tierra es plana $\lor$ La Tierra es redonda) $\land$ ( $\lnot$ La Tierra es plana)) $\lthen$ (La Tierra es redonda)}
  \jump
  Como venimos haciendo hasta ahora, asignaremos la letra ``p'' a la proposición
  ``La Tierra es plana'' y la letra ``r'' a la proposición ``La Tierra es redonda''
  para poder escribir y trabajar más cómodamente la fórmula.
  \jump
  Eso nos da lo siguiente:
  \colored{$((p \lor r) \land (\lnot p)) \lthen (r)$}
\end{frame}

%%%%%%%%%%%%%%%%%%%%%%%%%%%%%%%%%%%%%%%%%%%%%%%%%%%%%

\begin{frame}{Probando si un razonamiento tiene sentido}
  Nos es necesario armar la tabla de verdad de dicha fórmula. La dejamos expresada
  a continuación:
  \jump
  \small{\centerline{
    \begin{tabular}{c c | c | c | c || c}
      && Premisa 1 & Premisa 2 & & Premisas implican conclusión \\
      \hline
      $p$ & $r$ & $p \lor r$ & $\lnot p$ & $(p \lor r) \land (\lnot p)$ & $((p \lor r) \land (\lnot p)) \lthen (r)$ \\
      \hline
      \true  & \true  & \true  & \false & \false & \true  \\
      \true  & \false & \true  & \false & \false & \true  \\
      \false & \true  & \true  & \true  & \true  & \true  \\
      \false & \false & \false & \true  & \false & \true  \\
    \end{tabular}
  }}
\end{frame}

%%%%%%%%%%%%%%%%%%%%%%%%%%%%%%%%%%%%%%%%%%%%%%%%%%%%%

\begin{frame}{Análisis de la tabla de verdad}
  Analicemos primero la conjunción de las premisas.
  Esa tabla solo dará \fulltrue cuando las premisas sean todas \fulltrue.
  Todos los casos en donde alguna de las premisas era \fullfalse, hace que la
  conjunción sea \fullfalse.
  \jump
  Fijemonos en la implicación final. Si alguna de las premisas era \fullfalse,
  esto hace que el antecedente de la implicación sea \fullfalse, y por tanto,
  la implicación es \fulltrue (antecedente falso, da verdadero siempre, sin
  importar el consecuente).
  \jump
  Solo en aquellos casos en donde el antecedente es \fulltrue, puede la implicación
  dar resultados distintos, y dichos resultados dependen del consecuente. En este
  caso, el consecuente es la conclusión. Para que la implicación sea \fulltrue,
  la conclusión debe ser \fulltrue.
\end{frame}

%%%%%%%%%%%%%%%%%%%%%%%%%%%%%%%%%%%%%%%%%%%%%%%%%%%%%

\begin{frame}{Análisis de la tabla de verdad}
  Recordemos:
  \jump
  \bolder{En un razonamiento válido, si las premisas son verdaderas, entonces
  la conclusión es necesariamente verdadera.}
  \jump
  Eso es lo que prueba nuestra tabla de verdad. La conjunción nos marca que
  todas las premisas deben ser verdadera. En ese momento, la implicación final,
  nos dará verdadero solo si necesariamente la conclusión también lo es.
  \jump
  Si por algún motivo la conclusión nos diera \fullfalse cuando las premisas son
  todas \fulltrue, la implicación dará \fullfalse por resultado.
  \jump
  \bolder{Es decir, si la implicación final es una tautología, entonces se trata
  de un razonamiento válido. Caso contrario, diremos que el razonamiento es inválido.}
\end{frame}

%%%%%%%%%%%%%%%%%%%%%%%%%%%%%%%%%%%%%%%%%%%%%%%%%%%%%

\begin{frame}{Probando si un razonamiento tiene sentido}
  Recordemos, un razonamiento es válido cuando, si las premisas son verdaderas,
  entonces la conclusión también lo es.
  \jump
  Marquemos primero aquellas filas en donde la primer premisa es verdadera.
  \jump
  \small{\centerline{
    \begin{tabular}{c c | c | c | c || c}
      && Premisa 1 & Premisa 2 & & Conclusión \\
      \hline
      $p$ & $r$ & $p \lor r$ & $\lnot p$ & $(p \lor r) \land (\lnot p)$ & $((p \lor r) \land (\lnot p)) \lthen (r)$ \\
      \hline
      \rowcolor{gray}
      \true  & \true  & \true  & \false & \false & \true  \\
      \rowcolor{gray}
      \true  & \false & \true  & \false & \false & \true  \\
      \rowcolor{gray}
      \false & \true  & \true  & \true  & \true  & \true  \\
      \false & \false & \false & \true  & \false & \true  \\
    \end{tabular}
  }}
\end{frame}

%%%%%%%%%%%%%%%%%%%%%%%%%%%%%%%%%%%%%%%%%%%%%%%%%%%%%
%%%%%%%%%%%%%%%%%%%%%%%%%%%%%%%%%%%%%%%%%%%%%%%%%%%%%

\subsection{Implicación como condicional}

%%%%%%%%%%%%%%%%%%%%%%%%%%%%%%%%%%%%%%%%%%%%%%%%%%%%%

\begin{frame}{Otros usos de la implicación}
  La implicación también puede usarse en oraciones que no necesariamente
  expresan un razonamiento.
  \jump
  Por ejemplo, la siguiente oración:
  \jump
  \colored{Si la Tierra es plana entonces nos caeremos por el borde de la misma}
  \jump
  Aquí hay dos proposiciones, aunque no se noten a simple vista.
  \begin{enumerate}
    \item La Tierra es plana
    \item Nos caeremos por el borde de la Tierra
  \end{enumerate}
\end{frame}

%%%%%%%%%%%%%%%%%%%%%%%%%%%%%%%%%%%%%%%%%%%%%%%%%%%%%

\begin{frame}{Implicación como condicional}
  Decimos que la segunda proposición, ``Nos caeremos por el borde de la Tierra'',
  está condicionada por la primera ``La Tierra es plana''.
  \jump
  Sin embargo, la segunda proposición no se desprende necesariamente de la
  primera. Podríamos caernos por el borde de la Tierra si la misma fuera cúbica
  y no plana.
  \jump
  Es decir, el valor de verdad de cada una de estas proposiciones es
  independiente. Por lo tanto, no se trata de un razonamiento.
  \jump
  La implicación en este caso actúa como condición. Nos caeremos por el borde
  necesariamente si la tierra es plana, pero podríamos, o no, caernos en el caso
  de que no lo sea.
\end{frame}

%%%%%%%%%%%%%%%%%%%%%%%%%%%%%%%%%%%%%%%%%%%%%%%%%%%%%

\begin{frame}{Implicación como condicional}
  La implicación suele estar dada en el lenguaje natural por palabras similares
  a las que usamos como indicador de conclusión, como ``entonces'', ``así'',
  ``luego'', etc.
  \jump
  Pensemos en el siguiente ejemplo:
  \jump
  \colored{Si la Tierra es plana entonces nos caeremos por el borde. Si la
    Tierra es redonda entonces no nos caeremos por el borde. La Tierra es
    redonda o plana. La Tierra no es plana. Por lo tanto, no nos caeremos por
    el borde.}
  \jump
  Note como hay varias palabras que podrían ser indicadores de conclusión.
\end{frame}

%%%%%%%%%%%%%%%%%%%%%%%%%%%%%%%%%%%%%%%%%%%%%%%%%%%%%

\begin{frame}{Implicación como condicional}
  Si la Tierra es plana \colored{entonces} nos caeremos por el borde. Si la
  Tierra es redonda \colored{entonces} no nos caeremos por el borde. La Tierra es
  redonda o plana. La Tierra no es plana. \colored{Por lo tanto}, no nos caeremos
  por el borde.
  \jump
  Recordemos. Todo razonamiento tiene una o más premisas, y una y solo una
  conclusión. Por tanto, solo puede haber un indicador de conclusión.
  \jump
  El problema aquí es que estamos confundiendo indicadores de conclusión, con un
  indicador de la conectiva de implicación dentro de la oración.
\end{frame}

%%%%%%%%%%%%%%%%%%%%%%%%%%%%%%%%%%%%%%%%%%%%%%%%%%%%%

\begin{frame}{Implicación como condicional}
  Si analizamos las oraciones y las conectivas, el razonamiento quedará de la
  siguiente forma:\\
  \jump
  \colored{
    \begin{lreasoning}
      \lpremise{La Tierra es plana $\lthen$ Nos caeremos por el borde}
      \lpremise{La Tierra es redonda $\lthen \lnot$ Nos caeremos por el borde}
      \lpremise{La Tierra es redonda $\lor$ La Tierra es plana}
      \lpremise{$\lnot$ La Tierra es plana}
      \lconclusion{$\lnot$ Nos caeremos por el borde}
    \end{lreasoning}
  }
  \jump
  Como vemos sigue habiendo solo una conclusión, pero hay condiciones dentro
  de las premisas.
\end{frame}

%%%%%%%%%%%%%%%%%%%%%%%%%%%%%%%%%%%%%%%%%%%%%%%%%%%%%
%%%%%%%%%%%%%%%%%%%%%%%%%%%%%%%%%%%%%%%%%%%%%%%%%%%%%

\section{Equivalencias lógicas.}
\toc[currentsection,currentsubsection]

%%%%%%%%%%%%%%%%%%%%%%%%%%%%%%%%%%%%%%%%%%%%%%%%%%%%%

\begin{frame}{Distintos e iguales}
  Luego de tanto análisis, cabe preguntarse si dos oraciones que intuitivamente
  creemos que dicen lo mismo, realmente lo hacen.
  \jump
  Pensemos en las siguientes dos oraciones
  \begin{itemize}
    \item \colored{La Tierra gira alrededor del Sol y La Tierra gira sobre su eje}
    \item \colored{La Tierra gira sobre su eje y La Tierra gira alrededor del Sol}
  \end{itemize}
  \jump
  Será cierto que estas dos oraciones significan lo mismo, más allá que el orden
  en el que dicen las cosas no es el mismo. ¿Cómo podemos estar seguros?
\end{frame}

%%%%%%%%%%%%%%%%%%%%%%%%%%%%%%%%%%%%%%%%%%%%%%%%%%%%%

\begin{frame}{Equivalencia lógica}
  \begin{block}{Equivalencia lógica}
    Decimos que dos oraciones son lógicamente equivalentes si para toda valuación
    posible tiene exactamente el mismo valor de verdad en ambas oraciones.
  \end{block}
  \jump
  Dicho de otra forma, si dos oraciones tienen la misma tabla de verdad, entonces
  son equivalentes.
  \jump
  Ojo, equivalentes no significa iguales. Las oraciones pueden ser distintas,
  pero en términos lógicos expresan lo mismo.
\end{frame}

%%%%%%%%%%%%%%%%%%%%%%%%%%%%%%%%%%%%%%%%%%%%%%%%%%%%%

\begin{frame}{Equivalencia lógica}
  De hecho, si sabemos que dos oraciones son lógicamente equivalentes, entonces
  es indistinto cual usemos.
  \jump
  Las equivalencias entre formulas son tan útiles que es conveniente poder
  expresar que una fórmula es equivalente a otra.
  \jump
  Sin embargo, no podemos usar el signo de igualdad (``='') pues no es cierto
  que sean iguales.
\end{frame}

%%%%%%%%%%%%%%%%%%%%%%%%%%%%%%%%%%%%%%%%%%%%%%%%%%%%%

\subsection{Bicondicional}

%%%%%%%%%%%%%%%%%%%%%%%%%%%%%%%%%%%%%%%%%%%%%%%%%%%%%

\begin{frame}{Bicondicional}
  El \bolder{bicondicional} o \bolder{doble implicación} es una conectiva que
  indica equivalencia lógica.
  \jump
  Es común encontrarnos dicha conectiva utilizada como una implicación doble,
  es decir, donde el antecedente implica al consecuente, y el consecuente implica
  al antecedente.
  \jump
  El bicondicional se representa con $\liff$ (una flecha doble) y se suele leer
  como ``si y solo si''.
\end{frame}

%%%%%%%%%%%%%%%%%%%%%%%%%%%%%%%%%%%%%%%%%%%%%%%%%%%%%

\begin{frame}{Bicondicional}
  Dadas dos proposiciones cualquiera $p$ y $q$, decimos que $p$ vale si y solo
  si vale $q$ escribiendo $p \liff q$.
  \jump
  El bicondicional tiene la siguiente tabla de verdad:
  \jump
  \centerline{
    \begin{tabular}{c c | c}
      $p$ & $q$ & $p \liff q$ \\
      \hline
      \true  & \true  & \true  \\
      \true  & \false & \false \\
      \false & \true  & \false \\
      \false & \false & \true  \\
    \end{tabular}  
  }
  \jump
  Como vemos, el bicondicional solo da verdadero cuando ambas proposiciones
  tienen el mismo valor de verdad.
\end{frame}

%%%%%%%%%%%%%%%%%%%%%%%%%%%%%%%%%%%%%%%%%%%%%%%%%%%%%

\begin{frame}{Ejemplo Bicondicional}
  La mejor forma de entenderlo es con algunos ejemplos. Las siguientes son
  oraciones que suelen traducirse con bicondicionales:
  \begin{itemize}
    \item Utilizaré limones para el bizcochuelo si y solo si no hay naranjas en la verdulería.
    \item La Tierra es redonda si y solo si podemos darle la vuelta sin caernos.
    \item Si y solo si existe vida inteligente en otros planetas, sufriremos una
    invasión extraterrestre.
  \end{itemize}
\end{frame}

%%%%%%%%%%%%%%%%%%%%%%%%%%%%%%%%%%%%%%%%%%%%%%%%%%%%%
%%%%%%%%%%%%%%%%%%%%%%%%%%%%%%%%%%%%%%%%%%%%%%%%%%%%%

\section{Formalización.}
\subsection{Variables proposicionales.}
\toc[currentsection,currentsubsection]

%%%%%%%%%%%%%%%%%%%%%%%%%%%%%%%%%%%%%%%%%%%%%%%%%%%%%

\begin{frame}{Variables proposicionales}
  A la lógica, en tanto ciencia formal, no le va a interesar tanto las oraciones
  con las que trabajemos, sino la forma que tienen nuestras proposiciones y
  razonamientos.
  \jump
  Así, no solo vamos a extrapolar siempre la formula para poder trabajarla, sino
  que vamos a reemplazar las proposiciones atómicas por \bolder{variables proposicionales}.
  \jump
  Una \bolder{variable proposicional} consiste en una letra (generalmente, pero
  no necesariamente, comenzando desde la $p$ y siguiendo el orden alfabético). A
  dicha letra le corresponde semánticamente una proposición atómica.
\end{frame}

%%%%%%%%%%%%%%%%%%%%%%%%%%%%%%%%%%%%%%%%%%%%%%%%%%%%%

\begin{frame}{Variables proposicionales - Ejemplo}
  Así por ejemplo, la proposición \colored{``La Tierra gira en torno al Sol y
  La Tierra gira sobre su eje''}, como ya vimos, está compuesta de dos
  proposiciones atómicas. A cada proposición atómica le asignaremos una letra:
  \jump
  \begin{itemize}
    \item $p$ = La Tierra gira en torno al Sol
    \item $q$ = La Tierra gira sobre su eje
  \end{itemize}
  \jump
  De esta forma, la proposición compuesta en lenguaje formal quedaría como
  ``$p \land q$''.
  \jump
  A la asociación de letras con nociones semánticas se la conoce como \bolder{diccionario
  del lenguaje}.
\end{frame}

%%%%%%%%%%%%%%%%%%%%%%%%%%%%%%%%%%%%%%%%%%%%%%%%%%%%%

\begin{frame}{Variables proposicionales - Ejemplo}
  Pensemos a que frase corresponde la fórmula ``$p \lor q$''.
  \jump
  La primer definición que podríamos pensar es:
  \colored{``La Tierra gira en torno al Sol o La Tierra gira sobre su eje''}
  \jump
  Esa definición mantiene a $p$ y a $q$ con las mismas definiciones semánticas
  que tenían en la formula anterior. Sin embargo, si no se provee información
  de cual es el diccionario, $p$ y $q$ podrían ser cualquier cosa.
  \jump
  Con un diccionario distinto, la siguiente oración también corresponde a la
  fórmula $p \lor q$: \colored{``Llueve en Buenos Aires o nieva en Bariloche''}.
\end{frame}

%%%%%%%%%%%%%%%%%%%%%%%%%%%%%%%%%%%%%%%%%%%%%%%%%%%%%

\begin{frame}{Elección de las variables proposicionales}
  Es decir que en cada oración que analicemos, podemos reutilizar las mismas
  letras que ya utilizamos en análisis previos.
  \jump
  Si siempre seguimos la lógica de usar las letras comenzando por $p$ y
  siguiendo por $q$, $r$, etc. dos oraciones que tienen la misma forma
  deberían producir la misma fórmula.
  \jump
  Otra opción es elegir letras que sean representativas de lo que estamos
  hablando, por ejemplo, la proposición ``La Tierra gira sobre su eje'' podríamos
  denotarla con la letra $e$, por ``eje''.
\end{frame}

%%%%%%%%%%%%%%%%%%%%%%%%%%%%%%%%%%%%%%%%%%%%%%%%%%%%%

\begin{frame}{Elección de las variables proposicionales}
  La elección de la letra es arbitraria, y dos fórmulas separadas con la misma
  forma, aunque distintas letras pueden ser consideradas iguales.
  \jump
  Pensemos el ejemplo \colored{``La Tierra gira en torno al Sol y La Tierra gira sobre su eje''}.
  Si elegimos el siguiente diccionario:
  \begin{itemize}
    \item $p$ = La Tierra gira en torno al Sol
    \item $q$ = La Tierra gira sobre su eje
  \end{itemize}
  Obtendremos \colored{``$p \land q$''} como fórmula.
  \jump
  Si elegimos el siguiente diccionario:
  \begin{itemize}
    \item $s$ = La Tierra gira en torno al Sol
    \item $e$ = La Tierra gira sobre su eje
  \end{itemize}
  Entonces la fórmula será \colored{``$s \land e$''}
\end{frame}

%%%%%%%%%%%%%%%%%%%%%%%%%%%%%%%%%%%%%%%%%%%%%%%%%%%%%

\begin{frame}{Elección de las variables proposicionales}
  Si las dos fórmulas son independientes, ``$p \land q$'' y $s \land e$ son
  iguales, pues la letra usada es un producto de una selección arbitraria.
  \jump
  Vamos a decir que dos fórmulas son iguales si, modificando la asignación de
  letras en nuestro diccionario, podemos transformar la primer fórmula en la
  segunda, o viceversa.
\end{frame}

%%%%%%%%%%%%%%%%%%%%%%%%%%%%%%%%%%%%%%%%%%%%%%%%%%%%%

\begin{frame}{Variables proposicionales en razonamientos}
  Al analizar un razonamiento, debemos analizar todo el conjunto de oraciones
  para armar un único diccionario para el razonamiento entero.
  \jump
  Esto quiere decir que si una misma proposición aparece dos veces en distintas
  oraciones, le deberá corresponder la misma variable proposicional en ambas.
  \jump
  Por ejemplo, el siguiente razonamiento: \colored{``La Tierra es plana o
  redonda. La Tierra no es plana. Por lo tanto, la Tierra es redonda.''}
  \jump
  En este ejemplo, si analizamos de forma independiente las oraciones ``La Tierra es plana o
  redonda'' y la conclusión ``la Tierra es redonda'', podríamos acabar asignándole
  la fórmula ``$p \lor q$'' a la primera, y la fórmula ``$r$'' a la conclusión.
  Esto no es correcto, pues ``la Tierra es redonda'', debe ser asociado a la misma
  variable en todas las oraciones del razonamiento.
  \jump
  Así, las fórmulas de dicho razonamiento podrían ser ``$p \lor q$'', ``$\lnot p$''
  y ``$q$''.
\end{frame}

%%%%%%%%%%%%%%%%%%%%%%%%%%%%%%%%%%%%%%%%%%%%%%%%%%%%%

\begin{frame}{Fórmulas que expresan razonamientos}
  Ya habíamos visto que podemos expresar un razonamiento colocando
  cada premisa en un nuevo renglón, y finalmente, separada por una
  linea horizontal, la conclusión debajo.
  \jump
  Para el ejemplo anterior, esto quedaría así:\\
  \jump
  \colored{
  \begin{lreasoning}
    \lpremise{$p \lor q$}
    \lpremise{$\lnot p$}
    \lconclusion{$q$}
  \end{lreasoning}
  }
  \jump
  Otra forma de escribir lo mismo, como ya vimos, es pasar el razonamiento a
  una única fórmula, uniendo cada premisa con una conjunción, y estas, implicando
  lógicamente a la conclusión:\\
  \colored{$((p \lor q) \land (\lnot p)) \lthen (q)$}
\end{frame}

%%%%%%%%%%%%%%%%%%%%%%%%%%%%%%%%%%%%%%%%%%%%%%%%%%%%%

\begin{frame}{Fórmulas que expresan razonamientos}
  La primera tiene el problema de que requiere bastante espacio para expresar
  el razonamiento (múltiples renglones).
  \jump
  La segunda se vuelve complicada cuando las oraciones tienen muchas conectivas,
  y no queda claro donde empieza una premisa y donde termina. Además requiere la
  adición de muchos paréntesis que en ocaciones no ayudan, sino que ensucian
  la fórmula.
  \jump
  Por esto, existe y se usa mucho una tercer forma de escribir un razonamiento,
  el cual consiste en colocar las fórmulas para cada premisa separadas por comas,
  colocar un signo especial ``$\lseq$'' (al que llamamos secuente), y luego
  colocar la conclusión.
  \jump
  \colored{$p \lor q, \lnot p \lseq q$}
\end{frame}


%%%%%%%%%%%%%%%%%%%%%%%%%%%%%%%%%%%%%%%%%%%%%%%%%%%%%
%%%%%%%%%%%%%%%%%%%%%%%%%%%%%%%%%%%%%%%%%%%%%%%%%%%%%

\subsection{Fórmulas válidas}

%%%%%%%%%%%%%%%%%%%%%%%%%%%%%%%%%%%%%%%%%%%%%%%%%%%%%

\begin{frame}{Forma de las fórmulas}
  Como ya vimos, toda conectiva trabaja siempre sobre dos proposiciones (las
  cuales pueden ser simples o compuestas).
  \jump
  Necesitamos adicionar paréntesis para desambiguar acerca de cuales son las
  proposiciones afectadas por una conectiva.
  \jump
  A continuación veamos cuales son las formas que puede adoptar una proposición.
\end{frame}

%%%%%%%%%%%%%%%%%%%%%%%%%%%%%%%%%%%%%%%%%%%%%%%%%%%%%

\begin{frame}{Forma de las fórmulas}
  \begin{enumerate}
    \item Toda variable proposicional es una fórmula válida.
    \item Si \colored{$\Phi$} es una fórmula válida, entonces \colored{($\lnot \Phi$)} también lo es
    \item Si \colored{$\Phi$} y \colored{$\Psi$} son fórmulas válidas, entonces:
      \begin{itemize}
        \item \colored{$(\Phi \land \Psi)$} es una fórmula válida
        \item \colored{$(\Phi \lor \Psi)$} es una fórmula válida
        \item \colored{$(\Phi \lthen \Psi)$} es una fórmula válida
        \item \colored{$(\Phi \liff \Psi)$} es una fórmula válida
      \end{itemize}
  \end{enumerate}
  \jump
  Note como, si se usa una conectiva, se debe aplicar paréntesis.
\end{frame}

%%%%%%%%%%%%%%%%%%%%%%%%%%%%%%%%%%%%%%%%%%%%%%%%%%%%%

\begin{frame}{Forma de las fórmulas}
  Así, la siguiente fórmula es válida:
  \jump
  \colored{$((p \land q) \liff (\lnot q))$}
  \jump
  Pero esta no lo es:
  \jump
  \colored{$p \land q \liff \lnot q$}
  \jump
  Las reglas de producción de formulas nos indican además que cosas como
  \colored{$p \lnot q$} o \colored{$p \land \lor q$} no son válidas.
\end{frame}

%%%%%%%%%%%%%%%%%%%%%%%%%%%%%%%%%%%%%%%%%%%%%%%%%%%%%

\begin{frame}{Reglas de precedencia}
  Fórmulas muy grandes tendrán por tanto muchos paréntesis. Para evitar escribir
  tantos paréntesis podemos definir algunas reglas, que indican como desambiguar
  situaciones como \colored{$(p \lor q) \liff p$} y \colored{$p \lor (q \liff p)$}
  cuando se omiten los paréntesis.
  \jump
  \begin{enumerate}
    \item Omitimos siempre los paréntesis más externos
    \item La negación asocia más fuerte que todos los demás
    \item La conjunción asocia más fuerte que todas, menos la negación
    \item La disyunción asocia más fuerte que la implicación y el bicondicional
    \item La implicación asocia más fuerte que el bicondicional, pero menos que todo el resto
    \item El bicondicional asocia más débil
  \end{enumerate}
\end{frame}

%%%%%%%%%%%%%%%%%%%%%%%%%%%%%%%%%%%%%%%%%%%%%%%%%%%%%

\begin{frame}{Reglas de precedencia}
  Que quiere decir esto. Si tenemos la siguiente fórmula \colored{$\lnot p \lor q$},
  debemos interpretarla \colored{$((\lnot p) \lor q)$}. Como la negación asocia
  más fuerte, salvo que haya paréntesis, la negación es lo primero a resolver.
  \jump
  De forma similar \colored{$p \lor q \lthen r$} debe interpretarse como
  \colored{$((p \lor q) \lthen r)$} pues la disyunción asocia más fuerte que
  la implicación.
\end{frame}

%%%%%%%%%%%%%%%%%%%%%%%%%%%%%%%%%%%%%%%%%%%%%%%%%%%%%
%%%%%%%%%%%%%%%%%%%%%%%%%%%%%%%%%%%%%%%%%%%%%%%%%%%%%

\subsection{Resumen}

%%%%%%%%%%%%%%%%%%%%%%%%%%%%%%%%%%%%%%%%%%%%%%%%%%%%%

\begin{frame}{Resumen}
  Recapitulemos hasta acá.
  \jump
  \begin{itemize}
    \item La lógica proposicional o de orden cero, es la lógica que tiene por objeto de
      estudio las proposiciones.
    \item Una proposición es una oración escrita utilizando la función informativa del lenguaje.
    \item Toda proposición tiene un valor de verdad, ya sea \fulltrue o \fullfalse.
    \item Las proposiciones pueden ser atómicas, es decir sin tener conectivas.
    \item Las proposiciones pueden unirse entre si con conectivas, para dar lugar a
      proposiciones compuestas.
    \item Una proposición puede actuar como premisa, o como conclusión dentro
      de un razonamiento.
    \item En una fórmula lógica identificamos cada proposición atómica con una
      variable proposicional distinta.
  \end{itemize}
\end{frame}

%%%%%%%%%%%%%%%%%%%%%%%%%%%%%%%%%%%%%%%%%%%%%%%%%%%%%

\begin{frame}{Resumen - Cont}
  \begin{itemize}
    \item Una razonamiento consiste en una o más premisas y una y solo una
      conclusión.
    \item Los razonamientos pueden ser válidos o inválidos.
    \item Un razonamiento es válido cuando, si las premisas son verdaderas,
      entonces la conclusión necesariamente es verdadera.
    \item Podemos probar la validez de los razonamientos mediante el uso de
      tablas de verdad.
  \end{itemize}
\end{frame}

%%%%%%%%%%%%%%%%%%%%%%%%%%%%%%%%%%%%%%%%%%%%%%%%%%%%%

\begin{frame}{Resumen Conectivas}
  Conectivas lógicas y sus tablas de verdad:
  \jump
  \begin{columns}
    \begin{column}{0.33\textwidth}
      \centerline{Conjunción}
      \centerline{
        \begin{tabular}{c c | c}
          $p$ & $q$ & $p \land q$ \\
          \hline
          \true  & \true  & \true  \\
          \true  & \false & \false \\
          \false & \true  & \false \\
          \false & \false & \false \\
        \end{tabular}
      }
    \end{column}
    \begin{column}{0.33\textwidth}
      \centerline{Disyunción}
      \centerline{
        \begin{tabular}{c c | c}
          $p$ & $q$ & $p \lor q$ \\
          \hline
          \true  & \true  & \true  \\
          \true  & \false & \true  \\
          \false & \true  & \true  \\
          \false & \false & \false \\
        \end{tabular}
      }
    \end{column}
    \begin{column}{0.33\textwidth}
      \centerline{Negación}
      \centerline{
        \begin{tabular}{c | c}
          $p$ & $\lnot p$ \\
          \hline
          \true  & \false \\
          \false & \true  \\
        \end{tabular}
      }
    \end{column}
  \end{columns}
  \begin{columns}
    \begin{column}{0.5\textwidth}
      \centerline{Implicación}
      \centerline{
        \begin{tabular}{c c | c}
          $p$ & $q$ & $p \lthen q$ \\
          \hline
          \true  & \true  & \true  \\
          \true  & \false & \false \\
          \false & \true  & \true  \\
          \false & \false & \true  \\
        \end{tabular}
      }
    \end{column}
    \begin{column}{0.5\textwidth}
      \centerline{Bicondicional}
      \centerline{
        \begin{tabular}{c c | c}
          $p$ & $q$ & $p \liff q$ \\
          \hline
          \true  & \true  & \true  \\
          \true  & \false & \false \\
          \false & \true  & \false \\
          \false & \false & \true  \\
        \end{tabular}
      }
    \end{column}
  \end{columns}
\end{frame}

%%%%%%%%%%%%%%%%%%%%%%%%%%%%%%%%%%%%%%%%%%%%%%%%%%%%%

\begin{frame}{Resumen - Cont}
  \begin{itemize}
    \item La lógica trabaja a nivel sintáctico, por lo que le importa la forma,
      y no el contenido de las proposiciones.
    \item Una fórmula está compuesta de variables proposicionales y de conectivas
      que las unen.
    \item Omitimos paréntesis en las fórmulas con el objetivo de simplificar la
      lectura, pero las fórmulas no se prestan a ambigüedades.
  \end{itemize}
\end{frame}
