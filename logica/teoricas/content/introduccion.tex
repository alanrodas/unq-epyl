\section{¿Qué es la lógica?}
\subsection{Definición.}
\toc[currentsection,currentsubsection]

%%%%%%%%%%%%%%%%%%%%%%%%%%%%%%%%%%%%%%%%%%%%%%%%%%%%%

\begin{frame}{¿Qué es la lógica?}
  \begin{block}{Lógica}
    La lógica es la ciencia formal que estudia los principios de la
    demostración y la inferencia válida.
  \end{block}
  \jump
  O sea, es una ciencia muuuuuuy amplia que se se dedica a estudiar como
  razonamos los seres humanos, cuando un razonamiento es válido y cuando no,
  como describir el mundo de forma precisa y sin ambigüedades, entre otras
  cosas.
  \jump
  Comencemos por analizar la definición por partes.
\end{frame}

%%%%%%%%%%%%%%%%%%%%%%%%%%%%%%%%%%%%%%%%%%%%%%%%%%%%%

\begin{frame}{Ciencia formal}
  Se dice que es una \bolder{ciencia formal} pues estudia las formas, y no los
  objetos.
  \jump
  Otra ciencia formal que conocemos es la matemática.
  \jump
  Por ejemplo, en matemática estudiamos como se suman números, por ejemplo,
  $1+2$. Pero no nos centramos en si estamos sumando manzanas, naranjas o
  bananas, nos importa la forma (las cantidades) no los objetos.
  \jump
  La lógica, como veremos luego, es similar en ese sentido.
\end{frame}

%%%%%%%%%%%%%%%%%%%%%%%%%%%%%%%%%%%%%%%%%%%%%%%%%%%%%

\begin{frame}{Demostración}
  Una \bolder{demostración} es una prueba, matemática, que mediante el uso
  de teoremas y axiomas, argumenta de forma deductiva (luego veremos que
  significa esto) la verdad de una proposición matemática.
  \jump
  O sea, la lógica asiste en la matemática para demostrar que algunas cosas
  realmente valen.
  \jump
  En matemática, si no podemos demostrarlo, entonces no se puede demostrar su
  validez ni su invalidez.
\end{frame}

%%%%%%%%%%%%%%%%%%%%%%%%%%%%%%%%%%%%%%%%%%%%%%%%%%%%%

\begin{frame}{Inferencias válidas}
  La inferencia es el proceso mental mediante el cual, a partir de cierta
  información dada, se obtienen conclusiones. Es decir, se llega a un nuevo
  resultado a partir de cosas que ya se tienen, mediante la razón.
  \jump
  Por supuesto que llegar a resultados a partir de información previa,
  puede hacerse de forma correcta o incorrecta. Si razonamos mal, entonces
  las conclusiones a las que arribamos no van a tener sentido.
  \jump
  Nos interesa por tanto solo aquellos razonamientos que son válidos, es decir,
  en donde el proceso de razonamiento es correcto. Más adelante veremos que
  procesos son correctos y cuales no.
\end{frame}

%%%%%%%%%%%%%%%%%%%%%%%%%%%%%%%%%%%%%%%%%%%%%%%%%%%%%

\begin{frame}{Distintos tipos de lógicas y razonamientos}
  La lógica es tan amplia y asiste a tantas disciplinas, que hay distintos
  sistemas formales y distintos enfoques que pueden aplicarse.
  \jump
  Esto no quiere decir que sean independientes y distintas, sino que todas usan
  los mismos principios subyacentes.
  \jump
  Nos vamos a centrar en analizar la lógica desde el punto de vista de la
  matemática y las ciencias de la computación, y nos vamos a olvidar
  completamente del resto.
\end{frame}

%%%%%%%%%%%%%%%%%%%%%%%%%%%%%%%%%%%%%%%%%%%%%%%%%%%%%
%%%%%%%%%%%%%%%%%%%%%%%%%%%%%%%%%%%%%%%%%%%%%%%%%%%%%

\subsection{Razonamientos.}
\toc[currentsection,currentsubsection]

%%%%%%%%%%%%%%%%%%%%%%%%%%%%%%%%%%%%%%%%%%%%%%%%%%%%%

\begin{frame}{Razonamientos}
  Un razonamiento es entonces el proceso mediante el cual, partiendo de cierta
  información previa, a las que se denomina \bolder{premisas}, arribamos a una
  nueva información, a la que se denomina \bolder{conclusión}.
  \jump
  Hay varios tipos de razonamientos, que dependen de la forma:
  \begin{itemize}
    \item Razonamientos inductivos por enumeración
    \item Razonamientos inductivos por analogía
    \item Razonamientos deductivos
  \end{itemize}
\end{frame}

%%%%%%%%%%%%%%%%%%%%%%%%%%%%%%%%%%%%%%%%%%%%%%%%%%%%%

\begin{frame}{Razonamientos inductivos por enumeración}
  Son razonamientos que consisten en observar una serie de casos particulares
  que hablan sobre individuos de cierto tipo, que cumplen alguna propiedad.
  A partir de analizar esos casos, se arriba a que todos los individuos de ese
  tipo deberían cumplir esa propiedad.
  \jump
  Ejemplo
  \begin{enumerate}
    \item Vi un primer cuervo y era negro
    \item Vi un segundo cuervo y era negro
    \item Vi un tercer cuervo y era negro
    \item ...
    \item Vi el cuervo 500 y era negro
  \end{enumerate}
  Por tanto, concluyo que \colored{todos los cuervos son negros}.
\end{frame}

%%%%%%%%%%%%%%%%%%%%%%%%%%%%%%%%%%%%%%%%%%%%%%%%%%%%%

\begin{frame}{Razonamientos inductivos por enumeración}
  Surgen varias preguntas:
  \begin{itemize}
    \item ¿Son 500 cuervos suficiente muestra?
    \item ¿Cómo se que no había un cuervo 501 que fuera blanco?
    \item ¿Es imposible que mañana nazca un nuevo cuervo que sea blanco?
  \end{itemize}
  Este tipo de razonamientos se usan mucho en las ciencias empíricas (biología,
  química, física, etc.) y sirven para, a partir de una serie de experimentos,
  arribar a una conclusión probabilística.
  \jump
  Por ejemplo, si hago 1000 experimentos en el éxito de un medicamento para curar
  algo y solo 5 resultan fallidos, puedo aseverar que el medicamento funciona en
  el 99.95\% de los casos.
  \jump
  \bolder{La conclusión de un razonamiento inductivo no es necesariamente veradera}
\end{frame}

%%%%%%%%%%%%%%%%%%%%%%%%%%%%%%%%%%%%%%%%%%%%%%%%%%%%%

\begin{frame}{Razonamientos inductivos por analogía}
  Son razonamientos que consisten en observar una serie de casos particulares
  que hablan sobre individuos de cierto tipo, que cumplen alguna propiedad.
  A partir de analizar esos casos, se arriba a que el siguiente individuo que
  encuentre deberá cumplir dicha propiedad.
  \jump
  Ejemplo
  \begin{enumerate}
    \item Juan asaltó un banco y fue condenado a 20 años de prisión
    \item Luis asaltó un banco y fue condenado a 20 años de prisión
    \item María asaltó un banco y fue condenada a 20 años de prisión
    \item ...
    \item Ana asaltó un banco y fue condenada a 20 años de prisión
  \end{enumerate}
  Por tanto, \bolder{José quien acaba de asaltar un banco será condenado a
  20 años de prisión}.
\end{frame}

%%%%%%%%%%%%%%%%%%%%%%%%%%%%%%%%%%%%%%%%%%%%%%%%%%%%%

\begin{frame}{Razonamientos inductivos por analogía}
  Este tipo de razonamientos son muy utilizados en ciencias legales, en casos
  de jurisprudencia. También en ciencias empíricas.
  \jump
  Los razonamientos inductivos por analogía sufren de las mismas problemáticas
  que los inductivos por enumeración.
  \jump
  \bolder{La conclusión de un razonamiento inductivo no es necesariamente veradera}
\end{frame}

%%%%%%%%%%%%%%%%%%%%%%%%%%%%%%%%%%%%%%%%%%%%%%%%%%%%%

\begin{frame}{Razonamientos deductivos}
  Un razonamiento deductivo es aquel en el que las premisas implican lógica y
  necesariamente a la conclusión.
  \jump
  \bolder{Es decir, en un razonamiento deductivo, si las premisas son verdaderas, la
  conclusión necesariamente deberá ser verdadera.}
  \jump
  En un razonamiento deductivo, la conclusión no contiene nueva información,
  sino que contiene la misma información que tienen las premisas, replanteada
  de alguna forma particular.
  \jump
  Este tipo de razonamiento es el utilizado en la matemática y en las ciencias
  de la computación. Es el único razonamiento que nos va a interesar en esta
  materia.
\end{frame}

%%%%%%%%%%%%%%%%%%%%%%%%%%%%%%%%%%%%%%%%%%%%%%%%%%%%%

\begin{frame}{Razonamientos deductivos}
  El ejemplo clásico de razonamiento deductivo es:
  \begin{enumerate}
    \item Todos los humanos son mortales.
    \item Sócrates es humano.
    \item Por lo tanto, \colored{Sócrates es mortal}.
  \end{enumerate}
  \jump
  Sin embargo, no todos los razonamientos deductivos tienen esa forma.
  Más adelante veremos que hay formas, o esquemas, que hacen que un razonamiento
  sea válido.
\end{frame}

%%%%%%%%%%%%%%%%%%%%%%%%%%%%%%%%%%%%%%%%%%%%%%%%%%%%%

\begin{frame}{Razonamientos deductivos incorrectos}
  Un razonamiento deductivo puede arribar a conclusiones falsas en dos casos:
  \begin{enumerate}
    \item Cuando el razonamiento es inválido (tiene un esquema incorrecto)
    \item Cuando alguna o varias de las premisas (la información de partida) es falsa
  \end{enumerate}
  \jump
  Ejemplo de razonamiento inválido.
  \begin{enumerate}
    \item Todos los humanos son mortales
    \item Sócrates es un humano
    \item Por lo tanto, \colored{Sócrates no es mortal}.
  \end{enumerate}
  \jump
  Ejemplo de razonamiento válido con premisas falsas
  \begin{enumerate}
    \item Todos los humanos son inmortales
    \item Sócrates es un humano
    \item Por lo tanto, \colored{Sócrates es inmortal}.
  \end{enumerate}
\end{frame}

%%%%%%%%%%%%%%%%%%%%%%%%%%%%%%%%%%%%%%%%%%%%%%%%%%%%%

\begin{frame}{Válidez de un razonamiento}
  La validez de un razonamiento depende solamente de su estructura,
  independientemente del valor de sus premisas y de su conclusión.
  \jump
  En el segundo ejemplo de la diapositiva anterior, el razonamiento es
  válido, aunque la conclusión sea falsa pues se parte de premisas falsas.
  \jump
  En cambio el primer ejemplo es inválido, pues su forma no es válida.
  \jump
  Lo importante es: \bolder{Si un razonamiento deductivo es válido y parte
  de premisas verdaderas, siempre la conclusión será verdadera}.
\end{frame}

%%%%%%%%%%%%%%%%%%%%%%%%%%%%%%%%%%%%%%%%%%%%%%%%%%%%%

\begin{frame}{Estructura de razonamiento}
  El ejemplo clásico de razonamiento deductivo es:
  \begin{enumerate}
    \item Todos los humanos son mortales.
    \item Sócrates es humano.
    \item Por lo tanto, \colored{Sócrates es mortal}.
  \end{enumerate}
  \jump
  Otro ejemplo que sigue la misma estructura es:
  \begin{enumerate}
    \item Todos los perros son animales.
    \item Firulais es un perro.
    \item Por lo tanto, \colored{Firulais es un animal}.
  \end{enumerate}
  \jump
  Sin embargo, no todos los razonamientos deductivos tienen esa forma,
  más adelante veremos distintos formas que pueden tomar estos razonamientos.
\end{frame}

%%%%%%%%%%%%%%%%%%%%%%%%%%%%%%%%%%%%%%%%%%%%%%%%%%%%%
%%%%%%%%%%%%%%%%%%%%%%%%%%%%%%%%%%%%%%%%%%%%%%%%%%%%%

\section{La lógica como descripción del universo.}
\toc[currentsection,currentsubsection]

%%%%%%%%%%%%%%%%%%%%%%%%%%%%%%%%%%%%%%%%%%%%%%%%%%%%%

\begin{frame}{La lógica como descripción del universo - Ejemplo}
  La lógica tiene la capacidad de describir sin ambigüedades el universo,
  o al menos, una parte de él que nos resulte relevante.
  \jump
  Pensemos en el caso de una escuela rural en donde solo cursan dos chicos y
  dos chicas: Juan, Luis, María y Ana.
  \jump
  Estas personas tienen afinidad entre si en algunos casos, y entre otros no.
  Por lo tanto, vamos a decir que un chico ``quiere'' a otro si se da el caso de que
  tienen afinidad. Puede darse el caso de que un chico se quiera o no se quiera
  a si mismo.
\end{frame}

%%%%%%%%%%%%%%%%%%%%%%%%%%%%%%%%%%%%%%%%%%%%%%%%%%%%%

\begin{frame}{La lógica como descripción del universo - Ejemplo Cont}
  Podemos representar el universo con una tabla de doble entrada que leeremos de
  izquierda a derecha, como la siguiente:
  \jump
  \centerline{\begin{tabular}{| c | c | c | c | c |}
    \hline
          & Juan & Luis & María & Ana \\
    \hline
    Juan  & X &   & X &   \\
    \hline
    Luis  &   & X & X & X \\
    \hline
    María & X &   & X &   \\
    \hline
    Ana   & X &   &   & X \\
    \hline
  \end{tabular}}
  \jump
  En esta tabla podemos ver lo siguiente:
  \begin{itemize}
    \item Todos los chicos se quieren a si mismos
    \item Juan quiere a María
    \item María quiere a Juan
    \item Luis quiere a María y a Ana
    \item Ana quiere a Juan
  \end{itemize}
\end{frame}

%%%%%%%%%%%%%%%%%%%%%%%%%%%%%%%%%%%%%%%%%%%%%%%%%%%%%

\begin{frame}{Cantidad de posibles universos}
  Hay muchos posibles universos considerando solo nuestra escuela de $4$ chicos.
  Si una persona puede o no querer a otra, es decir hay dos posibilidades, y
  hay 16 posibles combinaciones de esas posibilidades, eso implica que existen
  $2^{16}$ (o $65.536$) universos distintos posibles.
  \jump
  Cuanta más información tenemos acerca del universo más podemos describirlo.
  Descripciones pobres nos imposibilitan saber exactamente de que universo se trata.
\end{frame}

%%%%%%%%%%%%%%%%%%%%%%%%%%%%%%%%%%%%%%%%%%%%%%%%%%%%%

\begin{frame}{La lógica como descripción del universo - Ejemplo}
  Hagamos ahora el paso inverso. Partiendo de información, intentemos determinar
  el estado del universo.
  \jump
  \begin{itemize}
    \item Nadie se quiere a si mismo
    \item Todo chico quiere a una y solo una chica
    \item Toda chica quiere a un y solo un chico
    \item Ningún chico quiere a un chico
    \item Ninguna chica quiere a una chica
    \item Ana quiere a Juan
    \item María quiere a Juan
  \end{itemize}
  \jump
  Si completamos nuestro cuadro veremos ahora que hay información que no podremos
  completar.
  \centerline{\begin{tabular}{| c | c | c | c | c |}
    \hline
          & Juan & Luis & María & Ana \\
    \hline
    Juan  &   &   & ? & ? \\
    \hline
    Luis  &   &   & ? & ? \\
    \hline
    María & X &   &   &   \\
    \hline
    Ana   & X &   &   &   \\
    \hline
  \end{tabular}}
\end{frame}

%%%%%%%%%%%%%%%%%%%%%%%%%%%%%%%%%%%%%%%%%%%%%%%%%%%%%

\begin{frame}{La lógica como descripción del universo - Ejemplo Cont}
  Hay 4 posibles universos que cumplen la condición.
  \begin{enumerate}
    \item Juan quiere a María y Luis quiere a María
    \item Juan quiere a María y Luis quiere a Ana
    \item Juan quiere a Ana y Luis quiere a María
    \item Juan quiere a Ana y Luis quiere a Ana
  \end{enumerate}
  \jump
  La información que tenemos no nos determina exactamente el universo, pero nos
  da suficiente información para algunas cosas.
  \jump
  Por ejemplo, tal vez mi interés pase por saber si Juan tiene chances de
  ponerse de novio con alguna de las chicas. No puedo saber con qué chica, pues
  no puedo saber exactamente si Juan quiere a María o a Ana, pero se que quiere
  a alguna de las dos, y que ambas lo quieren a el. Por tanto, seguro podrá
  ponerse de novio. Por otro lado, sabemos que Luis no tendrá chance de hacerlo.
\end{frame}

%%%%%%%%%%%%%%%%%%%%%%%%%%%%%%%%%%%%%%%%%%%%%%%%%%%%%

\begin{frame}{La lógica como descripción del universo}
  \bolder{Lo importante es que la lógica nos va a permitir saber cómo es el
  universo que se describe, y nos va a permitir describir un universo que
  observemos.}
  \jump
  Vamos a usar activamente esta propiedad para describir e interpretar cosas.
\end{frame}

%%%%%%%%%%%%%%%%%%%%%%%%%%%%%%%%%%%%%%%%%%%%%%%%%%%%%
%%%%%%%%%%%%%%%%%%%%%%%%%%%%%%%%%%%%%%%%%%%%%%%%%%%%%

\section{La lógica como generador de preguntas.}
\toc[currentsection,currentsubsection]

%%%%%%%%%%%%%%%%%%%%%%%%%%%%%%%%%%%%%%%%%%%%%%%%%%%%%

\begin{frame}{La lógica como generador de preguntas}
  Otra cosa que vamos a querer hacer es usar la lógica para formular
  preguntas más precisas y con menos ambigüedad.
  \jump
  Por ejemplo, formular preguntas acerca del universo, y, en base a la
  respuesta, poder determinar en que condiciones se encuentra el universo.
\end{frame}

%%%%%%%%%%%%%%%%%%%%%%%%%%%%%%%%%%%%%%%%%%%%%%%%%%%%%

\begin{frame}{La lógica como formulador de preguntas - Ejemplo}
  Imaginemos que nuestro universo es una gran verdulería. Un verdulero místico
  va a responder todas nuestras preguntas con una de dos posibles palabras,
  o bien con ``Si'' o bien con ``No''.
  \jump
  Ahora bien, a nuestro verdulero podemos preguntarle cosas como
  \begin{itemize}
    \item ¿Hay bananas?
    \item ¿Hay manzanas?
    \item ¿Hay naranjas?
  \end{itemize}
  El verdulero místico se fijará en su stock, y nos responderá a cada pregunta
  con si o no.
\end{frame}

%%%%%%%%%%%%%%%%%%%%%%%%%%%%%%%%%%%%%%%%%%%%%%%%%%%%%

\begin{frame}{La lógica como formulador de preguntas - Ejemplo Cont}
  Nuestro verdulero sabe mucho sobre las frutas y verduras que tiene en stock,
  y podemos preguntarle si tiene cualquiera de ellas. Pero nada sabe sobre
  recetas de cocina.
  \jump
  Supongamos ahora que queremos preparar una ensalada de frutas (usando bananas,
  manzanas y naranjas). No podemos simplemente preguntarle al verdulero
  ``¿Hay para una ensalada de frutas?'' pues no tiene idea de como se hace
  una ensalada de frutas.
  \jump
  Sin embargo, podemos reformular la pregunta en términos de cosas que el
  verdulero si sabe responder. Por ejemplo ``¿Hay bananas, manzanas y naranjas?''
  Es decir, en una sola pregunta englobamos tres preguntas más sencillas:
  \begin{enumerate}
    \item ¿Hay bananas?
    \item ¿Hay manzanas?
    \item ¿Hay naranjas?
  \end{enumerate}
  \jump
  Si la respuesta a todas ellas es afirmativa, sabemos que hay frutas suficientes
  para preparar una ensalada de frutas.
\end{frame}

%%%%%%%%%%%%%%%%%%%%%%%%%%%%%%%%%%%%%%%%%%%%%%%%%%%%%

\begin{frame}{La lógica como formulador de preguntas - Ejemplo Cont}
  Podríamos también querer preparar un buen bizcochuelo. Todo buen bizcochuelo
  lleva, o bien ralladura de limón, o bien ralladura de naranja.
  \jump
  En este caso cualquiera de las dos nos sirve. Por lo que preguntar
  ``¿Hay para preparar un bizcochuelo?'' podría ser equivalente a preguntar
  ``¿Hay naranjas o limones?''.
  \jump
  Es decir, se involucran dos preguntas más simples: 
  \begin{enumerate}    
    \item ¿Hay naranjas?
    \item ¿Hay limones?
  \end{enumerate}
  \jump
  Si la respuesta a cualquiera de ellas es afirmativa, podremos preparar un
  bizcochuelo.
\end{frame}

%%%%%%%%%%%%%%%%%%%%%%%%%%%%%%%%%%%%%%%%%%%%%%%%%%%%%

\begin{frame}{La lógica como formulador de preguntas - Ejemplo Cont}
  Nótese que preguntar ``¿Hay naranjas o limones?'' es muy distinto a
  preguntar ``¿Hay naranjas y limones?''. En el primer caso nos basta con
  que alguna de las respuestas sea afirmativa, mientras que en el segundo
  esperaríamos que ambas lo sean.
  \jump
  Comprender esas pequeñas diferencias y entender cuando utilizar un ``y'' y
  cuando un ``o'' es fundamental para un programador, y es parte de lo que
  estudia la lógica y que trataremos en futuras clases.
\end{frame}