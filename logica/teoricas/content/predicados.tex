
\section{Lógica de Predicados}
\subsection{Limitaciones de la lógica proposicional}
\frame{\tableofcontents[currentsection,currentsubsection]}

%%%%%%%%%%%%%%%%%%%%%%%%%%%%%%%%%%%%%%%%%%%%%%%%%%%%%

\begin{frame}{Limitaciones de la lógica proposicional}
  Intentemos analizar el siguiente razonamiento:
  \jump
  \colored{
    \begin{lreasoning}
      \lpremise{Todos los perros son animales.}
      \lpremise{Firularis es un perro.}
      \lconclusion{Firulais es un animal.}
    \end{lreasoning}
  }
  \jump
  Suena lógico, ¿verdad? Pareciera que esto debería ser cierto, que el
  razonamiento y las premisas desde las cual partimos, son verdaderas.
\end{frame}

%%%%%%%%%%%%%%%%%%%%%%%%%%%%%%%%%%%%%%%%%%%%%%%%%%%%%

\begin{frame}{Limitaciones de la lógica proposicional}
  Intentemos formalizar el razonamiento. Cada oración es independiente, y no
  contiene conectivas. Cada una es una única proposición. Por lo tanto, tenemos:
  \jump
  \colored{
    p = Todos los perros son animales\\
    q = Firulais es un perro\\
    r = Firulais es un animal\\
  }
  \jump
  Donde $p$ y $q$ son nuestras premisas y $r$ sería la conclusión. Quedando:\\
  \colored{
    $p, q \lseq r$
  }
  \jump
  Como ya vimos, como la conclusión es independiente de las premisas, esta puede
  ser \fulltrue o \fullfalse, y por tanto, el razonamiento es inválido.
\end{frame}

%%%%%%%%%%%%%%%%%%%%%%%%%%%%%%%%%%%%%%%%%%%%%%%%%%%%%

\begin{frame}{Limitaciones de la lógica proposicional}
  ¿Cuál es el problema en nuestro razonamiento? ¿Será que el razonamiento es
  realmente inválido y que nuestra intuición nos miente?
  \jump
  La problemática radica en que las proposiciones hablan de una condición que
  se da en el universo, pero no dicen nada acerca de la estructura interna de
  esa condición.
  \jump
  En este caso, parte de las cosas que dice $p$ se usa en $q$ y otra parte en $r$,
  el mismo individuo (Firulais) es mencionado tanto en $q$ como en $r$, y sin
  embargo la lógica proposicional no dice nada acerca de estos hechos.
\end{frame}

%%%%%%%%%%%%%%%%%%%%%%%%%%%%%%%%%%%%%%%%%%%%%%%%%%%%%

\begin{frame}{Lógica de predicados al rescate}
  Por eso, necesitamos un nuevo tipo de lógica, la \bolder{lógica de predicados},
  o también llamada \bolder{lógica de orden uno} o \bolder{lógica de primer orden}.
  \jump
  \bolder{La lógica de primer orden nos va a permitir formalizar oraciones que hablan
  sobre individuos, sobre las propiedades de esos individuos, y sobre como esos
  individuos se relacionan entre si.}
  \jump
  La lógica de orden uno engloba a la lógica de orden cero. Es decir, que todo
  lo que se puede formalizar con lógica proposicional se puede formalizar en
  lógica de predicados, pero no al revés.
\end{frame}

%%%%%%%%%%%%%%%%%%%%%%%%%%%%%%%%%%%%%%%%%%%%%%%%%%%%%
%%%%%%%%%%%%%%%%%%%%%%%%%%%%%%%%%%%%%%%%%%%%%%%%%%%%%

\subsection{Individuos y Propiedades}

\begin{frame}{Individuos}
  La lógica de predicados va a hablar sobre \bolder{individuos}.
  \jump
  Un individuo es un elemento único e irrepetible del universo. Por ejemplo,
  una persona (Juan, Luis, María, etc.) un animal (Firulais, Michifus, etc.),
  o un elemento más abstracto (un número, un color, etc.)
  \jump
  Lo importante para que algo sea un individuo es que sea identificable de
  forma inequívoca. Por ejemplo, si hablamos de ``Juan'', solo hay un ``Juan'',
  podemos saber quien es y señalarlo con el dedo. Si hubiera dos personas con
  el nombre ``Juan'' deberíamos desambiguar para saber claramente sobre quien
  estamos hablando.
\end{frame}

%%%%%%%%%%%%%%%%%%%%%%%%%%%%%%%%%%%%%%%%%%%%%%%%%%%%%

\begin{frame}{Universo de discurso}
  Como llamamos a nuestros individuos va a depender del contexto. Es decir,
  depende de lo que estemos hablando. Por ejemplo, si estamos hablando de un
  grupo de amigos, ``Juan'' será un individuo puntual al cual podremos
  identificar.
  \jump
  Sin embargo, si estamos hablando de todos los alumnos de la universidad,
  deberemos ser mucho más específicos acerca de qué ``Juan'' se trata.
  \jump
  Pero ``Juan'' es solo el nombre, y uno no es solo su nombre, uno es lo que es.
  Lo que importa no es el nombre sino el individuo en si.
\end{frame}

%%%%%%%%%%%%%%%%%%%%%%%%%%%%%%%%%%%%%%%%%%%%%%%%%%%%%

\begin{frame}{Mismo individuo, muchos nombres}
  Los números por ejemplo, son elementos que son únicos e irrepetibles. \colored{
  Cinco es cinco}, siempre.
  \jump
  Sin embargo, podemos decir \bolder{``cinco''}, pero si habláramos en otro
  idioma diríamos por ejemplo \bolder{``cinq''}, \bolder{``cinque''},
  \bolder{``five''}, \bolder{``fem''}, etc.
  \jump
  Incluso en nuestro lenguaje cotidiano, sin hablar otro idioma, tenemos símbolos
  que representan al mismo individuo como ``5'', o ``V'' (en números romanos).
  \jump
  Más aún, si escribimos ``3+2'' o ``4+1'', ¿Qué representan?. Podríamos decir
  que es otra forma de escribir cinco.
  \jump
  Lo que importa no es como lo escribimos, sino lo que estamos queriendo
  representar, el individuo en si. Suena filosófico, pero es así...al fin de
  cuentas, la lógica nace en la filosofía.
\end{frame}

%%%%%%%%%%%%%%%%%%%%%%%%%%%%%%%%%%%%%%%%%%%%%%%%%%%%%

\begin{frame}{Propiedades}
  Nos va a interesar hablar de ciertas cosas de esos individuos. Por ejemplo,
  vamos a querer decir cosas como \colored{``Juan es grande''},
  \colored{``Firulais es un perro''}, o \colored{``Cinco es un número primo''}.
  \jump
  Vamos a decir entonces que los individuos tienen \bolder{propiedades}.
  \jump
  \bolder{Una propiedad es una cualidad o atributo que puede o no aplicarse a un
  individuo.}  
\end{frame}

%%%%%%%%%%%%%%%%%%%%%%%%%%%%%%%%%%%%%%%%%%%%%%%%%%%%%

\begin{frame}{Propiedades - Cont}
  La idea es que vamos a pensar en una propiedad como algo que, dado un individuo,
  la misma puede o no aplicar a ese individuo. Si aplicamos la propiedad
  al individuo, la podemos tratar como una proposición. Es decir, es
  algo de lo que vamos a decir que es \fulltrue o \fullfalse.
  \jump
  Por ejemplo, si la propiedad es \colored{``Ser grande''}, vamos a
  \bolder{aplicar} a \colored{``Juan''} esa propiedad para obtener
  \colored{``Juan es grande''}. Esto como vemos es una proposición tradicional,
  es algo de lo que podemos decir que es \fulltrue o \fullfalse.
  \jump
  Si en cambio aplicamos a \colored{``Luis''} la misma propiedad, obtenemos la
  proposición \colored{``Luis es grande''} la cual tiene también un valor de
  verdad, no necesariamente igual a aplicar la propiedad a Juan.
\end{frame}

%%%%%%%%%%%%%%%%%%%%%%%%%%%%%%%%%%%%%%%%%%%%%%%%%%%%%

\begin{frame}{Propiedades Ejemplo}
  Con la lógica de predicados vamos a intentar describir el universo. Mejor dicho,
  vamos a describir una parte puntual y significativa del universo.
  \jump
  Supongamos entonces que queremos hablar de una escuela a la que solo asisten
  cuatro alumnos, dos chicos (Juan y Luis), y dos chicas (María y Ana).
  \jump
  Juan, Luis, María y Ana van a ser los individuos sobre los que vamos a hablar.
  Una posible propiedad sobre ellos será \colored{``Es hombre''}.
\end{frame}

%%%%%%%%%%%%%%%%%%%%%%%%%%%%%%%%%%%%%%%%%%%%%%%%%%%%%

\begin{frame}{Propiedades Ejemplo - Cont}
  Podemos representar las propiedades como una tabla de doble entrada, con los
  individuos como filas y las propiedades como columnas.
  \jump
  \centerline{
  \begin{tabular}{| r | c |}
    \hline
    & Es hombre \\
    \hline
    Juan & \\
    \hline
    Luis & \\
    \hline
    María & \\
    \hline
    Ana & \\
    \hline
  \end{tabular}}
  \jump
  La tabla podremos completarla con \fulltrue o \fullfalse, dependiendo del
  valor que toma la propiedad de la columna aplicada al individuo en esa fila.
\end{frame}

%%%%%%%%%%%%%%%%%%%%%%%%%%%%%%%%%%%%%%%%%%%%%%%%%%%%%

\begin{frame}{Propiedades Ejemplo - Cont}
  \centerline{
  \begin{tabular}{| r | c |}
    \hline
    & Es hombre \\
    \hline
    Juan  & \true \\
    \hline
    Luis  & \true \\
    \hline
    María & \false \\
    \hline
    Ana   & \false \\
    \hline
  \end{tabular}}
  \jump
  Vemos entonces como la propiedad ``Es hombre'' se aplica a ``Juan'', quedando
  \colored{``Juan es hombre''} y teniendo esa proposición el valor de verdad
  \fulltrue. Lo mismo sucede en el caso de Luis. Por otro lado, cuando aplicamos
  la propiedad a ``María'' y a ``Ana'', la proposición tiene valor \fullfalse.
\end{frame}

%%%%%%%%%%%%%%%%%%%%%%%%%%%%%%%%%%%%%%%%%%%%%%%%%%%%%

\begin{frame}{Propiedades que dependen de otras propiedades}
  Agreguemos ahora una segunda propiedad a nuestro ejemplo, la propiedad
  \colored{``Es mujer''}.
  \jump
  Sabemos (intuitivamente) que la propiedad ``Es mujer'' se aplica a ``Juan'',
  quedando \colored{``Juan es mujer''} y teniendo esa proposición el valor de
  verdad \fullfalse. Lo mismo ocurre con Luis.
  \jump
  También sabemos que para los casos de ``María'' y de ``Ana'', la propiedad
  aplicada a ellas tiene valor \fulltrue.
  \jump
  Veamos ambas propiedades en la tabla.
  \centerline{
  \begin{tabular}{| r | c | c |}
    \hline
    & Es hombre & Es mujer \\
    \hline
    Juan  & \true  & \false \\
    \hline
    Luis  & \true  & \false \\
    \hline
    María & \false & \true \\
    \hline
    Ana   & \false & \true \\
    \hline
  \end{tabular}}
\end{frame}

%%%%%%%%%%%%%%%%%%%%%%%%%%%%%%%%%%%%%%%%%%%%%%%%%%%%%

\begin{frame}{Propiedades que dependen de otras propiedades}
  Si bien podemos tratar a ambas propiedades como independientes,
  hay una clara relación entre \colored{``Es hombre''} y \colored{``Es mujer''}.
  \jump
  En particular, si alguien es hombre, entonces seguro no es mujer. Y si es mujer,
  entonces seguro no es hombre. Es decir, ambas son complementarias.
  \jump
  Ya conocemos una conectiva que representa el concepto de complemento, la
  negación. Podemos reformular la propiedad \colored{``Es mujer''} en términos
  de \colored{``Es hombre''}.
  \jump
  \colored{Es mujer = $\lnot$ Es hombre}
  \jump
  Así, vemos como podemos tener propiedades que son dependientes de
  otras propiedades.
\end{frame}

%%%%%%%%%%%%%%%%%%%%%%%%%%%%%%%%%%%%%%%%%%%%%%%%%%%%%

\begin{frame}{Propiedades que dependen de otras propiedades}
  Asumamos ahora que conocemos dos nuevas propiedades que aplican en nuestro ejemplo,
  \colored{``Usa zapatos''} y \colored{``Usa corbata''}.
  \jump
  Podemos crear nuevas propiedades a partir de las que conocemos. Podemos decir
  por ejemplo, que si un hombre usa zapatos y corbata, entonces es elegante.
  \jump
  \colored{Es elegante = Es hombre $\land$ Usa zapatos $\land$ Usa corbata}
\end{frame}

%%%%%%%%%%%%%%%%%%%%%%%%%%%%%%%%%%%%%%%%%%%%%%%%%%%%%

\begin{frame}{Propiedades que dependen de otras propiedades}
  ¿Qué significa entonces aplicar la propiedad \colored{``Es elegante''} a
  \colored{``Juan''}?
  \jump
  Significa que Juan debe cumplir con ser hombre, con usar zapatos y con usar
  corbata. Es decir que debemos aplicar a Juan cada una de las propiedades que
  componen a ser elegante, y que en todos los casos el resultado debe ser \fulltrue.
  \jump
  \colored{Juan es elegante = Juan es hombre $\land$ Juan usa zapatos $\land$ Juan usa corbata}
\end{frame}

%%%%%%%%%%%%%%%%%%%%%%%%%%%%%%%%%%%%%%%%%%%%%%%%%%%%%
%%%%%%%%%%%%%%%%%%%%%%%%%%%%%%%%%%%%%%%%%%%%%%%%%%%%%

\subsection{Relaciones simples}
\frame{\tableofcontents[currentsection,currentsubsection]}

%%%%%%%%%%%%%%%%%%%%%%%%%%%%%%%%%%%%%%%%%%%%%%%%%%%%%

\begin{frame}{¿Propiedades que incluyen individuos?}
  Pensemos ahora en una frase un poco más compleja, y tratemos de formalizarla:
  \colored{``Juan está enamorado de María''}.
  \jump
  Sencillo, podríamos tener una propiedad que sea
  \colored{``Estar enamorado de María''}, y nos basta con aplicar la propiedad
  a Juan para obtener la oración anterior.
  \jump
  Pero ¿Qué pasa si ahora queremos escribir  \colored{``Ana está enamorada de Luis''}?
  \jump
  Necesitamos otra propiedad \colored{``Estar enamorada de Luis''} y aplicar la
  misma a Ana.
  \jump
  Sin embargo, vemos que hay una relación entre ambas frases, ambas hablan de
  lo mismo, de estar enamorado de alguien más.
\end{frame}

%%%%%%%%%%%%%%%%%%%%%%%%%%%%%%%%%%%%%%%%%%%%%%%%%%%%%

\begin{frame}{Relaciones}
  Cuando vemos que en nuestra propiedad estamos mencionando a un individuo (Por ejemplo
  a María en ``Estar enamorado de María''), lo que queremos en realidad es expresar
  una relación entre dos individuos (en este caso, Juan a quien será aplicada la
  propiedad y María, que forma parte de la propiedad). Lo que buscamos es una
  \bolder{relación}.
  \jump
  Una relación es similar a una propiedad, en el sentido que se aplica sobre
  individuos, pero en lugar de aplicar sobre un solo individuo, aplica sobre dos.
  \jump
  Así, nuestra relación será \colored{``Está enamorado de''}, y si lo aplicamos
  a \colored{``Juan''} y a \colored{``María''} obtenemos la frase
  \colored{``Juan está enamorado de María''}.
  \jump
  \bolder{Una relación vincula dos individuos a través de una caracteristica.}
\end{frame}

%%%%%%%%%%%%%%%%%%%%%%%%%%%%%%%%%%%%%%%%%%%%%%%%%%%%%

\begin{frame}{Dirección de las relaciones}
  El hecho de que Juan esté enamorado de María no hace que María necesariamente
  le corresponda. Por lo que las relaciones son dirigidas.
  \jump
  Es decir, aplicar \colored{``Está enamorado de''} a \colored{``Juan''} y luego
  a \colored{``María''} representa \colored{``Juan está enamorado de María''}.
  Pero aplicar  \colored{``Está enamorado de''} primero a \colored{``María''} y
  en segundo lugar a \colored{``Juan''} nos dará \colored{``María está enamorada de Juan''}.
  \jump
  El valor de verdad de ambas proposiciones formadas no es necesariamente idéntico.
  \jump
  \bolder{El orden en el que aplicamos la relación a los distintos individuos da
  como resultado diferentes proposiciones con diferente valor de verdad.}
\end{frame}

%%%%%%%%%%%%%%%%%%%%%%%%%%%%%%%%%%%%%%%%%%%%%%%%%%%%%

\begin{frame}{Relaciones complejas}
  De forma similar a lo que ocurre con las propiedades, las relaciones pueden
  estar dadas en términos de otras relaciones.
  \jump
  Por ejemplo, podríamos decir que estar enamorado requiere de dos cosas,
  atracción física y atracción intelectual. Así:
  \jump
  \colored{Está enamorado de = Siente atracción fisica $\land$ Siente atracción intelectual}
\end{frame}

%%%%%%%%%%%%%%%%%%%%%%%%%%%%%%%%%%%%%%%%%%%%%%%%%%%%%

\begin{frame}{Relaciones complejas - Ejemplo}
  La frase \colored{Siente atracción física} nos resulta un poco obvia en su aplicación.
  \jump
  Por ejemplo, apliquemos \colored{``Está enamorado de''} a ``Juan'' y a ``María''.
  \jump
  Cómo sabemos que la equivalencia es:
  \jump
  \colored{Juan siente atracción fisica por María $\land$ Juan siente atracción intelectual por María}
\end{frame}

%%%%%%%%%%%%%%%%%%%%%%%%%%%%%%%%%%%%%%%%%%%%%%%%%%%%%

\begin{frame}{Relaciones con orden en la aplicación}
  Pensemos ahora en la relación \colored{``es amado por''}. Podríamos definir
  ``es amado por'' en términos de ``está enamorado de''. Por ejemplo:
  \colored{María es amada por Juan = Juan está enamorado de María}.
  \jump
  Pero si definimos:
  \jump
  \colored{Es amado por = Esta enamorado de}
  \jump
  No resulta evidente que el orden de mis individuos cambia en la equivalencia.
\end{frame}

%%%%%%%%%%%%%%%%%%%%%%%%%%%%%%%%%%%%%%%%%%%%%%%%%%%%%

\begin{frame}{Parámetros}
  Para solucionar el problema anterior, tenemos que introducir un nuevo elemento.
  Los \bolder{parámetros}.
  \jump
  Un \bolder{parámetro} no es más que un nombre que va a representar a un individuo,
  pero que al momento de la definición, no sabemos quien es.
  \jump
  Vamos a usar los parámetros en nuestras propiedades y relaciones. Así por ejemplo
  en lugar de decir que la propiedad es \colored{``Es hombre''} vamos a decir
  \colored{``$x$ es hombre''}.
  \jump
  ¿Quién es $x$? La respuesta es $x$ va a ser reemplazado en el texto por un individuo
  cuando apliquemos la propiedad al mismo. Por ejemplo, si aplicamos ahora
  \colored{``$x$ es hombre''} a \colored{``Juan''}, lo que hacemos es reemplazar
  la $x$ por ``Juan'' y quedarnos con la oración formada \colored{``Juan es hombre''}.
\end{frame}

%%%%%%%%%%%%%%%%%%%%%%%%%%%%%%%%%%%%%%%%%%%%%%%%%%%%%

\begin{frame}{Parámetros - Cont}
  Esto aplica también a las relaciones, ahora que tenemos más de un individuo,
  necesitamos más de un parámetro.
  \jump
  Volvamos a nuestra relación \colored{``Es amado por''}. Si lo
  replanteamos usando parámetros podemos escribir
  \colored{``$x$ es amado por $y$''}. Si aplicamos entonces ahora a
  ``Juan'' y ``María'', obtenemos  \colored{``Juan es amado por María''}.
  \jump
  Ahora, si miramos la definición de es amado por, podemos dejar más claro que
  el orden de los individuos se invierte.
  \jump
  \colored{$x$ es amado por $y$ = $y$ está enamorado de $x$}
  \jump
  Luego veremos que llamarlos $x$ e $y$ es arbitrario, pero de momento vamos a
  decir que $x$ representa al individuo que aplicamos la relación primero e $y$
  al segundo.
\end{frame}

%%%%%%%%%%%%%%%%%%%%%%%%%%%%%%%%%%%%%%%%%%%%%%%%%%%%%

\begin{frame}{Parámetros - Cont}
  No solo es el caso en donde se invierte el orden de los parámetros en
  donde sirve, sino donde el individuo se repite en la frase equivalente.
  \jump
  Pensemos en la relación \colored{``Se casará con''}. Una persona se
  va a casar con otra si están enamorados mutuamente. Tratemos de plantearlo
  en términos de nuestra otra relación:
  \jump
  \colored{Se casará con = Está enamorado de $\land$ Está enamorado de}
  \jump
  El segundo ``Está enamorado de'' debería indicar de alguna forma que ya no es
  el primer individuo del segundo, sino el segundo del primero. Sin embargo, lo
  que escribimos no dice nada de eso.
\end{frame}

%%%%%%%%%%%%%%%%%%%%%%%%%%%%%%%%%%%%%%%%%%%%%%%%%%%%%

\begin{frame}{Parámetros - Cont}
  Replanteemos la relación utilizando parámetros. Es decir, pensemos ahora
  que significa \colored{``$x$ se casará con $y$''}.
  \jump
  \colored{$x$ se casará con $y$ = $x$ está enamorado de $y$ $\land$ $y$ está enamorado de $x$}
  \jump
  Queda de esta forma claramente expresado cuales son las relaciones que se deben
  cumplir.
\end{frame}

%%%%%%%%%%%%%%%%%%%%%%%%%%%%%%%%%%%%%%%%%%%%%%%%%%%%%

\begin{frame}{Representando relaciones}
  Podemos representar una relación de forma similar a las propiedades, es decir,
  mediante una tabla de doble entrada.
  \jump
  A diferencia de la tabla sencilla de las propiedades, en la cual podíamos
  representar más de una propiedad en la misma tabla, aquí toda la tabla representa
  una sola relación. Las filas representarán el individuo al que aplicamos primero
  la relación ($x$) y las columnas al que aplicamos segundo ($y$).
\end{frame}

%%%%%%%%%%%%%%%%%%%%%%%%%%%%%%%%%%%%%%%%%%%%%%%%%%%%%

\begin{frame}{Representando relaciones}
  Veamos la tabla de nuestro ejemplo:
  \jump
  \begin{tabular}{| r | c | c | c | c |}
    \hline
    $x$ está enamorado de $y$ & Juan   & Luis   & María  & Ana \\
    \hline
    Juan  & \false & \false & \true  & \false \\ 
    \hline
    Luis  & \false  & \true & \false & \false \\ 
    \hline
    María & \true  & \true  & \false & \false \\ 
    \hline
    Ana   & \false & \true & \false  & \false \\ 
    \hline
  \end{tabular}
  \jump
  En la tabla se ve como Juan está enamorado de María, y esta está enamorada de
  él, por lo que van a casarse. Ana está enamorada de Luis, pero Luis solo
  está enamorado de si mismo.
\end{frame}

%%%%%%%%%%%%%%%%%%%%%%%%%%%%%%%%%%%%%%%%%%%%%%%%%%%%%

\begin{frame}{Relaciones múltiples}
  Podemos tener relaciones que tienen más de dos individuos como elementos.
  \jump
  Por ejemplo la relación \colored{``$x$ conoce a $y$ gracias a $z$''}. Podríamos
  decir entonces \colored{``Ana conoce a Luis gracias a María''}.
  \jump
  Para obtener esa proposición basta aplicar la relación primero a Ana, luego a
  Luis y por último a María.
  \jump
  Este tipo de relaciones ya no se pueden interpretar utilizando una tabla, por
  lo que hay que aplicar nuestra mejor capacidad mental y de abstracción para
  comprender plenamente estas relaciones.
\end{frame}

%%%%%%%%%%%%%%%%%%%%%%%%%%%%%%%%%%%%%%%%%%%%%%%%%%%%%
%%%%%%%%%%%%%%%%%%%%%%%%%%%%%%%%%%%%%%%%%%%%%%%%%%%%%

\subsection{Todos}
\toc[currentsection,currentsubsection]

%%%%%%%%%%%%%%%%%%%%%%%%%%%%%%%%%%%%%%%%%%%%%%%%%%%%%

\begin{frame}{Todos}
  Volvamos a nuestro ejemplo de la escuela y sus chicos. ¿Qué significaría la
  frase \colored{``Todos son inteligentes''}?
  \jump
  Pensémoslo en términos de nuestra tabla de propiedades:
  \jump
  \centerline{\begin{tabular}{| r | c |}
    \hline
    & $x$ es inteligente\\
    \hline
    Juan & \\
    \hline
    Luis & \\
    \hline
    María & \\
    \hline
    Ana & \\
    \hline
  \end{tabular}}
  \jump
  ¿Cómo deberíamos completar los espacios en blanco? 
\end{frame}

%%%%%%%%%%%%%%%%%%%%%%%%%%%%%%%%%%%%%%%%%%%%%%%%%%%%%

\begin{frame}{Todos - Cont}
  Podemos pensar la frase \colored{``Todos son inteligentes''} como una propiedad
  que depende de otras propiedades. En particular:
  \jump
  \colored{Todos son inteligentes = Juan es inteligente $\land$ Luis es
    inteligente $\land$ María es inteligente $\land$ Ana es inteligente
  }
  \jump
  Es decir, la idea de que todos son inteligentes nos dice que cada uno de los
  chicos es inteligente.
  \jump
  Precisamente, \bolder{La palabra ``Todos'' nos indica que la propiedad
  aplica a cada uno de los individuos de nuestro universo}
\end{frame}

%%%%%%%%%%%%%%%%%%%%%%%%%%%%%%%%%%%%%%%%%%%%%%%%%%%%%

\begin{frame}{Todos - Cont}
  Si vamos al caso del ejemplo anterior, nuestra tabla para interpretar el
  universo quedaría así:
  \jump
  \centerline{\begin{tabular}{| r | c |}
    \hline
    & $x$ es inteligente\\
    \hline
    Juan  & \fulltrue \\
    \hline
    Luis  & \fulltrue \\
    \hline
    María & \fulltrue \\
    \hline
    Ana   & \fulltrue \\
    \hline
  \end{tabular}}
\end{frame}

%%%%%%%%%%%%%%%%%%%%%%%%%%%%%%%%%%%%%%%%%%%%%%%%%%%%%

\begin{frame}{Nadie}
  ¿Y qué pasa si decimos \colored{``Nadie se porta mal''}?
  \jump
  Nadie es lo mismo que decir que, ni Juan, ni Luis, ni María, ni Ana se portan
  mal. Podemos pensarlo como una dependencia de otras propiedades.
  \jump
  \colored{Nadie se porta mal = ($\lnot$ Juan se porta mal) $\land$
    ($\lnot$ Luis se porta mal) $\land$ ($\lnot$ María se porta mal) $\land$
    ($\lnot$ Ana se porta mal)
  }
  \jump
  Como ya sabemos, La negación implica que la propiedad sea falsa. Es decir que,
  en nuestra tabla, esto estaría representado de la siguiente forma:
  \jump
  \centerline{\begin{tabular}{| r | c | c |}
    \hline
    & $x$ es inteligente & $x$ se porta mal \\
    \hline
    Juan  & \fulltrue & \fullfalse \\
    \hline
    Luis  & \fulltrue & \fullfalse \\
    \hline
    María & \fulltrue & \fullfalse \\
    \hline
    Ana   & \fulltrue & \fullfalse \\
    \hline
  \end{tabular}}
\end{frame}

%%%%%%%%%%%%%%%%%%%%%%%%%%%%%%%%%%%%%%%%%%%%%%%%%%%%%

\begin{frame}{Alguien}
  Hay otro tipo de oraciones que nos brindan información lógica acerca del
  universo. Pensemos en el siguiente ejemplo \colored{``Alguien juega a la pelota''}.
  \jump
  Podemos pensar en términos de nuestra tabla.
  \centerline{\begin{tabular}{| r | c |}
    \hline
    & $x$ juega a la pelota \\
    \hline
    Juan  &  \\
    \hline
    Luis  &  \\
    \hline
    María &  \\
    \hline
    Ana   &  \\
    \hline
  \end{tabular}}
  \jump
  ¿Cómo completamos la tabla?
\end{frame}

%%%%%%%%%%%%%%%%%%%%%%%%%%%%%%%%%%%%%%%%%%%%%%%%%%%%%

\begin{frame}{Alguien}
  La realidad es que la frase no nos dice mucho. Solo nos
  dice que hay alguien que juega a la pelota, pero bien podría ser que haya más
  de una persona que lo haga.
  \jump
  Es decir solo sabemos que alguno de los casos es \fulltrue, pero el resto
  podrían ser \fullfalse o \fulltrue.
  \jump
  Es decir, hablar de alguno, es hablar de disyunciones entre los individuos.
  \jump
  \colored{Alguien juega a la pelota = Juan juega a la pelota $\lor$ Luis juega
    a la pelota $\lor$ María juega a la pelota $\lor$ Ana juega a la pelota}
  \jump
  Para que la disyunción sea verdadera, alguno de los términos de la disyunción
  debe ser verdadera, pero no sabemos cuál. Si combinamos esa información con
  otra, tal vez podamos deducir quien sea.
\end{frame}

%%%%%%%%%%%%%%%%%%%%%%%%%%%%%%%%%%%%%%%%%%%%%%%%%%%%%

\begin{frame}{Todos los...}
  Otro tipo de oración que solemos utilizar cuando hablamos, involucran hablar
  de un subgrupo de elementos.
  \jump
  Por ejemplo la oración \colored{``Todos los hombres se escaparon de la escuela''}
  \jump
  ¿Qué significa esto en nuestro ejemplo? Juan y Luis, ambos son hombres, y
  por tanto se escaparon.
  \jump
  ¿Qué pasa con las chicas? La respuesta es, no sabemos. La frase no dice nada
  acerca de si María o Ana se escaparon o no de la escuela.
\end{frame}

%%%%%%%%%%%%%%%%%%%%%%%%%%%%%%%%%%%%%%%%%%%%%%%%%%%%%

\begin{frame}{Todos los...}
  En este caso hay efectivamente una relación entre la propiedad de ser hombre
  y la propiedad de escaparse de la escuela. Podemos ponerlo de esta forma:
  \jump
  \colored{Todos los hombres se escaparon de la escuela =\\
    \quad$x$ es hombre $\lthen$ $x$ se escapó de la escuela}
  \jump
  ¿Quién es $x$ en este caso? El individuo puntual al que aplicamos nuestra
  propiedad, pero, en este caso, debemos aplicarlo a cada uno de los individuos
  de nuestro universo, pues tenemos la palabra ``Todos'' adelante.
  \jump
  Apliquemos a \colored{``Juan''} nuestra propiedad. Como
  \colored{``Juan es hombre''} es \fulltrue, para que la implicación sea
  verdadera entonces \colored{``Juan se escapó de la escuela''} debe
  necesariamente ser \fulltrue.
  \jump
  Apliquemos ahora a \colored{``María''}. Como \colored{``María es hombre''} es
  \fullfalse, la implicación ya es verdadera, independientemente del valor de
  \colored{``María se escapó de la escuela''}.
\end{frame}

%%%%%%%%%%%%%%%%%%%%%%%%%%%%%%%%%%%%%%%%%%%%%%%%%%%%%

\begin{frame}{Todos los... - Cont}
  Es decir, sabiendo que \colored{``Todos los hombres se escaparon de la escuela''},
  podemos completar la tabla de la siguiente forma.
  \jump
  \centerline{\begin{tabular}{| r | c | c |}
    \hline
    & $x$ es hombre & $x$ se escapó de la escuela \\
    \hline
    Juan  & \fulltrue  & \fulltrue \\
    \hline
    Luis  & \fulltrue  & \fulltrue \\
    \hline
    María & \fullfalse & ? \\
    \hline
    Ana   & \fullfalse & ? \\
    \hline
  \end{tabular}}
  \jump
  En los lugares en donde hay un signo de pregunta, no podemos completar nada,
  pues la frase no contiene información sobre esos casos.
\end{frame}

%%%%%%%%%%%%%%%%%%%%%%%%%%%%%%%%%%%%%%%%%%%%%%%%%%%%%

\begin{frame}{Algún...}
  De forma similar a ``todos los ...'' podemos decir frases con la forma
  \colored{``Alguna mujer se escapó de la escuela''}.
  \jump
  Es decir, que puede haber dos casos, o bien María se escapó de la escuela o bien
  Ana se escapó de la escuela. Podríamos decir que:
  \jump
  \colored{Alguna mujer se escapó de la escuela =
    (María se escapó de la escuela) $\lor$ (Ana se escapó de la escuela)}
  \jump
  Pero ¿Por qué agregamos solo a María y Ana a nuestra fórmula equivalente?
  Sencillamente porque son las únicas mujeres. Pero cuando decíamos ``alguno'',
  vimos que debíamos hacer una disyunción con todos los elementos del universo.
  Debemos entonces buscar una formula equivalente, en donde los casos de aquellos
  que son hombres, seguro den \fullfalse.
\end{frame}

%%%%%%%%%%%%%%%%%%%%%%%%%%%%%%%%%%%%%%%%%%%%%%%%%%%%%

\begin{frame}{Algún...}
  Así, la idea es que necesitamos realizar una conjunción de disyunciones.
  \jump
  \colored{Alguna mujer se escapó de la escuela =
    (Juan es mujer $\land$ Juan se escapó de la escuela) $\lor$\\ 
    (Luis es mujer $\land$ Luis se escapó de la escuela) $\lor$\\ 
    (María es mujer $\land$ María se escapó de la escuela) $\lor$\\ 
    (Ana es mujer $\land$ Ana se escapó de la escuela)
  }
  \jump
  Si analizamos esa fórmula, en el caso de Juan y Luis, la conjunción dará \fullfalse
  pues no es cierto que sean mujeres. Mientras que en el caso de María
  dará \fulltrue solo si María se escapó de la escuela. Lo mismo ocurre para Ana.
\end{frame}

%%%%%%%%%%%%%%%%%%%%%%%%%%%%%%%%%%%%%%%%%%%%%%%%%%%%%

\section{Formalización de la lógica de predicados}
\toc[currentsection,currentsubsection]

%%%%%%%%%%%%%%%%%%%%%%%%%%%%%%%%%%%%%%%%%%%%%%%%%%%%%

\begin{frame}{Formalización}
  Hasta ahora hemos solamente trabajado con la lógica de predicados de forma
  natural, intentando interpretar de forma intuitiva las oraciones que nos encontramos.
  \jump
  Comencemos entonces a formalizar los conceptos que hemos visto hasta ahora.
\end{frame}

%%%%%%%%%%%%%%%%%%%%%%%%%%%%%%%%%%%%%%%%%%%%%%%%%%%%%
%%%%%%%%%%%%%%%%%%%%%%%%%%%%%%%%%%%%%%%%%%%%%%%%%%%%%

\subsection{Representando individuos}
\toc[currentsection,currentsubsection]

%%%%%%%%%%%%%%%%%%%%%%%%%%%%%%%%%%%%%%%%%%%%%%%%%%%%%

\begin{frame}{Constantes}
  Las \bolder{constantes} representan a los individuos de nuestro universo.
  \jump
  En la bibliografía se denotan en general con letras minúsculas, donde la letra
  suele estar relacionada con el individuo al que representa (Ej. \colored{j = Juan},
  \colored{m = María})
  \jump
  Vamos a tener una constante para cada individuo del universo que podamos (o
  mejor dicho, que necesitemos) nombrar.
\end{frame}

%%%%%%%%%%%%%%%%%%%%%%%%%%%%%%%%%%%%%%%%%%%%%%%%%%%%%

\begin{frame}{Funciones}
  Las \bolder{funciones} son una forma de denotar a un individuo sin hacer una
  referencia directa al mismo.
  \jump
  Una función debe ser aplicada a uno o más constantes (individuos) y expresará
  un y solo un individuo. Actúan como una función en matemática.
  \jump
  El ejemplo más fácil de entender sería la función \colored{sucesor}. Esta función,
  dado un número, devuelve ese número sumado en uno. Así por ejemplo:\\
  \colored{sucesor(0) = 1}\\
  \colored{sucesor(1) = 2}\\
  \colored{...}\\
  \colored{sucesor(n) = n+1}\\
  \jump
  Una función puede estar expresada en lenguaje formal, como en el caso de ``sucesor''
  que usa lenguaje matemático, o puede usar lenguaje coloquial.
\end{frame}

%%%%%%%%%%%%%%%%%%%%%%%%%%%%%%%%%%%%%%%%%%%%%%%%%%%%%

\begin{frame}{Funciones - Cont}
  Pero las funciones no son solo para números. Por ejemplo, la función
  \colored{vecino(x)}, que dado una persona, devuelve el vecino de esa persona.
  \jump
  Como definimos quien es vecino de quien es ambiguo, y no va a estar dado por
  un lenguaje formal. En todo caso, tenemos que tratar de definir que significa
  en un lenguaje no ambiguo. Por ejemplo, para cada persona del universo, decir
  exactamente quien es el vecino, armando una lista.
  \jump
  Las funciones se suelen representar en la bibliografía con una letra minúscula,
  de forma similar a una constante.
\end{frame}

%%%%%%%%%%%%%%%%%%%%%%%%%%%%%%%%%%%%%%%%%%%%%%%%%%%%%

\begin{frame}{Aridad}
  Las funciones toman parámetros. Es decir, esperan ser aplicadas a una cantidad
  específica de individuos. Esto se conoce como \bolder{aridad}
  \jump
  La \bolder{aridad} es la cantidad de parámetros que espera la función. Si una función
  espera un solo parámetro, se dice que tiene aridad 1, si espera dos parámetros,
  se dice que tiene aridad 2, etc.
  \jump
  Por ejemplo, la función \bolder{sucesor(x)} tiene aridad 1. La función
  \bolder{suma(x, y)} tiene aridad 2.
  \jump
  Al formalizarlo se vuelve sencillo expresar que nombre le ponemos al parámetro,
  es decir al individuo al que le aplicamos primero la función. En este caso,
  el primer individuo se llama $x$, y el segundo $y$. Pero si formalizamos la
  función como \bolder{suma(a, b)}, es la misma función, pero llamamos
  $a$ y $b$ a nuestros parámetros. Es simplemente el nombre que le ponemos al
  individuo para hablar de él cuando definimos que significa ``suma''.
\end{frame}

%%%%%%%%%%%%%%%%%%%%%%%%%%%%%%%%%%%%%%%%%%%%%%%%%%%%%

\begin{frame}{Aplicación}
  Cuando aplicamos una función, en lugar de colocar el nombre del parámetro
  colocamos el nombre del individuo al cual le aplicamos la misma (Es decir,
  nuestra constante).
  \jump
  Así por ejemplo, si lo que queremos es obtener el vecino de Juan, diremos
  \colored{``v(j)''},  donde ``j'' es la constante que representa a Juan,
  y \colored{``v(x)''} es la función que denota a un vecino de x.
  \jump
  Es decir, que hay que separar en dos partes. Por un lado, la definición de
  nuestros elementos (El \bolder{diccionario}). Por otro lado, la fórmula en si,
  es decir, lo que queremos expresar, utilizando esas definiciones.
\end{frame}

%%%%%%%%%%%%%%%%%%%%%%%%%%%%%%%%%%%%%%%%%%%%%%%%%%%%%

\begin{frame}{Aplicación - Cont}
  Ejemplo:
  \jump
  \colored{
    $j$ = Juan\\
    $v(x)$ = el vecino de $x$\\
    ~\\
    $v(j)$
  }
  \jump
  Las primeras dos líneas definen el diccionario, mientras que la última define
  lo que queremos expresar, en este caso ``el vecino de Juan''.
  \jump
  Note como la $x$ se reemplaza por la $j$ para pasar a ser ``el vecino de $j$'',
  primero, y como luego analizamos $j$ para interpretarlo como ``el vecino de Juan''.
\end{frame}

%%%%%%%%%%%%%%%%%%%%%%%%%%%%%%%%%%%%%%%%%%%%%%%%%%%%%

\begin{frame}{Resumen}
  Podemos representar los individuos de nuestro universo de dos formas:
  \begin{itemize}
    \item Mediante constantes
    \item Mediante funciones que se aplican a una o más constantes
  \end{itemize}
  \jump
  Denotamos a los individuos con letras minúsculas, ya sea una constante o una
  función.
  \jump
  Para ser más expresivos podemos usar nombres en minúscula, por ejemplo, ``juan''
  puede ser una constante, y ``vecino'' una función.
\end{frame}

%%%%%%%%%%%%%%%%%%%%%%%%%%%%%%%%%%%%%%%%%%%%%%%%%%%%%
%%%%%%%%%%%%%%%%%%%%%%%%%%%%%%%%%%%%%%%%%%%%%%%%%%%%%

\subsection{Predicados}
\toc[currentsection,currentsubsection]

%%%%%%%%%%%%%%%%%%%%%%%%%%%%%%%%%%%%%%%%%%%%%%%%%%%%%

\begin{frame}{Predicados}
  Hasta ahora vimos como representar individuos, pero no las propiedades y
  las relaciones sobre ellos.
  \jump
  Las propiedades y relaciones se representan mediante \bolder{predicados}.
  \jump
  \bolder{Un predicado es una función que se aplica a uno o más individuos y que al
  hacerlo expresa un valor de verdad, ya sea \fulltrue o \fullfalse.}
  \jump
  \bolder{Los predicados también tienen aridad. Un predicado de aridad uno se
  corresponde con una propiedad del individuo. Un predicado de aridad dos se
  corresponde a una relación entre individuos. Un predicado de mayor aridad se
  corresponde a relaciones entre múltiples individuos.}
\end{frame}

%%%%%%%%%%%%%%%%%%%%%%%%%%%%%%%%%%%%%%%%%%%%%%%%%%%%%

\begin{frame}{Predicados - Cont}
  Los predicados se suelen representar en la bibliografía con letras
  mayúsculas (P, Q, R), en general relacionadas con lo que representan.
  \jump
  Así por ejemplo la propiedad \colored{``$x$ es hombre''} se puede representar
  como \colored{``$H(x)$''}.
  \jump
  Para ganar expresividad podemos utilizar una palabra que comience con mayúscula,
  por ejemplo \colored{``Hombre(x)''}.
\end{frame}

%%%%%%%%%%%%%%%%%%%%%%%%%%%%%%%%%%%%%%%%%%%%%%%%%%%%%

\begin{frame}{Aplicación}
  Los predicados se aplican de la misma forma que las funciones. Es decir,
  reemplazamos el parámetro por la constante que representa al individuo al
  cual queremos aplicar el predicado.
  \jump
  Así podemos tener el siguiente ejemplo.\\
  \colored{
    $j$ = Juan\\
    $m$ = María\\
    $E(x, y)$ = $x$ está enamorado de $y$\\
    ~\\
    $E(j, m)$
  }
  \jump
  Esta expresión nos dice \colored{``Juan está enamorado de María''}.
\end{frame}

%%%%%%%%%%%%%%%%%%%%%%%%%%%%%%%%%%%%%%%%%%%%%%%%%%%%%

\begin{frame}{Aplicación - Cont}
  Note que \colored{``$E(j, m)$''} no es lo mismo que \colored{``$E(m, j)$''}.
  Mientras el primero indica que \colored{``Juan está enamorado de María''},
  el segundo nos dice que \colored{``María está enamorada de Juan''}.
  \jump
  El orden en el que ponemos nuestras constantes es relevante, y nos indica quien
  debería ser tomado por $x$ y quien por $y$.
\end{frame}

%%%%%%%%%%%%%%%%%%%%%%%%%%%%%%%%%%%%%%%%%%%%%%%%%%%%%

\begin{frame}{Uniendo predicados}
  Como un predicado aplicado representa un valor de verdad, podemos unir varios
  predicados usando conectivas lógicas. El valor de verdad final de toda la
  fórmula será el resultado de evaluar los predicados aplicados y luego
  usar las reglas vistas en la lógica proposicional para obtener el valor de
  aplicar las conectivas.
  \jump
  Por ejemplo en el siguiente ejemplo:\\
  \colored{$E(j, m) \land E(m, j)$}\\
  Estamos diciendo que Juan y María están enamorados uno del otro.
  \jump
  En el siguiente:\\
  \colored{$E(j, m) \land \lnot E(m, j)$}\\
  Estamos diciendo que Juan ama a María pero María no le corresponde.
\end{frame}

%%%%%%%%%%%%%%%%%%%%%%%%%%%%%%%%%%%%%%%%%%%%%%%%%%%%%

\subsection{Cuantificadores y Variables}
\frame{\tableofcontents[currentsection,currentsubsection]}

%%%%%%%%%%%%%%%%%%%%%%%%%%%%%%%%%%%%%%%%%%%%%%%%%%%%%

\begin{frame}{Cuantificadores}
  Lo último que nos falta ver es como representar oraciones que contienen las
  palabras ``Todos'', ``Alguno'' y ``Ninguno''.
  \jump
  Para representar esto surge la idea de \bolder{cuantificador}.
  \jump
  \bolder{Un cuantificador es una expresión que indica la cantidad de veces que
  un predicado es \fulltrue si se aplica el mismo a cada uno de los individuos
  del universo}.
  \jump
  Vamos a ver tres cuantificadores, el \bolder{cuantificador universarl}, el
  \bolder{cuantificador existencial} y el \bolder{cuantificador existencial negado}.
\end{frame}

%%%%%%%%%%%%%%%%%%%%%%%%%%%%%%%%%%%%%%%%%%%%%%%%%%%%%

\begin{frame}{Cuantificadores}
  Para entender mejor el concepto supongamos que en nuestro universo tenemos
  $n$ individuos (donde $n$ es un número cualquiera).
  \jump
  Tendremos entonces las siguientes constantes representando a dichos individuos.\\
  \colored{$c_1, c_2, c_3, ..., c_{n-1}, c_n$}\\
  Supongamos también que contamos con un predicado $P(x)$.
\end{frame}

%%%%%%%%%%%%%%%%%%%%%%%%%%%%%%%%%%%%%%%%%%%%%%%%%%%%%

\begin{frame}{Cuantificador Universal}
  El \bolder{Cuantificador Universal} se utiliza para afirmar que \bolder{todos}
  los individuos cumplen el predicado.
  \jump
  Es decir, que si aplicamos el predicado a cada uno de los individuos del universo,
  este debe dar \fulltrue en todos los casos.
  \jump
  El cuantificador universal se representa con el símbolo $\forall$.
  \jump
  Si queremos decir que ``Todos los individuos cumplen el predicado P'' escribimos:\\
  \colored{$\forall x . P(x)$}\\
  \jump
  En nuestro ejemplo, esto es equivalente a decir:\\
  \colored{$\forall x . P(x) = P(c_1) \land P(c_2) \land P(c_3) \land ... \land P(c_{n-1}) \land P(c_n)$}\\
\end{frame}

%%%%%%%%%%%%%%%%%%%%%%%%%%%%%%%%%%%%%%%%%%%%%%%%%%%%%

\begin{frame}{Cuantificador Universal}
  Los cuantificadores no solo aplican a un predicado, sino a toda la fórmula que
  viene luego del cuantificador.
  \jump
  Así por ejemplo \colored{``$\forall x. P(x) \land \lnot Q(x)$''} es equivalente
  a decir que todos los individuos cumplen \colored{``$P(x) \land \lnot Q(x)$''}.
  \jump
  Así, se cumple lo siguiente:\\
  \footnotesize
  \colored{$\forall x . P(x) \land \lnot Q(x) = (P(c_1) \land \lnot Q(c_1)) \land
  (P(c_2) \land \lnot Q(c_2)) \land ... \land
  \land (P(c_n) \land \lnot Q(c_n))$}
\end{frame}

%%%%%%%%%%%%%%%%%%%%%%%%%%%%%%%%%%%%%%%%%%%%%%%%%%%%%

\begin{frame}{Variables}
  Pero, ¿Qué significa $\forall x$?, o mejor dicho, ¿De donde sale esa $x$?
  \jump
  La $x$ es una \bolder{variable}. Una \bolder{variable} representa, de forma
  similar a un parámetro, a un individuo. Pero ¿A qué individuo? La respuesta es,
  a todos.
  \jump
  La idea es, el cuantificador dice, \colored{``Para todo individuo del universo,
  llamémoslo $x$, ese $x$ cumple el siguiente predicado''}.
  \jump
  ¿Por qué es importante ponerle un nombre al individuo? Porque podemos cuantificar
  más de un individuo. Por ejemplo, \colored{``Para todo par de numero, llamemos
  al primero $x$ y al segundo $y$, se cumple ...''}.
  \jump
  Eso podemos formalizarlo como: \colored{$\forall x. \forall y. ...$}. Otra
  forma de representar lo mismo es \colored{$\forall x, y. ...$}.
\end{frame}

%%%%%%%%%%%%%%%%%%%%%%%%%%%%%%%%%%%%%%%%%%%%%%%%%%%%%

\begin{frame}{Cuantificador Existencial}
  El \bolder{cuantificador existencial} es para indicar que existe \bolder{algún}
  elemento (o más de uno) que cumplen el predicado.
  \jump
  Es decir, nos indica que si aplicamos el predicado a cada individuo del universo,
  habrá al menos uno para el cual el predicado evaluará a \fulltrue.
  \jump
  El cuantificador existencial se representa con el símbolo \bolder{``$\exists$''}.
  \jump
  Si queremos decir que ``Algún individuo cumple el predicado P'' escribimos:\\
  \colored{$\exists x . P(x)$}\\
  \jump
  En nuestro ejemplo, esto es equivalente a decir:\\
  \colored{$\exists x . P(x) = P(c_1) \lor P(c_2) \lor P(c_3) \lor ... \lor P(c_{n-1}) \lor P(c_n)$}\\
\end{frame}

%%%%%%%%%%%%%%%%%%%%%%%%%%%%%%%%%%%%%%%%%%%%%%%%%%%%%

\begin{frame}{Cuantificador Existencial Negado}
  El \bolder{cuantificador existencial negado} es para indicar que \bolder{no}
  existe \bolder{ningún} individuo que cumple el predicado.
  \jump
  Es decir, nos indica que si aplicamos el predicado a cada individuo del universo,
  todos los individuos evaluarán a \fullfalse.
  \jump
  El cuantificador existencial negado se representa con el símbolo \bolder{``$\nexists$''}.
  \jump
  Si queremos decir que ``Ningún individuo cumple el predicado P'' escribimos:\\
  \colored{$\nexists x . P(x)$}\\
\end{frame}

%%%%%%%%%%%%%%%%%%%%%%%%%%%%%%%%%%%%%%%%%%%%%%%%%%%%%

\begin{frame}{Cuantificador Existencial Negado - Cont}
  En este sentido se pueden encontrar las siguientes equivalencias:\\
  \colored{$\nexists x . P(x) = \forall x . \lnot P(x)$}
  \jump
  O lo que es lo mismo:\\
  \colored{$\nexists x . P(x) = (\lnot P(c_1)) \land (\lnot P(c_2)) \land ... \land (\lnot P(c_n))$}
  \jump
  Se llama ``cuantificador existencial negado'' porque decir \colored{Ninguno ...}
  es lo mismo que decir \colored{No existe alguno que ...}.
  \jump
  Sin embargo, vemos que, a nivel de fórmula, está más cerca del cuantificador
  universal, pues también es lo mismo que decir \colored{Todos no cumplen ...}
\end{frame}

%%%%%%%%%%%%%%%%%%%%%%%%%%%%%%%%%%%%%%%%%%%%%%%%%%%%%
%%%%%%%%%%%%%%%%%%%%%%%%%%%%%%%%%%%%%%%%%%%%%%%%%%%%%

\section{La lógica de predicados en la matemática}
\frame{\tableofcontents[currentsection,currentsubsection]}

%%%%%%%%%%%%%%%%%%%%%%%%%%%%%%%%%%%%%%%%%%%%%%%%%%%%%

\begin{frame}{Lógica en la matemática}
  La lógica de predicados es muy utilizada en matemática para exponer propiedades
  sobre distintos elementos (por ejemplo, números, funciones, etc.)
  \jump
  La operaciones aritméticas como la suma, la resta, la multiplicación y la división,
  son funciones de la lógica de predicados (En particular, de aridad dos. Dados
  dos números, la suma devuelve otro número, por lo que es una forma de representar
  individuos).
  \jump
  Las comparaciones, como la igualdad, mayor qué, menor o igual qué, etc. son
  predicados (También en este caso de aridad dos, dados dos números, la igualdad
  me dice \fulltrue si efectivamente son iguales, y \fullfalse en caso contrario).
\end{frame}

%%%%%%%%%%%%%%%%%%%%%%%%%%%%%%%%%%%%%%%%%%%%%%%%%%%%%

\begin{frame}{Notación lógico-matemática}
  Como diferenciación interesante con respecto a la lógica tradicional, en matemática
  no es necesario definir el diccionario (al menos no para las operaciones básicas),
  pues el significado de los símbolos utilizados ya es conocido. Por eso, la notación matemática
  se mezcla con la notación de la lógica de predicados.
  \jump
  Por ejemplo, en lugar de escribir ``\colored{$suma(x, y)$}'' escribimos directamente
  ``\colored{x + y}''. También en lugar de escribir ``\colored{Igual(x, y)}''
  escribimos ``\colored{x = y}''.
  \jump
  Sin embargo, solo cambia la notación. Semánticamente, el significado es el mismo.
\end{frame}

%%%%%%%%%%%%%%%%%%%%%%%%%%%%%%%%%%%%%%%%%%%%%%%%%%%%%

\begin{frame}{Ejemplo conmutatividad de la suma}
  Las cosas se entienden mejor con un ejemplo, veamos entonces como expresar
  lógicamente algunas expresiones comunes.
  \jump
  \colored{La suma es conmutativa para todo par de números}.
  \jump
  \colored{
    $\forall x \in \Re, y \in \Re. x + y = y + x$
  }
  \jump
  En notación lógica tradicional escribiríamos:
  \colored{
    $\forall x.\forall y. EsReal(x) \land EsReal(y) \lthen Iguales(suma(x, y), suma(y, x))$
  }
\end{frame}

%%%%%%%%%%%%%%%%%%%%%%%%%%%%%%%%%%%%%%%%%%%%%%%%%%%%%

\begin{frame}{Mas ejemplos}
  Asociatividad:\\
  \colored{
    $\forall x \in \Re, y \in \Re, z \in \Re. (x + y) + z = x + (y + z)$
  }
  \jump
  Distributividad:\\
  \colored{
    $\forall x \in \Re, y \in \Re, z \in \Re. x \cdot (y + z) = x \cdot y  + x \cdot z$
  }
  \jump
  Neutros:\\
  \colored{
    $\forall x \in \Re. x + 0 = x$
  }\\
  \colored{
    $\forall x \in \Re. x \cdot 1 = x$
  }
\end{frame}
