% !TeX encoding = UTF-8
% !TeX spellcheck = es_AR
% arara: pdflatex
% arara: pdflatex
% arara: pdflatex
% arara: clean: {files: [machete_markdown.aux, machete_markdown.out, machete_markdown.log]}
% arara: clean: {files: [machete_markdown.fls, machete_markdown.fdb_latexmk, machete_markdown.synctex.gz]}

\documentclass[12pt, addpoints]{../../common/epyl_exam_template}

\title{Machete de Markdown}
\date{Primer semestre 2018}

\begin{document}
    \makeexamheader
    \makeexamtitle
    \examrule

    \begin{lstlisting}[language=markdown]
# Yo soy un titulo

Yo también soy un titulo exactamente igual
==========================================

Con dos iguales basta, pero se ve mejor el de arriba
==

## Yo soy un subtitulo

Igual que yo
------------

Con dos también basta
--

### Yo soy un titulo mucho menos importante
#### Yo aun menos importante
##### Yo todavía menos
###### A mi ya ni me registran como titulo, pero lo soy

Esto es texto normal, si uno quiere comenzar un nuevo párrafo (*\Suppressnumber*)
basta dejar dos lineas de espacio.(*\Reactivatenumber*)

Este por ejemplo, es un nuevo párrafo. Se pueden agregar en el (*\Suppressnumber*)
palabras en *itálica*, que es lo mismo que _cursiva_ o a veces
llamada también *oblicua*.(*\Reactivatenumber*)

También podemos poner en **negrita** o en __mas oscurito__ una (*\Suppressnumber*)
palabra o frase. Además valen links o enlaces usando
[Texto del enlace](url_a_enlazar). Por lo tanto
[Ir a Google](http://google.com) es un buen ejemplo.(*\Reactivatenumber*)

Ademas, se puede de forma similar poner imágenes con (*\Suppressnumber*)
![Descripción de la imagen](url de la imagen),
![markdown](https://bournetocode.com/projects/
AQA_AS_Theory/pages/img/markdown.jpg)(*\Reactivatenumber*)

Con tres guiones medios, se puede lograr una linea de separación, (*\Suppressnumber*)
pero ojo, hay q asegurarse de que no se pegue al texto porque de lo
contrario se entiende como subtitulo. (*\Reactivatenumber*)

---

También vale si se genera toda la linea, y queda mas bonito

---------------------------------------------------------------------

Veamos ahora algunas listas, por ejemplo una lista numerada

1. Donde yo soy el primer elemento
2. Yo el segundo
3. Y yo el tercero

En realidad para ponerse técnicos, no hace falta que indique 1, 2, 3.

1. Primero
1. Segundo
1. Tercero

También es una lista numerada. Ahora vamos a las listas no numeradas:

* Yo soy el primer elemento
* Yo el segundo
* Yo el tercero

- También podemos usar signo de menos para los ítems
+ O signos de mas, es indistinto

* Si se indenta, entonces se tienen listas dentro de listas
  + Como esta
  + Que se observa aquí

Se pueden mezclar numeradas y no numeradas

* El primero tiene dos partes
  1. Primero
  2. Segundo
* El segundo solo una
  1. Que es esta
    \end{lstlisting}

\end{document}
