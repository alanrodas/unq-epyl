\section{Lenguajes.}
\toc[currentsection,currentsubsection]

%%%%%%%%%%%%%%%%%%%%%%%%%%%%%%%%%%%%%%%%%%%%%%%%%%%%%

\begin{frame}{Lenguajes}
	\begin{block}{Lenguaje}
		\bolder{Vamos a llamar lenguaje a un conjunto de símbolos y reglas que
			determinan textos que son válidos y textos que no lo son.}
	\end{block}
\end{frame}

%%%%%%%%%%%%%%%%%%%%%%%%%%%%%%%%%%%%%%%%%%%%%%%%%%%%%

\begin{frame}{Idiomas}
  \begin{columns}
    \begin{column}{0.46\textwidth}
      Los idiomas que hablamos todos los días son lenguajes.
      \jump
      Cada uno tiene su propio conjunto de símbolos (letras, números, signos de
      puntuación, etc.) y de reglas (palabras válidas, reglas gramaticales, etc.)
      \jump
      Generalmente se denomina “lenguaje natural” al lenguaje que uno emplea para
      comunicarse habitualmente, es decir, a nuestro idioma.
    \end{column}

    \begin{column}{0.46\textwidth}
      \image[scale=0.5]{img/idiomas.png}{Banderas de distintos países.}
    \end{column}
  \end{columns}
\end{frame}

%%%%%%%%%%%%%%%%%%%%%%%%%%%%%%%%%%%%%%%%%%%%%%%%%%%%%

\begin{frame}{Idiomas Artificiales}
  Hay idiomas que consisten en lenguajes artificiales, con distintas finalidades:
  artísticas, filantrópicas, científicas, etc.
  \jump
  El Klingon, el Esperanto, el Lojban son algunos ejemplos.
  \jump
  \image[scale=0.45]{img/idiomas_artificiales.png}{Banderas de los lenguajes de idiomas artificiales.}
\end{frame}

%%%%%%%%%%%%%%%%%%%%%%%%%%%%%%%%%%%%%%%%%%%%%%%%%%%%%

\begin{frame}{Sintaxis}
  Todo lenguaje tiene una sintaxis.
  \jump
  La sintaxis consiste en los símbolos y las reglas que determinan cómo disponer
  los mismos.
  \jump
  Si un texto contiene solamente símbolos válidos, dispuestos correctamente de
  forma tal que se cumplen todas las reglas, entonces ese texto es parte del lenguaje,
  caso contrario no lo será.
\end{frame}

%%%%%%%%%%%%%%%%%%%%%%%%%%%%%%%%%%%%%%%%%%%%%%%%%%%%%

\begin{frame}{Sintaxis, orden y reglas de composición}
  Siempre hay un orden, por ejemplo, en español, al hacer una pregunta se deben
  colocar los signos de interrogación que delimitan la pregunta (¿?).
  \jump
  \bolder{¿Cuál es tu nombre?}
  \jump
  En inglés, una pregunta está delimitada por la forma de la oración, terminando
  con el signo de interrogación de cierre (?).
  \jump
  \bolder{What is your name?}
\end{frame}

%%%%%%%%%%%%%%%%%%%%%%%%%%%%%%%%%%%%%%%%%%%%%%%%%%%%%

\begin{frame}{Semántica}
  Todo lenguaje tiene además una semántica.
  \jump
  La semántica indica qué significa una determinada secuencia de símbolos.
  Por ejemplo, que representación mental nos genera al leer la palabra ``Pelota''
  o ``Correr''.
  \jump
  Es eso que generalmente vemos descrito en el diccionario, aunque la semántica
  varía de acuerdo al contexto.
\end{frame}

%%%%%%%%%%%%%%%%%%%%%%%%%%%%%%%%%%%%%%%%%%%%%%%%%%%%%

\begin{frame}{Yodalang: Lenguaje de ejemplo}
  Por ejemplo, podemos definir un lenguaje que contenga como símbolos solo
  las letras del idioma español en mayúscula, y el espacio en blanco:
  \jump
  \textbf{A, B, C, D, E, F, G, H, I, J, K, L, M, N, Ñ, O, P, Q, R, S, T, U, V, W, X, Y, Z.}
  \jump
  También vamos a aceptar las letras acentuadas.
  \jump
  Y como reglas, vamos a definir que los símbolos se pueden agrupar uno tras
  otro, formando palabras como en el español, pero que a diferencia de este
  idioma, en cada renglón solo puede haber tres palabras (o sea, solo puede
  haber dos espacios por renglón).
\end{frame}

%%%%%%%%%%%%%%%%%%%%%%%%%%%%%%%%%%%%%%%%%%%%%%%%%%%%%

\begin{frame}{Yodalang: Sintaxis}
  En el lenguaje que definimos estas frases son válidas:
  \begin{itemize}
    \item \success{VÁLIDO ESTO ES.}
    \item \success{INVÁLIDO ESTO NO.}
    \item \success{VENCER SITH DEBES.}
  \end{itemize}
  \jump
  Estas frases no lo son:
  \begin{itemize}
    \item \error{VÁLIDO ESTO NO ES.}
    \item \error{DARTH VADER TU PADRE ES.}
  \end{itemize}
\end{frame}

%%%%%%%%%%%%%%%%%%%%%%%%%%%%%%%%%%%%%%%%%%%%%%%%%%%%%

\begin{frame}{Yodalang: Semántica}
  En el lenguaje que definimos esas frases no significan absolutamente nada,
  pues nunca definimos su semántica.
  \jump
  Deberíamos, para definir el lenguaje, decir que significa cada palabra de
  nuestro lenguaje para determinar qué es lo que se debe interpretar de cada
  renglón, y como las frases deben ser interpretadas.
  \jump
  \textit{Ojo, cómo usamos los mismos símbolos del español, y las mismas
  palabras, nuestra mente tiende a asociar automáticamente un significado, pero
  esto no es válido.}
\end{frame}

%%%%%%%%%%%%%%%%%%%%%%%%%%%%%%%%%%%%%%%%%%%%%%%%%%%%%

\begin{frame}{Yodalang: Semántica, continuación}
  Sin definir la semántica, es imposible determinar qué significa una frase:
  \jump
  \bolder{VENCER SITH DEBES.}
  \begin{itemize}
    \item ¿Debemos interpretar \textbf{``Debes vencer a los Sith''} o
      \textbf{``Sith, debes vencer''}?
    \item ¿Será que \textbf{``VENCER SITH DEBES''} es lo mismo que
      \textbf{``DEBES SITH VENCER''}?
  \end{itemize}
\end{frame}

%%%%%%%%%%%%%%%%%%%%%%%%%%%%%%%%%%%%%%%%%%%%%%%%%%%%%

\begin{frame}{Yodalang: Desambiguación}
  Un humano puede entender lenguajes con sintaxis y semánticas complejas gracias
  a que puede pensar y desambiguar.
  \jump
  \bolder{Las computadoras no pueden desambiguar ni entender contextos}.
  \jump
  Los lenguajes que las computadoras pueden comprender y procesar suelen tener
  una sintaxis sencilla y una semántica sin ambigüedades.
\end{frame}

%%%%%%%%%%%%%%%%%%%%%%%%%%%%%%%%%%%%%%%%%%%%%%%%%%%%%
%%%%%%%%%%%%%%%%%%%%%%%%%%%%%%%%%%%%%%%%%%%%%%%%%%%%%

\section{Lenguajes de Marcado.}
\toc[currentsection,currentsubsection]

%%%%%%%%%%%%%%%%%%%%%%%%%%%%%%%%%%%%%%%%%%%%%%%%%%%%%

\begin{frame}{Lenguajes de Marcado}
  Un lenguaje de marcado es un lenguaje en donde algunos símbolos cobran un
  significado especial. Son marcas que indican algo, por ejemplo qué función
  cumple un determinado texto, o cómo debería verse en pantalla con algún
  visualizador.
  \jump
  \begin{columns}
    \begin{column}{0.3\textwidth}
      Ej. Whatsapp.
    \end{column}
    \begin{column}{0.3\textwidth}
      \image[scale=0.5]{img/whatsapp.png}{Whatsapp mostrando mensajes con estilos y el código de marcado que lo genera}
    \end{column}
  \end{columns}
\end{frame}

%%%%%%%%%%%%%%%%%%%%%%%%%%%%%%%%%%%%%%%%%%%%%%%%%%%%%

\begin{frame}{Lenguajes de marcado: Utilidad}
  \begin{itemize}
    \item Un lenguaje de marcado sirve para comunicar contenido de forma semántica.
    \item Un lenguaje de marcado no es un lenguaje de programación.
    \item Los lenguajes de marcado son utilizados en la industria editorial y de
      la comunicación, así como entre autores, editores e impresores.
    \item Los lenguajes de marcado son ampliamente usado para generar documentación
      que acompaña a proyectos de software.
    \item El lenguaje de marcado más popular, HTML, sirve para diseñar y
      estructurar un sitio web.
  \end{itemize}
\end{frame}

%%%%%%%%%%%%%%%%%%%%%%%%%%%%%%%%%%%%%%%%%%%%%%%%%%%%%

\begin{frame}{Lenguajes de marcado: Características}
  Suelen ser \bolder{fáciles de leer y escribir por humanos}, pero no contienen
  ambigüedades de ningún tipo en su semántica, permitiendo que sean
  \bolder{interpretables por una computadora}.
  \jump
  Hay muchos lenguaje de marcado (cada lenguaje tiene su conjunto específico de reglas)
  \jump
  Cada lenguaje de marcado tiene una \bolder{utilidad específica}: generar
  documentos en PDF, escribir documentación de proyectos, describir páginas web, etc.
  \jump
  Vamos a mencionar algunos de ellos.
\end{frame}

%%%%%%%%%%%%%%%%%%%%%%%%%%%%%%%%%%%%%%%%%%%%%%%%%%%%%

\subsection{LaTeX.}
\toc[currentsection,currentsubsection]

%%%%%%%%%%%%%%%%%%%%%%%%%%%%%%%%%%%%%%%%%%%%%%%%%%%%%

\begin{frame}{LaTeX}
  \bolder{LaTeX} es un lenguaje que incluye marcas que \bolder{permiten generar
  documentos listos para imprimir} (PDF o similares).
  \jump
  Es un lenguaje muy complejo, pero muy poderoso.
  \jump
  Es el lenguaje utilizado habitualmente para escribir papers científicos, tesis
  de licenciatura, libros técnicos, y mucho más.
  \jump
  \textit{\colored{Lo van a tener que usar en algún punto de la carrera, tanto
  para entregar TPs como para el trabajo final de la carrera.}}
\end{frame}

%%%%%%%%%%%%%%%%%%%%%%%%%%%%%%%%%%%%%%%%%%%%%%%%%%%%%

\begin{frame}{LaTeX: Ejemplo}
  \begin{columns}
    \begin{column}{0.5\textwidth}
      \image[scale=0.35]{img/latex_code.png}{Código que genera un documento LaTeX}
    \end{column}
    \begin{column}{0.5\textwidth}
      \image[scale=0.35]{img/latex_compiled.png}{El documento LaTeX compilado en un documento}
    \end{column}
  \end{columns}
\end{frame}

%%%%%%%%%%%%%%%%%%%%%%%%%%%%%%%%%%%%%%%%%%%%%%%%%%%%%

\subsection{BBCode.}
\toc[currentsection,currentsubsection]

%%%%%%%%%%%%%%%%%%%%%%%%%%%%%%%%%%%%%%%%%%%%%%%%%%%%%

\begin{frame}{BBCode}
  \bolder{BBCode} es un lenguaje de marcado que es \bolder{utilizado en foros de
  internet} para embellecer los mensajes que envían los usuarios.
  \jump
  El código BBCode permite marcar que texto irá en negrita, subrayado, determinar
  el color, etc.
  \jump
  El código BBCode permite incluso agregar emoticones y cosas similares
  escribiendo simplemente una marca en el texto.
\end{frame}

%%%%%%%%%%%%%%%%%%%%%%%%%%%%%%%%%%%%%%%%%%%%%%%%%%%%%

\begin{frame}{BBCode: Ejemplo}
  Puede verse un ejemplo completo en:
  \colored{\href{http://www.roleplayerguild.com/bbcode}{http://www.roleplayerguild.com/bbcode}}
  \image[scale=0.47]{img/bbcode.png}{Muestra de las posibilidades de BBCode.}
\end{frame}

%%%%%%%%%%%%%%%%%%%%%%%%%%%%%%%%%%%%%%%%%%%%%%%%%%%%%

\subsection{WikiText.}
\toc[currentsection,currentsubsection]

%%%%%%%%%%%%%%%%%%%%%%%%%%%%%%%%%%%%%%%%%%%%%%%%%%%%%

\begin{frame}{WikiText}
  Es un lenguaje de marcado \bolder{diseñado específicamente para páginas de wikis},
  por ejemplo Wikipedia.
  \jump
  Incluye marcas de estilo, como negrita, resaltado, etc. pero el símbolo usado
  para esas marcas depende del software que se utilice para la wiki.
  \jump
  Wikipedia ha propuesto un estándar conocido como MediaWiki que ha sido
  ampliamente adoptado.
\end{frame}

%%%%%%%%%%%%%%%%%%%%%%%%%%%%%%%%%%%%%%%%%%%%%%%%%%%%%

\begin{frame}{WikiText: Ejemplo}
  \image[scale=0.43]{img/wikitext.png}{Muestra de codigo de WikiText en Wikipedia}
\end{frame}

%%%%%%%%%%%%%%%%%%%%%%%%%%%%%%%%%%%%%%%%%%%%%%%%%%%%%

\subsection{Markdown.}
\toc[currentsection,currentsubsection]

%%%%%%%%%%%%%%%%%%%%%%%%%%%%%%%%%%%%%%%%%%%%%%%%%%%%%

\begin{frame}{Markdown}
  Es un lenguaje de marcado \bolder{diseñado para maximizar la facilidad de lectura},
  tanto en su entrada como en su salida.
  \jump
  Se inspira fuertemente en convenciones usadas al momento de enviar correos
  electrónicos en texto plano.
  \jump
  Muchos sistemas de blogs permiten escribir artículos en código Markdown.
  \jump
  En programación es ampliamente utilizado para escribir documentación de los
  programas.
  \jump
  \textit{Veremos más de este lenguaje en las próximas clases}
\end{frame}

%%%%%%%%%%%%%%%%%%%%%%%%%%%%%%%%%%%%%%%%%%%%%%%%%%%%%

\subsection{HTML}
\toc[currentsection,currentsubsection]

%%%%%%%%%%%%%%%%%%%%%%%%%%%%%%%%%%%%%%%%%%%%%%%%%%%%%

\begin{frame}{HTML}
  \bolder{HyperText Markup Language} (HTML) es un lenguaje de marcado que sirve para
  \bolder{describir páginas web}.
  \jump
  Es el lenguaje de marcado más extendido en su uso.
  \jump
  El principio básico del HTML son las \bolder{etiquetas}, que actúan de marcas para
  indicar distintos tipos de contenidos en un sitio web.
  \jump
  HTML se orienta a describir qué contenidos van a ser presentados en un sitio
  web, y estructurarlos de forma semántica.
  \jump
  \textit{También lo veremos en profundidad en la próxima clase}
\end{frame}
