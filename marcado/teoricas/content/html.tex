\section{HTML.}
\subsection{El lenguaje.}
\toc[currentsection,currentsubsection]

%%%%%%%%%%%%%%%%%%%%%%%%%%%%%%%%%%%%%%%%%%%%%%%%%%%%%

\begin{frame}{HTML}
  \bolder{HyperText Markup Language} (HTML) es un lenguaje de marcado que sirve
  para describir páginas web.
  \jump
  Los archivos HTML son archivos de texto, que pueden editarse con cualquier
  editor de texto, y visualizarse con un \bolder{Browser}.
\end{frame}

%%%%%%%%%%%%%%%%%%%%%%%%%%%%%%%%%%%%%%%%%%%%%%%%%%%%%

\begin{frame}{Browser}
  El \bolder{Browser} o \bolder{Navegador web} permite visualizar archivos HTML
  que se encuentran en el equipo, así como también visualizar archivos HTML
  que se encuentran en otros equipos (Internet).
  \jump
  Al solicitar un sitio web en el browser, este realiza distintos mecanismos
  para determinar en que equipo se encuentra el archivo HTML a visualizar, lo
  descarga, y se lo presenta al usuario.
  \jump
  Hay muchos browsers (algunos más populares que otros) de distintas compañías.
  Internet Explorer o Edge de Microsoft, Chrome de Google, Safari de Apple,
  Firefox de la Fundación Mozilla, Opera de Opera Software, son algunos de los
  más conocidos.
  \image[scale=0.2]{img/browsers.png}{Logos de los navegadores más populares}
\end{frame}

%%%%%%%%%%%%%%%%%%%%%%%%%%%%%%%%%%%%%%%%%%%%%%%%%%%%%

\begin{frame}[fragile]{Etiquetas HTML}
	Una etiqueta HTML tiene un nombre que va dentro de los signos de etiqueta.
	\jump
	La mayoría de las etiquetas deben abrir y cerrar, las etiquetas de apertura
  llevan los signos de ``<'' (menor) y ``>'' (mayor) para indicar el nombre,
  mientras que las de cierre llevan ``</'' (menor seguido de barra) y ``>'' (mayor).
	\jump
	Ejemplo apertura:
	\begin{lstlisting}[language=HTML]
<nombre_de_etiqueta>
	\end{lstlisting}

	Ejemplo cierre
	\begin{lstlisting}[language=HTML]
</nombre_de_etiqueta>
	\end{lstlisting}
\end{frame}

%%%%%%%%%%%%%%%%%%%%%%%%%%%%%%%%%%%%%%%%%%%%%%%%%%%%%

\begin{frame}[fragile]{Contenido HTML}
  El contenido de una etiqueta se coloca entre la etiqueta de apertura y la de
  cierre.
	\jump
	Ejemplo:
	\begin{lstlisting}[language=HTML]
<nombre_etiqueta> Contenido </nombre_etiqueta>
	\end{lstlisting}
\end{frame}

%%%%%%%%%%%%%%%%%%%%%%%%%%%%%%%%%%%%%%%%%%%%%%%%%%%%%

\begin{frame}[fragile]{Etiquetas anidadas en HTML}
	Una etiqueta puede contener otras etiquetas dentro, como contenido.
	\jump
	Ejemplo:
	\begin{lstlisting}[language=HTML]
<etiqueta1>
	<etiqueta2>
		Contenido de la etiqueta
	</etiqueta2>
</etiqueta1>
	\end{lstlisting}
\end{frame}

%%%%%%%%%%%%%%%%%%%%%%%%%%%%%%%%%%%%%%%%%%%%%%%%%%%%%

\begin{frame}[fragile]{Atributos}
	Algunas etiquetas llevan atributos, valores que indican cosas acerca de dicha
	etiqueta.
	\jump
	Un atributo se indica dentro de las marcas de etiquetas, luego del nombre de
	la etiqueta, indicando el nombre del atributo, el signo igual y el valor
	asociado de dicho atributo entre comillas.
	\jump
	Ejemplo:
	\begin{lstlisting}[language=HTML]
<nombre_de_etiqueta atributo="valor">
	\end{lstlisting}
\end{frame}

%%%%%%%%%%%%%%%%%%%%%%%%%%%%%%%%%%%%%%%%%%%%%%%%%%%%%

\begin{frame}[fragile]{Atributos Cont.}
	Los atributos se separan con espacios, y puede haber tantos como se quiera.
	\jump
  Ejemplo:
	\begin{lstlisting}[language=HTML]
<nombre attr1="val1" attr2="val2">
	\end{lstlisting}
\end{frame}

%%%%%%%%%%%%%%%%%%%%%%%%%%%%%%%%%%%%%%%%%%%%%%%%%%%%%

\subsection{Comunicación del código.}
\toc[currentsection,currentsubsection]

%%%%%%%%%%%%%%%%%%%%%%%%%%%%%%%%%%%%%%%%%%%%%%%%%%%%%

\begin{frame}[fragile]{Comunicación e Indentación}
	El código
	\begin{lstlisting}[language=HTML]
<etiqueta1><etiqueta2>Contenido de la etiqueta
</etiqueta2></etiqueta1>
	\end{lstlisting}
	Es equivalente a
	\begin{lstlisting}[language=HTML]
<etiqueta1>
	<etiqueta2>
		Contenido de la etiqueta
	</etiqueta2>
</etiqueta1>
	\end{lstlisting}
\end{frame}

%%%%%%%%%%%%%%%%%%%%%%%%%%%%%%%%%%%%%%%%%%%%%%%%%%%%%

\begin{frame}[fragile]{Indentación}
	Sin embargo, el segundo se lee mejor.
	\jump
	Esto no es casual, vamos a elegir siempre indentar correctamente el código.
	\jump
	La indentación es el tabulado, o sangría que se coloca en el código, de forma
	tal de hacerlo más legible. También implica los saltos de línea de forma correcta.
	\jump
\begin{lstlisting}[language=HTML]
<etiqueta1>
	<etiqueta2>
		Contenido de la etiqueta
	</etiqueta2>
</etiqueta1>
\end{lstlisting}
\end{frame}

%%%%%%%%%%%%%%%%%%%%%%%%%%%%%%%%%%%%%%%%%%%%%%%%%%%%%

\begin{frame}[fragile]{Comentarios}
	Los comentarios permiten agregar información que ayudan a entender el código
	del sitio web, pero que es completamente ignorado por el navegador.
	\jump
	Un comentario es similar a una etiqueta, pero comienza con ``<!--'' y termina
	con ``-->''.
	\begin{lstlisting}[language=HTML]
<!-- Esto es un comentario,
	puede ponerse lo que se desee aquí -->
	\end{lstlisting}
\end{frame}

%%%%%%%%%%%%%%%%%%%%%%%%%%%%%%%%%%%%%%%%%%%%%%%%%%%%%

\subsection{Etiquetas del HTML}
\toc[currentsection,currentsubsection]

%%%%%%%%%%%%%%%%%%%%%%%%%%%%%%%%%%%%%%%%%%%%%%%%%%%%%

\begin{frame}{Etiquetas básicas}
	Veamos algunos ejemplos básicos de etiquetas:
	\begin{itemize}
		\item \bolder{p}: Delimita un párrafo
		\item \bolder{strong}: Indica que una palabra o frase debe estar remarcada,
			en negrita.
		\item \bolder{em} Indica que una palabra o frase debe estar enfatizada, en
			itálica.
		\item \bolder{section}: Delimita un sección, probablemente conteniendo varios párrafos.
			Cumple solo una función organizacional, pero no visual.
	\end{itemize}
\end{frame}

%%%%%%%%%%%%%%%%%%%%%%%%%%%%%%%%%%%%%%%%%%%%%%%%%%%%%

\begin{frame}{Etiquetas de títulos}
	También podemos generar títulos:
	\begin{itemize}
		\item \bolder{h1}: Título
		\item \bolder{h2}: Subtitulo
		\item \bolder{h3}: Título de 3er nivel
		\item \bolder{h4}: Título de 4to nivel
		\item \bolder{h5}: Título de 5to nivel
		\item \bolder{h6}: Título de 6to nivel
	\end{itemize}
\end{frame}

%%%%%%%%%%%%%%%%%%%%%%%%%%%%%%%%%%%%%%%%%%%%%%%%%%%%%

\begin{frame}{Etiquetas de listas}
	Se pueden crear listas:
	\begin{itemize}
		\item \bolder{ul}: Lista sin orden (con viñetas). Requiere ítems adentro.
		\item \bolder{ol}: Lista ordenada (numerada). Requiere ítems adentro.
		\item \bolder{li}: Item de una lista
	\end{itemize}
\end{frame}

%%%%%%%%%%%%%%%%%%%%%%%%%%%%%%%%%%%%%%%%%%%%%%%%%%%%%

\begin{frame}{Etiquetas de enlace e imágen}
	O enlaces e imágenes:
	\begin{itemize}
		\item \bolder{a}: Un enlace. Requiere un atributo obligatorio:
		\begin{itemize}
			\item \bolder{href}: Indica a donde se esa enlazando.
		\end{itemize}
		\item \bolder{img}: Una imagen. Requiere un atributo obligatorio:
		\begin{itemize}
			\item \bolder{src}: Indica donde se ubica la imagen que se desea mostrar
		\end{itemize}
		Y otros opcionales:
		\begin{itemize}
			\item \bolder{width}: Indica el ancho de la imagen
			\item \bolder{height}: Indica el alto de la imagen
		\end{itemize}
	\end{itemize}
\end{frame}

%%%%%%%%%%%%%%%%%%%%%%%%%%%%%%%%%%%%%%%%%%%%%%%%%%%%%

\begin{frame}{Etiquetas de tablas}
	Se pueden crear tablas de datos (datos de dos entradas)
	\begin{itemize}
		\item \bolder{table}: Una tabla, requiere que se declare dentro el contenido.
		\item \bolder{tr}: Genera una fila en la tabla. Requiere celdas dentro.
		\item \bolder{td}: Genera una celda en la tabla, requiere contenido dentro.
	\end{itemize}
\end{frame}

%%%%%%%%%%%%%%%%%%%%%%%%%%%%%%%%%%%%%%%%%%%%%%%%%%%%%
%%%%%%%%%%%%%%%%%%%%%%%%%%%%%%%%%%%%%%%%%%%%%%%%%%%%%

\subsection{Estructura de un documento}
\toc[currentsection,currentsubsection]

%%%%%%%%%%%%%%%%%%%%%%%%%%%%%%%%%%%%%%%%%%%%%%%%%%%%%

\begin{frame}[fragile]{Estructura básica de documento}
	Todo documento HTML comienza siempre con una pseudo-etiqueta que tiene por
	finalidad indicarle al navegador que se trata de un documento HTML (Otros
	lenguajes usan etiquetas similares para describir cosas distintas a páginas
	web).
	\jump
	La pseudo-etiqueta debe ser en todos los casos la siguiente:
	\begin{lstlisting}[language=HTML]
<!DOCTYPE html>
	\end{lstlisting}
\end{frame}

%%%%%%%%%%%%%%%%%%%%%%%%%%%%%%%%%%%%%%%%%%%%%%%%%%%%%

\begin{frame}{Estructura básica de documento - Cont.}
	Luego de la pseudo-etiqueta debe haber una única etiqueta que englobe todos
	los elementos siguientes.
	\jump
	Esta etiqueta tiene por nombre \bolder{html} y debe ser la única etiqueta externa.
	\jump
	La etiqueta html debe contener a su vez dos sub-etiquetas, \bolder{head} y \bolder{body}.
\end{frame}

%%%%%%%%%%%%%%%%%%%%%%%%%%%%%%%%%%%%%%%%%%%%%%%%%%%%%

\begin{frame}{Estructura básica de documento - Cont.}
	Dentro de la etiqueta head se agregará meta-información que el browser puede
	utilizar para saber como mostrar el contenido, como el título que debe poner
	en la ventana, o el ícono a mostrar en la pestaña, o el lenguaje en el que se
	espera se encuentre la página.
	\jump
	La etiqueta body en cambio engloba a todos los elementos que representa
	contenido que va a ser mostrado en la página web. Si es algo visible en el
	sitio va dentro del body.
\end{frame}

%%%%%%%%%%%%%%%%%%%%%%%%%%%%%%%%%%%%%%%%%%%%%%%%%%%%%

\begin{frame}[fragile]{Estructura básica de documento - Cont.}
	Como vamos a escribir en español nos va a interesar agregar una de las etiquetas
	de metadatos dentro de head que va a indicarle que el archivo contiene caracteres
	en español. La etiqueta \bolder{meta} con el atributo \bolder{charset}.

	Así que la estructura general del documento nos va a quedar así:
	\begin{lstlisting}[language=HTML]
<!DOCTYPE html>
<html>
	<head>
		<meta charset="utf-8">
		<!-- Otros metadatos relevantes -->
	</head>
	<body>
		<!-- Aquí va el contenido del sitio -->
	</body>
</html>
	\end{lstlisting}
\end{frame}

%%%%%%%%%%%%%%%%%%%%%%%%%%%%%%%%%%%%%%%%%%%%%%%%%%%%%
%%%%%%%%%%%%%%%%%%%%%%%%%%%%%%%%%%%%%%%%%%%%%%%%%%%%%

\subsection{Ejemplos}
\toc[currentsection,currentsubsection]

%%%%%%%%%%%%%%%%%%%%%%%%%%%%%%%%%%%%%%%%%%%%%%%%%%%%%

\begin{frame}[fragile]{Ejemplos de HTML - 1}
	\begin{lstlisting}[language=HTML]
<!DOCTYPE html>
<html>
	<head> <meta charset="utf-8"> </head>
	<body>
		<h1>Titulo principal</h1>
		<p>
			Este es el primer párrafo. En el se
			puede agregar <strong>negritas</strong>
			o <em>itálicas</em>.
		</p>
		<p>
			Podemos agregar otros párrafos
		</p>
	</body>
</html>
	\end{lstlisting}
\end{frame}

%%%%%%%%%%%%%%%%%%%%%%%%%%%%%%%%%%%%%%%%%%%%%%%%%%%%%

\begin{frame}[fragile]{Ejemplos de HTML - 1 Cont.}
	\image[scale=0.4]{img/html_ejemplo_1.png}{Resultado del ejemplo anterior}
\end{frame}

%%%%%%%%%%%%%%%%%%%%%%%%%%%%%%%%%%%%%%%%%%%%%%%%%%%%%

\begin{frame}[fragile]{Ejemplos de HTML - 2}
	\begin{lstlisting}[language=HTML]
<!DOCTYPE html>
<html>
	<head> <meta charset="utf-8"> </head>
	<body>
		<h1>Enlaces y subtítulos</h1>
		<p>
			También podemos agregar enlaces como:
			<a href="http://google.com">Ir a Google</a>
		</p>
		<h2>O subtítulos</h2>
		<p>
			Y luego comenzar en nuevo párrafo.
		</p>
	</body>
</html>
	\end{lstlisting}
\end{frame}

%%%%%%%%%%%%%%%%%%%%%%%%%%%%%%%%%%%%%%%%%%%%%%%%%%%%%

\begin{frame}[fragile]{Ejemplos de HTML - 2 Cont.}
	\image[scale=0.4]{img/html_ejemplo_2.png}{Resultado del ejemplo anterior}
\end{frame}

%%%%%%%%%%%%%%%%%%%%%%%%%%%%%%%%%%%%%%%%%%%%%%%%%%%%%

\begin{frame}[fragile]{Ejemplos de HTML - 3}
	\begin{lstlisting}[language=HTML]
<!DOCTYPE html>
<html>
	<head>
		<meta charset="utf-8">
	</head>
	<body>
		<h1>Listas</h1>
		<h3>Una lista numerada:</h3>
		<ol>
			<li>El primer elemento</li>
			<li>El segundo elemento</li>
			<li>El tercero</li>
		</ol>
	\end{lstlisting}
\end{frame}

%%%%%%%%%%%%%%%%%%%%%%%%%%%%%%%%%%%%%%%%%%%%%%%%%%%%%

\begin{frame}[fragile]{Ejemplos de HTML - 3 - Cont.}
	\begin{lstlisting}[language=HTML]
		<h3>Una lista con viñetas:</h3>
		<ul>
			<li>El primer elemento</li>
			<li>El segundo elemento</li>
			<li>El tercero</li>
		</ul>
	</body>
</html>
	\end{lstlisting}
\end{frame}

%%%%%%%%%%%%%%%%%%%%%%%%%%%%%%%%%%%%%%%%%%%%%%%%%%%%%

\begin{frame}[fragile]{Ejemplos de HTML - 3 Cont.}
	\image[scale=0.4]{img/html_ejemplo_3.png}{Resultado del ejemplo anterior}
\end{frame}

%%%%%%%%%%%%%%%%%%%%%%%%%%%%%%%%%%%%%%%%%%%%%%%%%%%%%

\begin{frame}[fragile]{Ejemplos de HTML - 4}
	\begin{lstlisting}[language=HTML]
<!DOCTYPE html>
<html>
	<head> <meta charset="utf-8"> </head>
	<body>
		<h1>Una tabla de ejemplo</h1>
		<table>
			<tr>
				<th>Nombre</th>
				<th>Apellido</th>
				<th>Edad</th>
			</tr> <tr>
				<td>Juan</td>
				<td>Perez</td>
				<td>25</td>
	\end{lstlisting}
\end{frame}

%%%%%%%%%%%%%%%%%%%%%%%%%%%%%%%%%%%%%%%%%%%%%%%%%%%%%

\begin{frame}[fragile]{Ejemplos de HTML - 4 Cont.}
	\begin{lstlisting}[language=HTML]
			</tr> <tr>
				<td>Cosme</td>
				<td>Fulanito</td>
				<td>23</td>
			</tr> <tr>
				<td>Susana</td>
				<td>Mengano</td>
				<td>28</td>
			</tr>
		</table>
	</body>
</html>
	\end{lstlisting}
\end{frame}

%%%%%%%%%%%%%%%%%%%%%%%%%%%%%%%%%%%%%%%%%%%%%%%%%%%%%

\begin{frame}[fragile]{Ejemplos de HTML - 4 Cont.}
	\image[scale=0.4]{img/html_ejemplo_4.png}{Resultado del ejemplo anterior}
\end{frame}


%%%%%%%%%%%%%%%%%%%%%%%%%%%%%%%%%%%%%%%%%%%%%%%%%%%%%
%%%%%%%%%%%%%%%%%%%%%%%%%%%%%%%%%%%%%%%%%%%%%%%%%%%%%

\subsection{Etiquetas de estructura}
\toc[currentsection,currentsubsection]

%%%%%%%%%%%%%%%%%%%%%%%%%%%%%%%%%%%%%%%%%%%%%%%%%%%%%

\begin{frame}[fragile]{Etiquetas de estructura}
	Unas de las etiquetas más importantes son las etiquetas que indican estructura.
	\jump
	Estas etiquetas no cumplen una función visual, sino que simplemente indican
	secciones importantes del sitio, agrupando los contenidos de forma semántica
	\jump
	El poder agrupar semánticamente los contenidos permite no solo comprender
	mejor el código, sino también que servicios como Google u otros comprender
	mejor el contenido del sitio e indexarlo correctamente.
\end{frame}

%%%%%%%%%%%%%%%%%%%%%%%%%%%%%%%%%%%%%%%%%%%%%%%%%%%%%

\begin{frame}[fragile]{Etiquetas de estructura}
	\begin{itemize}
		\item \bolder{header}: Indica la cabecera del sitio (logo de la página con el
			nombre, otros elementos que suelen ser constante en los diseños de los sitios).
			Pueden haber headers dentro de una sección o artículo, en casos muy raros.
			NO DEBE SER CONFUNDIDO CON HEAD.
		\item \bolder{nav}: Indica una barra de navegación que contiene links a distintas
			secciones del sitio. Suele estar dentro del header.
		\item \bolder{footer}: El footer es el pie de página, suele indicar cosas como
			el nombre del autor, información de contacto, información de copyright, etc.
		\item \bolder{article}: Se puede usar en sitios de noticias y blogs para englobar
			la nota y otros contenidos.
		\item \bolder{section}: Indica una sección del sitio, por ejemplo, dentro de una
			nota, puede haber varias secciones.
\end{itemize}
\end{frame}

%%%%%%%%%%%%%%%%%%%%%%%%%%%%%%%%%%%%%%%%%%%%%%%%%%%%%

\begin{frame}[fragile]{Etiquetas de estructura}
	\begin{itemize}
		\item \bolder{section}: Indica una sección del sitio, por ejemplo, dentro de una
			nota, puede haber varias secciones.
		\item \bolder{aside}:
			Suele usarse para contenidos secundarios, que no son parte prioritaria del
			contenido del sitio. Por ejemplo, cosas que van en las barras laterales de
			un sitio, incluyendo publicidades, calendarios, etc.
		\item \bolder{div}: Aquellas agrupaciones que no caen en ninguna de las
			otras categorías o que se usan por cuestiones semánticas o de estilo.
\end{itemize}
\end{frame}

%%%%%%%%%%%%%%%%%%%%%%%%%%%%%%%%%%%%%%%%%%%%%%%%%%%%%

\begin{frame}[fragile]{Etiquetas de estructura - Ejemplo}
	Ej. La información en un sitio web de noticias:
	\begin{lstlisting}[language=HTML]
<!DOCTYPE html>
<html>
<head>
	<meta charset="utf-8">
</head>
<body>
<section>
	<h2>Noticias destacadas del día</h2>
	<article>
		<h4>Los liberales se levantan en armas</h4>
	\end{lstlisting}
\end{frame}

%%%%%%%%%%%%%%%%%%%%%%%%%%%%%%%%%%%%%%%%%%%%%%%%%%%%%


\begin{frame}[fragile]{Etiquetas de estructura - Ejemplo}
	\begin{lstlisting}[language=HTML]
		<p>
			El coronel <strong>Aureliano
			Buendía</strong> habría marchado
			a la guerra al grito de <em>
			Viva el partido liberal</em>.
			Se esperan fuertes repercusiones
			en todo Macondo.
		</p>
	</article>
	\end{lstlisting}
\end{frame}

%%%%%%%%%%%%%%%%%%%%%%%%%%%%%%%%%%%%%%%%%%%%%%%%%%%%%

\begin{frame}[fragile]{Etiquetas de estructura - Ejemplo Cont.}
	\begin{lstlisting}[language=HTML]
	<article>
		<h4>Se desata la guerra contra Eurasia</h4>
		<p>
			La guerra contra <em>Eurasia</em> ha
			comenzado y soldados de toda Oceanía
			marchan hacia la frontera para defender
			la nación.
		</p>
	</article>
</section>
</body>
</html>
	\end{lstlisting}
\end{frame}

%%%%%%%%%%%%%%%%%%%%%%%%%%%%%%%%%%%%%%%%%%%%%%%%%%%%%

\begin{frame}[fragile]{Etiquetas de estructura - Ejemplo Cont.}
	\image[scale=0.4]{img/html_ejemplo_seccionado.png}{Resultado del ejemplo anterior}
\end{frame}

%%%%%%%%%%%%%%%%%%%%%%%%%%%%%%%%%%%%%%%%%%%%%%%%%%%%%
%%%%%%%%%%%%%%%%%%%%%%%%%%%%%%%%%%%%%%%%%%%%%%%%%%%%%

\section{Uso de URIs}
\toc[currentsection,currentsubsection]

%%%%%%%%%%%%%%%%%%%%%%%%%%%%%%%%%%%%%%%%%%%%%%%%%%%%%

\begin{frame}[fragile]{URI: Referencias absolutas}
	Supongamos el siguiente ejemplo:
	\jump
	\colored{<img src=”C:/Usuarios/Juancito/Escritorio/Sitio/mi\_foto.png”>}
	\jump
	Si enviamos nuestro archivo HTML a otra persona, y este intenta abrir el
	archivo, no verá la imagen, pues no la tiene en su equipo.
	\jump
	Peor aún, la ruta completa probablemente no exista en la máquina de nadie más
	que Juancito.
	\jump
	El problema es que se están usando URIs absolutas que referéncian a un archivo
	local.
\end{frame}

%%%%%%%%%%%%%%%%%%%%%%%%%%%%%%%%%%%%%%%%%%%%%%%%%%%%%

\begin{frame}[fragile]{URI: Referencias web}
	Ahora el siguiente ejemplo:
	\jump
	\colored{<img src=”https://imgs.xkcd.com/comics/tags\_2x.png”>}
	\jump
	Si enviamos nuestro archivo HTML a otra persona, este verá la imagen,
	siempre y cuando tenga conexión a internet (http y https se usan para recursos
	web).
	\jump
	No está bueno si no sabemos con seguridad que la otra persona tiene conexión.
\end{frame}

%%%%%%%%%%%%%%%%%%%%%%%%%%%%%%%%%%%%%%%%%%%%%%%%%%%%%

\begin{frame}[fragile]{URI: Rutas locales}
	Un sitio web de verdad está compuesto, no solo de su HTML, sino de todos
	los archivos que lo acompañan, por ejemplo, imágenes, videos, etc.
	\jump
	Generalmente uno recurre a colocar todo dentro de una carpeta, con subcarpetas
	de ser necesario. Una vez hecho esto, uno puede referenciar a los archivos de
	esa carpeta con URIs relativas.
\end{frame}

%%%%%%%%%%%%%%%%%%%%%%%%%%%%%%%%%%%%%%%%%%%%%%%%%%%%%

\begin{frame}{URI: Rutas locales}
	Consideremos la siguiente estructura:
	\begin{columns}
		\begin{column}{0.5\textwidth}
			\image[scale=0.4]{img/folders.png}{Estructuras de carpetas del proyecto HTML}
		\end{column}
		\begin{column}{0.5\textwidth}
			Desde el archivo sitio.html ahora podemos hacer referencia a las imágenes
			del sitio refiriéndonos siempre al camino que hay que seguir desde la
			carpeta principal del proyecto.
			\jump
			\colored{<img src=”fotos/principal.png”>}
			\jump
			\colored{<img src=”fotos/personales/juan.jpg”>}
		\end{column}
	\end{columns}
\end{frame}

%%%%%%%%%%%%%%%%%%%%%%%%%%%%%%%%%%%%%%%%%%%%%%%%%%%%%

\begin{frame}[fragile]{TIP Entregas}
	Cuando debamos enviar un sitio web que contenga recursos adicionales al HTML,
	siempre debemos utilizar rutas relativas y enviar toda la carpeta completa.
	\jump
	Cuando no se tengan archivos adicionales, basta con el archivo HTML.
\end{frame}

%%%%%%%%%%%%%%%%%%%%%%%%%%%%%%%%%%%%%%%%%%%%%%%%%%%%%
%%%%%%%%%%%%%%%%%%%%%%%%%%%%%%%%%%%%%%%%%%%%%%%%%%%%%

\section{Relación con Markdown}
\toc[currentsection,currentsubsection]

%%%%%%%%%%%%%%%%%%%%%%%%%%%%%%%%%%%%%%%%%%%%%%%%%%%%%

\begin{frame}[fragile]{Relación con Markdown}
	Para ser visualizado, los visualizadores de Markdown
	transforman el archivo en un archivo HTML, para luego
	mostrar el mismo.
	\jump
	Así se puede transformar un código de markdown a un código en HTML.
\end{frame}

%%%%%%%%%%%%%%%%%%%%%%%%%%%%%%%%%%%%%%%%%%%%%%%%%%%%%

\begin{frame}[fragile]{Relación con Markdown - Ejemplo}
	\begin{lstlisting}
# Titulo
## Subtitulo

Esto es texto **simple**
y aquí hay un [enlace](http://google.com)

* Elemento 1
* Elemento 2
* Elemento 3
		\end{lstlisting}
\end{frame}

%%%%%%%%%%%%%%%%%%%%%%%%%%%%%%%%%%%%%%%%%%%%%%%%%%%%%
\begin{frame}[fragile]{Relación con Markdown - Ejemplo Cont}
		\begin{lstlisting}[language=HTML]
<html>
	<head> <meta charset="utf-8"> </head>
	<body>
		<h1>Titulo</h1>  <h2>Subtitulo</h2>
		<p> Esto es texto <strong>simple</strong>
			y aquí hay un <a href="http://google.com">
			enlace</a></p>
		<ul>
			<li>Elemento 1</li>
			<li>Elemento 2</li>
			<li>Elemento 3</li>
		</ul>
	</body>
</html>
		\end{lstlisting}
\end{frame}

%%%%%%%%%%%%%%%%%%%%%%%%%%%%%%%%%%%%%%%%%%%%%%%%%%%%%
%%%%%%%%%%%%%%%%%%%%%%%%%%%%%%%%%%%%%%%%%%%%%%%%%%%%%

\section{Estándar}
\toc[currentsection,currentsubsection]

%%%%%%%%%%%%%%%%%%%%%%%%%%%%%%%%%%%%%%%%%%%%%%%%%%%%%

\begin{frame}[fragile]{W3C}
	La w3c (\bolder{World Wide Web Consortium}) es un consorcio internacional
	creado por \bolder{Tim Berners-Lee}, creador de la URL, el HTTP y el HTML.
	\jump
	El consorcio comienza con la participación del MIT (Massachusetts Institute of
	Technology) al que luego se unió el INRIA (Institut National de Recherche en
	Informatique et en Automatique) de Francia, el ERCIM (European Research
	Consortium for Informatics and Mathematics) y la Universidad de Keiō en Japón.
	\jump
	La organización cuenta con miembros especialistas e investigadores en ciencias
	de la computación. Está encargada de definir los estándares de diferentes
	lenguajes y tecnologías: HTML, XML, SVG, SOAP, RDP, entre otras.
\end{frame}

%%%%%%%%%%%%%%%%%%%%%%%%%%%%%%%%%%%%%%%%%%%%%%%%%%%%%

\begin{frame}[fragile]{Estándar HTML}
	El lenguaje HTML está en su versión 5 (HTML5), pero existen versiones
	anteriores.
	\jump
	Cada nueva versión es compatible con las anteriores en mayor parte.
	Algunas etiquetas desaparecen, porque se vuelven obsoletas, y otras nuevas
	que resultan necesarias aparecen.
	\jump
	El estándar HTML5 es un estándar vivo, es decir que no está en su versión
	final, sino que constantemente se está cambiando.
\end{frame}

%%%%%%%%%%%%%%%%%%%%%%%%%%%%%%%%%%%%%%%%%%%%%%%%%%%%%

\begin{frame}[fragile]{Estándar HTML - Cont.}
	Un archivo HTML correcto debería cumplir con todos los estándares postulados
	por la w3c.
	\jump
	A veces, los navegadores web no cumplen los estándares y por tanto toman
	determinación de mostrar el HTML incluso si este no es correcto.
	\jump
	Para validar que un HTML es correcto podemos usar el validador que proporciona
	la w3c en: \colored{\href{https://validator.w3.org/}{https://validator.w3.org/}}
	\jump
	Archivos que no cumplen el estándar pueden parecer verse bien en un navegador,
	pero puede verse mal en otro.
\end{frame}

%%%%%%%%%%%%%%%%%%%%%%%%%%%%%%%%%%%%%%%%%%%%%%%%%%%%%

\begin{frame}[fragile]{Resultados de validación}
	El validador da dos tipos de mensajes en caso de que haya cosas incorrectas en
	el sitio:
	\begin{itemize}
		\item \bolder{Errores}: cuando se rompe una regla HTML estricta
		\item \bolder{advertencias}: Cuando no se siguen buenas prácticas que son
			recomendables, o se tiene elementos innecesarios en el sitio web.
	\end{itemize}
	Ej. error: la etiqueta ``hgroup'' está en un lugar no permitido:
	\image[scale=0.3]{img/w3c_error.png}{Resultado del validador de la W3C con errores}
	Ej. advertencia: la etiqueta “h1” debería ser utilizado solo como
	encabezado principal del sitio:
	\image[scale=0.3]{img/w3c_warn.png}{Resultado del validador de la W3C con advertencias}
\end{frame}

%%%%%%%%%%%%%%%%%%%%%%%%%%%%%%%%%%%%%%%%%%%%%%%%%%%%%
%%%%%%%%%%%%%%%%%%%%%%%%%%%%%%%%%%%%%%%%%%%%%%%%%%%%%

\section{Estilos e interactividad}
\toc[currentsection,currentsubsection]

%%%%%%%%%%%%%%%%%%%%%%%%%%%%%%%%%%%%%%%%%%%%%%%%%%%%%

\begin{frame}[fragile]{Estilos}
	Las páginas web que podemos crear con HTML, se ven bastante planas.
	Sin embargo, los sitios que vemos en internet se ven muy distintos.
	\jump
	La pregunta sería entonces, ¿Cómo hacer para tener un sitio como alguno de los
	siguientes?
	\jump
	\image[scale=0.4]{img/css.png}{Muestra de CSS}
\end{frame}

%%%%%%%%%%%%%%%%%%%%%%%%%%%%%%%%%%%%%%%%%%%%%%%%%%%%%

\begin{frame}[fragile]{Cascade Style Sheets}
	\bolder{CSS} es un lenguaje adicional que permite describir “cómo” se va a
	ver un determinado contenido en una página web.
	\jump
	También es desarrollado y mantenido por la W3C.
	\jump
	Permite definir, por ejemplo, tipos y tamaños de fuentes, colores de texto y
	fondo, alineados de texto, sombreado de elementos, transparencias, posicionado
	de elementos complejos, etc.
	\jump
	Puede agregarse al HTML mediante una etiqueta “style”, o mejor aún, en un
	archivo separado utilizando la etiqueta “link”
	\jump
	\small
	\colored{\textit{Aprender este lenguaje no es pertinente a esta asignatura.}}
\end{frame}

%%%%%%%%%%%%%%%%%%%%%%%%%%%%%%%%%%%%%%%%%%%%%%%%%%%%%

\begin{frame}[fragile]{Interactividad}
	\bolder{JavaScript} es un lenguaje de programación que ejecuta directamente
	en el navegador.
	\jump
	También desarrollado y mantenido por la W3C.
	\jump
	Permite realizar interactividad con el usuario de formas avanzadas, mediante
	el agregado o quitado de etiquetas y contenido al HTML tras una determinada
	acción del usuario.
	\jump
	\small
	\colored{\textit{Aprender este lenguaje no es pertinente a esta asignatura.}}
\end{frame}