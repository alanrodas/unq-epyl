\section{Computadoras.}
\subsection{¿Qué son las computadoras?}
\toc[currentsection, currentsubsection]

%%%%%%%%%%%%%%%%%%%%%%%%%%%%%%%%%%%%%%%%%%%%%%%%%%%%%

\begin{frame}{Computadoras}
	La \bolder{computadora} es una máquina electrónica o electromecánica que recibe
	datos, los analiza, procesa y transforma, convirtiéndolos en información
	conveniente y útil para el posterior uso por seres humanos.
  \jump
	Una computadora está formado físicamente por numerosos \bolder{componentes
  electrónicos y mecánicos} que, en conjunto y coordinados por algún
  \bolder{programa}, pueden realizar diversas tareas a grandes velocidades.
	\jump
	Están constituidas de dos partes esenciales, el \bolder{hardware}
	y el \bolder{software}.
\end{frame}

%%%%%%%%%%%%%%%%%%%%%%%%%%%%%%%%%%%%%%%%%%%%%%%%%%%%%


\begin{frame}[shrink]{Otros tipos de computadoras}
  Cuando se habla de computadora, generalmente se piensa en la típica
  \bolder{computadora de escritorio}, o en una \bolder{notebook}.
  \jump
  Note que nuestra definición de computadora, abarca a todo dispositivo con
  componentes electrónicos, independientemente del tamaño, la forma o la utilidad
  del mismo.
  \jump
  Ejemplos de otros dispositivos que en esencia son también computadoras incluyen:
  \begin{columns}
    \begin{column}{0.33\textwidth}
      \begin{itemize}
        \item celulares.
        \item relojes inteligentes.
        \item sistemas de control de autos.
      \end{itemize}
    \end{column}
    \begin{column}{0.33\textwidth}
      \begin{itemize}
        \item tablets.
        \item robots.
        \item sistemas de domótica (IoT).
      \end{itemize}
    \end{column}
    \begin{column}{0.33\textwidth}
      \begin{itemize}
        \item calculadoras.
        \item juguetes electrónicos.
        \item y muchas mas...
      \end{itemize}
    \end{column}
  \end{columns}
\end{frame}

%%%%%%%%%%%%%%%%%%%%%%%%%%%%%%%%%%%%%%%%%%%%%%%%%%%%%
%%%%%%%%%%%%%%%%%%%%%%%%%%%%%%%%%%%%%%%%%%%%%%%%%%%%%

\subsection{Hardware.}
\toc[currentsection, currentsubsection]

%%%%%%%%%%%%%%%%%%%%%%%%%%%%%%%%%%%%%%%%%%%%%%%%%%%%%

\begin{frame}{Hardware}
  El \bolder{hardware} es la estructura física de la computadora. Comprende a
  todos los elementos \bolder{electrónicos y mecánicos} que componen al equipo,
  independientemente de la ubicación de los mismos.
  \jump
  Es decir, todos los circuitos, botones, teclas, palancas, perillas, pantallas,
  displays, dispositivos de impresión, placas, cables, circuitos, etc.
  \jump
  Una definición más pragmática sería:\\
  \bolder{si no anda y lo puedo patear, es hardware}
\end{frame}

%%%%%%%%%%%%%%%%%%%%%%%%%%%%%%%%%%%%%%%%%%%%%%%%%%%%%

\begin{frame}{Hardware}
  Ejemplo de partes de hardware en una computadora de escritorio:
  \begin{columns}
    \begin{column}{0.5\textwidth}
      \image[scale=0.15]{img/computadora}{Imagen de los componentes internos de una computador}
    \end{column}
    \begin{column}{0.5\textwidth}
      \begin{enumerate}
        \item Monitor-
        \item Placa madre (Motherboard).
        \item Microprocesador o CPU.
        \item Puertos SATA.
        \item Memoria RAM.
        \item Placas de expansión.
        \item Fuente de alimentación.
        \item Unidad de disco óptico.
        \item Unidad de disco duro.
        \item Teclado.
        \item Mouse.
      \end{enumerate}
    \end{column}
  \end{columns}
\end{frame}

%%%%%%%%%%%%%%%%%%%%%%%%%%%%%%%%%%%%%%%%%%%%%%%%%%%%%

\begin{frame}[shrink]{Hardware}
  Dentro de las partes más destacables del hardware se incluyen:
  \begin{itemize}
    \item \bolder{CPU}: (\textit{Central Processing Unit} - en español, Unidad
      Central de Procesamiento) Es un circuito que se encarga de coordinar a
      todos los componentes, realizar cálculos, ejecutar programas, etc.
    \item \bolder{Memoria RAM}: Es un circuito capaz de almacenar información
      mientras la computadora tenga energía. Guarda datos sobre el programa que
      se esta ejecutando, los archivos abiertos, etc.
    \item \bolder{Fuente de alimentación}: Es un transformador de electricidad
      que viene del toma corrientes a 220 voltios, al nivel de voltaje que requiere la
      máquina (12 voltios, 5 voltios, 3 voltios, etc.). Se encarga de darle energía a todos los
      componentes de la computadora.
    \item \bolder{Placa madre}:  Es el circuito principal de la computadora, y
      es a donde se conectan el CPU, la Memoria RAM, y a lo que se le da
      principalmente energía mediante la Fuente de alimentación. Posee los
      circuitos necesarios para conectar estos componentes, y agregar otros
      adicionales a través de ranuras estandarizadas, como los
      \bolder{Puertos SATA}.
  \end{itemize}
\end{frame}

%%%%%%%%%%%%%%%%%%%%%%%%%%%%%%%%%%%%%%%%%%%%%%%%%%%%%

\begin{frame}[shrink]{Periféricos}
  Dentro del hardware, los dispositivos que se conectan a la CPU y a la
  placa madre suelen denominarse \bolder{periféricos}.
  \jump
  Los periféricos incluyen cientos de tipos de componentes, y se caracterizan en
  diferentes categorías (bastante poco claras y muy dependientes de la
  bibliografía empleada).
  \jump
  A continuación hay ejemplos de estas clasificaciones:
\end{frame}

%%%%%%%%%%%%%%%%%%%%%%%%%%%%%%%%%%%%%%%%%%%%%%%%%%%%%

\begin{frame}[shrink]{Clasificación de Periféricos}
  \begin{itemize}
    \item \bolder{Entrada}: Sirven para ingresar información a la computadora,
      ejemplos son el \bolder{teclado}, \bolder{mouse}, \bolder{webcams},
      \bolder{microfonos}, \bolder{joysticks}, \bolder{scanners}, etc.
    \item \bolder{Salida}: Sirven para obtener información de la computadora,
      ejemplos son el \bolder{monitor}, \bolder{impresora}, \bolder{parlantes},
      \bolder{indicadores lumínicos}, \bolder{indicadores vibratorios}, etc.
    \item \bolder{Entrada/Salida}: Sirven tanto para ingresar datos, como para
      obtenerlos. Acá caen dispositivos como \bolder{pantallas táctiles}, 
      \bolder{impresoras multifunción}, etc.
    \item \bolder{Almacenamiento}: Incluye todo dispositivo que sirva para almacenar
      información y leerla posteriormente, como \bolder{discos rígidos},
      \bolder{unidades de CD/DVD}, \bolder{pendrives}, etc. Muchos autores lo
      clasifican directamente como dispositivos de entrada y salida.
  \end{itemize}
\end{frame}

%%%%%%%%%%%%%%%%%%%%%%%%%%%%%%%%%%%%%%%%%%%%%%%%%%%%%

\begin{frame}[shrink]{Efectos prácticos de la clasificación de periféricos}
  La clasificación es medio arbitraria en muchos casos. Por ejemplo, que pasa si
  el teclado cuenta con indicadores lumínicos para determinar si está activa una
  función; o si un joystick vibra cuando el jugador realiza una acción.
  \jump
  A los efectos prácticos, esta categorización es irrelevante, y solo nos
  importa la función principal del dispositivo.
\end{frame}

%%%%%%%%%%%%%%%%%%%%%%%%%%%%%%%%%%%%%%%%%%%%%%%%%%%%%

\begin{frame}{Redes}
  Las computadoras pueden además conectarse entre si, formando \bolder{Redes de
  computadoras}, que comparten información, o procesan datos de forma conjunta.
  \jump
  Internet, no es más que eso, una enorme y compleja red de computadoras
  conectadas entre si, compartiendo información, mediante reglas y protocolos
  específicos.
\end{frame}

%%%%%%%%%%%%%%%%%%%%%%%%%%%%%%%%%%%%%%%%%%%%%%%%%%%%%

\begin{frame}{Computadora como caja negra}
  Más allá de como es el hardware de la computadora (algo que verán
  con más detalle en futuras materias), a los efectos prácticos solo nos
  interesa pensarlo como una caja, a la cual le brindamos información,
  y tras transformarla de alguna forma, nos devuelve información.
  \jump
  \image{img/caja_negra}{Gráfico que muestra el proceso de caja negra descripto anteriormente.}
  \emptyline
  Lo que si nos va a interesar, es que pasa a nivel software.
\end{frame}

%%%%%%%%%%%%%%%%%%%%%%%%%%%%%%%%%%%%%%%%%%%%%%%%%%%%%
%%%%%%%%%%%%%%%%%%%%%%%%%%%%%%%%%%%%%%%%%%%%%%%%%%%%%

\subsection{Software.}
\toc[currentsection, currentsubsection]

%%%%%%%%%%%%%%%%%%%%%%%%%%%%%%%%%%%%%%%%%%%%%%%%%%%%%

\begin{frame}[shrink]{Software}
  El \bolder{software} es la parte intangible de la computadora. Es decir, es
  toda señal eléctrica que recorre los circuitos, todo programa, todo archivo
  informático, etc.
  \jump
  \bolder{Una computadora sin software no sirve para nada}\\
  Toda computadora viene de fábrica con algún software mínimo que permite al
  menos encender la computadora y manejar a bajo nivel los distintos puertos de
  la placa madre.
  \jump
  Nuevamente la definición pragmática sería:\\
  \bolder{Si no anda y solo lo puedo insultar pero no golpear, entonces es
    software}
\end{frame}

%%%%%%%%%%%%%%%%%%%%%%%%%%%%%%%%%%%%%%%%%%%%%%%%%%%%%

\begin{frame}[shrink]{Utilidad del Software}
  El hardware no sirve para nada sin un \bolder{software} (un programa) que lo controle y
  determine cómo se deben procesar los datos.
  \jump
  Toda computadora viene de fábrica con algún software mínimo que permite al
  menos encender la computadora y manejar a bajo nivel los distintos puertos de
  la placa madre.
  \jump
  Luego hay programas que permiten manipular la información almacenada
  en el equipo, y ejecutar otros programas de forma sencilla, conocidos
  como \bolder{sistemas operativos}. Ejemplos de sistemas operativos
  son Windows, macOS y Linux (luego charlamos más sobre estos).
  \jump
  Otro conjunto de software son los \bolder{programas informáticas} en
  donde se incluyen todas las aplicaciones que usamos habitualmente como Word,
  Excel, PowerPoint, o programas de dibujo como Paint y GIMP, o de audio como
  Winamp, iTunes, Audacity, y de todo otro tipo.
\end{frame}

%%%%%%%%%%%%%%%%%%%%%%%%%%%%%%%%%%%%%%%%%%%%%%%%%%%%%

\begin{frame}{Información}
  Además, las computadoras almacenan información de forma digital. Las fotos
  digitales, nuestros archivos de video, los archivos del sistema, las carpetas,
  etc. son todos también parte del software.
  \jump
  Todas las señales eléctricas que se envían internamente en la computadora para
  mostrar información en pantalla, la forma en la que se determina que hacer con
  un archivo, etc. todo es parte del software.
\end{frame}

%%%%%%%%%%%%%%%%%%%%%%%%%%%%%%%%%%%%%%%%%%%%%%%%%%%%%

\begin{frame}[shrink]{Binario}
  Las computadoras, como dispositivos eléctricos, solo permiten distinguir
  dos valores, \bolder{presencia o ausencia de electricidad} (dos niveles
  de voltaje distintos).
  \jump
  Así, toda información que maneje una computadora se encuentra en última
  instancia codificada como \bolder{cero} (ausencia de electricidad) o
  \bolder{uno} (presencia de electricidad).
  \jump
  Los números naturales pueden ser codificados de forma sencilla como una
  secuencia de ceros y unos, en lo que se conoce como \bolder{sistema binario}.
  Por ejemplo:
  \begin{columns}
    \begin{column}{0.33\textwidth}
      \begin{itemize}
        \item 0 = 0
        \item 3 = 11
        \item 6 = 110
      \end{itemize}
    \end{column}
    \begin{column}{0.33\textwidth}
      \begin{itemize}
        \item 1 = 1
        \item 4 = 100
        \item 7 = 111
      \end{itemize}
    \end{column}
    \begin{column}{0.33\textwidth}
      \begin{itemize}
        \item 2 = 10
        \item 5 = 101
        \item ...
      \end{itemize}
    \end{column}
  \end{columns}
  \emptyline
  El texto también puede codificarse como sistema binario, representando cada
  letra con un número.
\end{frame}

%%%%%%%%%%%%%%%%%%%%%%%%%%%%%%%%%%%%%%%%%%%%%%%%%%%%%

\begin{frame}{Interpretación de binario}
  \bolder{Internamente, todo en la computadora son ceros y unos, y es la forma en la
  que la computadora, o mejor dicho, el software de la computadora, interpreta
  dichos ceros y unos lo que hace que representen cosas distintas, como texto,
  imagenes, graficos 3D, planillas de cálculo o incluso otros programas.}
\end{frame}

%%%%%%%%%%%%%%%%%%%%%%%%%%%%%%%%%%%%%%%%%%%%%%%%%%%%%
%%%%%%%%%%%%%%%%%%%%%%%%%%%%%%%%%%%%%%%%%%%%%%%%%%%%%

\section{Archivos informáticos.}
\toc[currentsection, currentsubsection]

%%%%%%%%%%%%%%%%%%%%%%%%%%%%%%%%%%%%%%%%%%%%%%%%%%%%%

\begin{frame}{Archivos informáticos}
  Un \bolder{archivo informático} es el equivalente digital a un archivo en papel.
  Los archivos informáticos consisten en \bolder{cadenas de bits} (ceros y unos,
  que es la forma en la que la computadora guarda información) que se almacenan en
  algún orden y forma específicos (\bolder{codificados}), y que \bolder{interpretados}
  de alguna forma particular \bolder{representan} información específica.
  \jump
  \bolder{La codificación puede responder a un estándar o no}.
  \jump
  Dependiendo de la \bolder{codificación} y de la \bolder{interpretación} que se le
  da a un archivo, pueden distinguirse varios \bolder{tipos de archivos}.
\end{frame}

%%%%%%%%%%%%%%%%%%%%%%%%%%%%%%%%%%%%%%%%%%%%%%%%%%%%%

\begin{frame}{Tipo de archivo}
  \begin{itemize}
    \item \bolder{Archivos ejecutables}: Son los programas que corremos en el
      equipo (Word, Excel, Aplicaciones de celulares, Editores de fotos, etc.).
      Más adelante veremos que significa que un programa se \textit{ejecute}.
    \item \bolder{Archivos de datos binarios}: Son los archivos que solamente
      pueden ser leídos por programas específicos (Documentos de Word, imágenes,
      videos, audio, etc.)
    \item \bolder{Archivos de texto plano}: Son archivos que usan una codificación
      estándar y en donde su contenido representa letras del alfabeto (algún
      alfabeto). Pueden ser leídos por un \bolder{Editor de texto}. Los
      programadores trabajamos principalmente con este tipo de archivos.
  \end{itemize}
\end{frame}

%%%%%%%%%%%%%%%%%%%%%%%%%%%%%%%%%%%%%%%%%%%%%%%%%%%%%

\begin{frame}{Archivos de texto plano}
  Los archivos de texto plano \bolder{no tienen formato alguno}, no hay estilos.
  El texto es solo eso, texto. No hay negrita, no hay subrayado, no hay imágenes.
  Todo son letras, números, símbolos y espacios en blanco (caracteres).
  \jump
  Un archivo de texto plano puede representar datos de índole muy diversa, que van
  desde código de un programa, una página web, una imagen, etc.
  \jump
  Para editar un archivo de texto plano se necesita un \bolder{editor de texto}.
  \jump
  Muchas veces se confunden a los \bolder{archivos de texto} con los \bolder{
  documentos de texto}. No son la misma cosa.
\end{frame}

%%%%%%%%%%%%%%%%%%%%%%%%%%%%%%%%%%%%%%%%%%%%%%%%%%%%%

\begin{frame}{Editor de texto}
  Un \bolder{editor de texto} (también llamado \bolder{procesador de texto}) es
  un programa que permite manipular un archivo de texto. No importa que sistema
  operativo usen, probablemente haya un editor de texto ya instalado en su
  computadora.
  \jump
  Hay procesadores de texto que son muy simples y otros que agregan funciones para
  que se vuelva más fácil realizar tareas específicas. Muchos están orientados
  específicamente a programadores, otros a diseñadores de páginas web, otros a
  escritores, etc.
\end{frame}

%%%%%%%%%%%%%%%%%%%%%%%%%%%%%%%%%%%%%%%%%%%%%%%%%%%%%

\begin{frame}{Algunos editores de texto genéricos}
  \begin{columns}
    \begin{column}{0.5\textwidth}
      Windows:
      \begin{itemize}
        \item \bolder{Notepad}
        \item Notepad++
        \item Edit
      \end{itemize}
      Linux:
      \begin{itemize}
        \item \bolder{Gedit}
        \item \bolder{Pluma}
        \item \bolder{Kate}
        \item Vim
        \item Emacs
        \item Nano
      \end{itemize}
    \end{column}
    \begin{column}{0.5\textwidth}
      MacOS:
      \begin{itemize}
        \item \bolder{TextEdit}
        \item Textmate
      \end{itemize}
      Editores de texto multiplataforma:
      \begin{itemize}
        \item Atom
        \item Sublime Text
        \item Visual Studio Code
      \end{itemize}
    \end{column}
  \end{columns}
  ~\jump
  \centerline{\bolder{Hay muchos otros...}}
\end{frame}

%%%%%%%%%%%%%%%%%%%%%%%%%%%%%%%%%%%%%%%%%%%%%%%%%%%%%

\begin{frame}{Extensiones de archivo}
  La \bolder{extensión de archivo} permite identificar el \bolder{tipo de archivo}.
  Consiste en un conjunto de letras que siguen a un punto (.) y que se colocan
  como sufijo al nombre del archivo.
  \jump
  Por ejemplo los archivos con tipo ``imagen con codificación JPEG'', tendrá como
  extensión de archivo ``.jpg''.
  \jump
  Así una fotografía bajo el nombre de ``vacaciones'' con dicha codificación
  tendrá como nombre completo ``vacaciones.jpg''.
  \jump
  Distintos tipos de archivo tienen distintas extensiones.
\end{frame}

%%%%%%%%%%%%%%%%%%%%%%%%%%%%%%%%%%%%%%%%%%%%%%%%%%%%%

\begin{frame}{Algunas extensiones de archivo conocidas}
  \begin{columns}
    \begin{column}{0.33\textwidth}
      Fotos e imágenes:
      \begin{itemize}
        \item .jpg
        \item .jpeg
        \item .png
        \item .bmp
        \item .tiff
        \item .gif
        \item .svg
      \end{itemize}
    \end{column}
    \begin{column}{0.33\textwidth}
      Audio:
      \begin{itemize}
        \item .mp3
        \item .ogg
        \item .wav
        \item .3gp
        \item .m4a
        \item .flac
        \item .aiff
      \end{itemize}
    \end{column}
    \begin{column}{0.33\textwidth}
      Video:
      \begin{itemize}
        \item .mp4
        \item .avi
        \item .divx
        \item .xvid
        \item .mov
        \item .wmv
        \item .flv
        \item .mkv
      \end{itemize}
    \end{column}
  \end{columns}
\end{frame}

%%%%%%%%%%%%%%%%%%%%%%%%%%%%%%%%%%%%%%%%%%%%%%%%%%%%%

\begin{frame}{Algunas extensiones de archivo conocidas}
  \begin{columns}
    \begin{column}{0.33\textwidth}
      Archivos comprimidos:
      \begin{itemize}
        \item .zip
        \item .7z
        \item .rar
        \item .tar
        \item .gz
        \item .zipx
      \end{itemize}
    \end{column}
    \begin{column}{0.33\textwidth}
      Documentos:
      \begin{itemize}
        \item .doc
        \item .docx
        \item .odt
        \item .xls
        \item .xlsx
        \item .ods
        \item .ppt
        \item .pptx
        \item .odp
        \item .pdf
        \item .eps
      \end{itemize}
    \end{column}
    \begin{column}{0.33\textwidth}
      Archivos de texto plano:
      \begin{itemize}
        \item .txt
        \item .md
        \item .markdown
        \item .xml
        \item .html
        \item .json
        \item .js
        \item .css
        \item .c
        \item .java
      \end{itemize}
    \end{column}
  \end{columns}
\end{frame}

%%%%%%%%%%%%%%%%%%%%%%%%%%%%%%%%%%%%%%%%%%%%%%%%%%%%%

\begin{frame}{Visualizar extensiones de archivos}
  Muchos sistemas operativos ocultan las extensiones de archivo.
  \jump
  Sin embargo siempre se puede visualizar el \bolder{nombre completo de archivo}, con la
  extensión incluida. Puede seguir el tutorial en los siguientes enlaces:

  Windows:
  \begin{itemize}
    \item \colored{\href{https://support.microsoft.com/es-ar/help/865219/how-to-show-or-hide-file-name-extensions-in-windows-explorer}{https://support.microsoft.com/es-ar/help/865219/how-to-show-or-hide-file-name-extensions-in-windows-explorer}}
    \item \colored{\href{https://helpx.adobe.com/es/x-productkb/global/show-hidden-files-folders-extensions.html}{https://helpx.adobe.com/es/x-productkb/global/show-hidden-files-folders-extensions.html}}
  \end{itemize}

  MacOS:
  \begin{itemize}
    \item \colored{\href{https://support.apple.com/kb/PH19072}{https://support.apple.com/kb/PH19072}}
  \end{itemize}

  Linux:
  \begin{itemize}
    \item Habilitado por defecto en casi todas las distribuciones.
  \end{itemize}
\end{frame}

%%%%%%%%%%%%%%%%%%%%%%%%%%%%%%%%%%%%%%%%%%%%%%%%%%%%%

\begin{frame}{Extensiones de archivo para texto plano}
  \begin{itemize}
    \item Al guardar un archivo utilizando un procesador de texto, dependiendo del
      editor, podemos agregar la extensión que queremos.
    \item Si no nos deja seleccionar la extensión, podemos escribirla manualmente.
    \item Si aún así nos pone otra extensión, podemos guardar con la extensión que
      nos habilite, y renombrar el archivo luego, cambiándole la extensión.
  \end{itemize}
\end{frame}

%%%%%%%%%%%%%%%%%%%%%%%%%%%%%%%%%%%%%%%%%%%%%%%%%%%%%

\begin{frame}{Visualizadores}
  Algunos tipos de archivo requieren de un \bolder{visualizador} para poder ver
  su contenido.
  \jump
  El visualizador no es más que un programa (archivo ejecutable) capaz de leer un
  archivo y presentar la información al usuario en pantalla o a través de algún
  otro periférico (parlantes, impresora, etc.).
  \jump
  Muchas veces el visualizador es el mismo programa que se usa para editar el
  archivo, pero otras no es el caso (Ej. reproductor de videos, de audio, de
  imágenes)
\end{frame}

%%%%%%%%%%%%%%%%%%%%%%%%%%%%%%%%%%%%%%%%%%%%%%%%%%%%%

\begin{frame}{Visualizadores para algunos archivos de texto}
  No solo los archivos binarios requieren de un visualizador. Algunos archivos de
  texto pueden ser abiertos por visualizadores especiales que mostrarán su
  contenido de alguna forma especial.
  \jump
  \textit{Se verán ejemplos en la próxima clase.}
\end{frame}

%%%%%%%%%%%%%%%%%%%%%%%%%%%%%%%%%%%%%%%%%%%%%%%%%%%%%
%%%%%%%%%%%%%%%%%%%%%%%%%%%%%%%%%%%%%%%%%%%%%%%%%%%%%

\section{Directorios.}
\subsection{Directorios Informáticos.}
\toc[currentsection, currentsubsection]

%%%%%%%%%%%%%%%%%%%%%%%%%%%%%%%%%%%%%%%%%%%%%%%%%%%%%

\begin{frame}{Directorios Informáticos}
  Un \bolder{directorio} informático, también llamado muchas veces
  \bolder{carpeta} informática, es una representación digital de una carpeta
  física.
  \jump
  Al igual que los archivos informáticos, es parte del \bolder{software} de una
  computadora.
  \jump
  Permite agrupar múltiples archivos en un lugar de fácil acceso, dando lugar
  a una mejor organización.
  \jump
  Así, por ejemplo, todos los archivos que representan fotografías pueden estar
  agrupadas en un mismo directorio.
\end{frame}

%%%%%%%%%%%%%%%%%%%%%%%%%%%%%%%%%%%%%%%%%%%%%%%%%%%%%

\begin{frame}{Directorios Informáticos - Cont}
  Todo directorio tiene un nombre que lo identifica, y se encuentra dentro de
  algún directorio.
  \jump
  Es decir, los directorios se guardan dentro de otros directorios, dando lugar
  a una \bolder{estructura de árbol}.
  \jump
  Hay un único directorio, llamado \bolder{raíz} que no se encuentra dentro de
  ningún otro directorio. En Linux y MacOS la carpeta raíz se representa como
  ``/'' (barra), mientras que en Windows hay una por cada disco rígido en el
  equipo y se identifican con una letra seguida de dos puntos
  (ej. ``C:'', ``D:'', etc.)
\end{frame}

%%%%%%%%%%%%%%%%%%%%%%%%%%%%%%%%%%%%%%%%%%%%%%%%%%%%%

\begin{frame}{Estructura de árbol de directorios: Windows}
  \image[scale=0.5]{img/filesystem_windows.png}{Imagen que muestra el sistema de archivos de Windows}
\end{frame}

%%%%%%%%%%%%%%%%%%%%%%%%%%%%%%%%%%%%%%%%%%%%%%%%%%%%%

\begin{frame}{Estructura de árbol de directorios: Linux}
  \image[scale=0.2]{img/filesystem_linux.png}{Imagen que muestra el sistema de archivos de Linux}
\end{frame}

%%%%%%%%%%%%%%%%%%%%%%%%%%%%%%%%%%%%%%%%%%%%%%%%%%%%%

\begin{frame}{Directorios: Contenido}
  Un directorio, puede contener otros directorios, o archivos, o
  ambos.
  \jump
  Un directorio ``B'' que se encuentra dentro de un directorio ``A''
  se dice que es un \bolder{subdirectorio} de ``A''.
  \jump
  Un directorio que no contiene ni directorios ni archivos, se dice que
  está \bolder{vacío}.
  \jump
  \bolder{Todo archivo informático se encuentra en algún directorio}.
  \jump
  A la visualización completa de un directorio y todos sus subdirectorios
  se los suele denominar \bolder{árbol} o \bolder{jerarquía} de carpetas.
\end{frame}

%%%%%%%%%%%%%%%%%%%%%%%%%%%%%%%%%%%%%%%%%%%%%%%%%%%%%
%%%%%%%%%%%%%%%%%%%%%%%%%%%%%%%%%%%%%%%%%%%%%%%%%%%%%

\subsection{Rutas.}
\toc[currentsection, currentsubsection]

%%%%%%%%%%%%%%%%%%%%%%%%%%%%%%%%%%%%%%%%%%%%%%%%%%%%%

\begin{frame}{Rutas}
  Una ruta es la ubicación exacta de un archivo dentro del equipo,
  indicando todos los directorios y subdirectorios por los que se debe
  pasar para encontrar el mismo, partiendo desde el directorio raíz.
  \jump
  Así, la ruta \bolder{\url{C:\\Users\\Juan\\Imagenes\\foto.jpg}} indica que,
  se debe acceder a la carpeta raíz del disco ``C'', de allí ingresar a la
  carpeta ``Users'' desde allí a ``Juan'', una vez en esa carpeta acceder
  a ``Imágenes'' y finalmente, allí se encontrará el archivo ``foto.jpg''.
\end{frame}

%%%%%%%%%%%%%%%%%%%%%%%%%%%%%%%%%%%%%%%%%%%%%%%%%%%%%

\begin{frame}{Rutas}
  \begin{columns}
    \begin{column}{0.5\textwidth}
      \image[scale=0.7]{img/filesystem_tree.png}{Diagrama del sistema de archivos}
    \end{column}
    \begin{column}{0.5\textwidth}
      Algunas rutas útiles en la jerarquía anterior podrían ser:
      \begin{itemize}
        \item \colored{\url{C:\\Images\\Koala.jpg}}
        \item \colored{\url{C:\\Images\\Penguins.jpg}}
        \item \colored{\url{C:\\book\\Quick_Guide.pdf}}
        \item \colored{``\url{C:\\Music\\05}~\url{Legs.wma}''}
      \end{itemize}
      En el último caso, la ruta debe estar entre comillas,
      pues contiene espacios.
    \end{column}
  \end{columns}
\end{frame}

%%%%%%%%%%%%%%%%%%%%%%%%%%%%%%%%%%%%%%%%%%%%%%%%%%%%%

\begin{frame}{Identificación inequivoca de archivos en la máquina}
  Una ruta debe poder identificar un archivo o directorio en el equipo de forma inequívoca.
  \jump
  A consecuencia de esto, en un mismo directorio no pueden haber dos archivos (o directorio)
  con el mismo nombre, pues tendrían la misma ruta.
  \jump
  Tampoco puede haber un directorio y un archivo con el mismo nombre (Ojo, si el
  archivo se llama ``juan.jpg'' y el directorio se llama ``juan'' si se puede).
  \jump
  Si puede haber dos archivos con el mismo nombre en diferentes directorios,
  pues tienen distinta ruta.
\end{frame}

%%%%%%%%%%%%%%%%%%%%%%%%%%%%%%%%%%%%%%%%%%%%%%%%%%%%%

\begin{frame}{Intransferibilidad de rutas}
  En casos de archivos del sistema o determinados programas, estos
  se encuentran instalados siempre en el mismo lugar en todos los equipos, por
  lo que la misma ruta se puede usar en diversos equipos para identificar a los
  archivos.
  \jump
  Una ruta que habla de archivos del usuario, es intransferible a otro equipo,
  pues la estructura de carpetas en diferentes equipos no es necesariamente igual.
\end{frame}

%%%%%%%%%%%%%%%%%%%%%%%%%%%%%%%%%%%%%%%%%%%%%%%%%%%%%

\begin{frame}{Identificación del sistema operativo mediante una ruta}
  La forma de la ruta da lugar a identificar además el sistema operativo del
  equipo. Si la ruta comienza con una letra y dos puntos, o utiliza barras
  invertidas (\textbackslash), entonces es un equipo con windows. Si comienza
  con una barra y utiliza barras simples (\slash) entonces se trata de un
  equipo con Linux o MacOS.
  \jump
  \centerline{Ejemplo ruta Windows}
  \centerline{\colored{\url{C:\\Users\\Juan\\Documents\\Guide.pdf}}}
  \jump
  \centerline{Ejemplo ruta Linux/MacOS}
  \centerline{\colored{\url{/home/Juan/Documents/Guide.pdf}}}
\end{frame}

%%%%%%%%%%%%%%%%%%%%%%%%%%%%%%%%%%%%%%%%%%%%%%%%%%%%%

\begin{frame}{Rutas relativas}
  Cuando se quiere hacer referencia a un archivo desde otro
  (algo que haremos más adelante), se puede utilizar rutas relativas.
  \jump
  Una ruta relativa consiste en una ruta que indica como llegar al archivo,
  no desde el directorio raíz, sino desde el directorio en donde se
  encuentra el archivo desde donde vamos a referenciar.
  \jump
  Esto permite poder tener rutas que son un poco más independientes
  entre máquina y máquina.
\end{frame}

%%%%%%%%%%%%%%%%%%%%%%%%%%%%%%%%%%%%%%%%%%%%%%%%%%%%%

\begin{frame}{Rutas relativas: Ejemplo}
  Un ejemplo es con un sitio web que incluye imágenes (que como veremos más
  adelante requiere indicar en un archivo, la ruta hacia la imagen a agregar).
  \jump
  Uno quisiera poder diseñar el sitio en su máquina personal, pero eventualmente
  subir el sitio a internet, guardando los archivos en un servidor. Si en el
  sitio las rutas a las imágenes hacen referencia a mi computadora, y el servidor
  no tiene exactamente la misma jerarquía de carpetas, entonces en el sitio no se
  verán las imágenes.
  \jump
  La solución es el uso de rutas relativas.
  \jump
  El proceso de indicar donde queda un archivo mediante una ruta en otro se
  conoce como \bolder{referenciar}.
\end{frame}

%%%%%%%%%%%%%%%%%%%%%%%%%%%%%%%%%%%%%%%%%%%%%%%%%%%%%

\begin{frame}{Rutas relativas: Ejemplo - Cont}
  \begin{columns}
    \begin{column}{0.5\textwidth}
      \image[scale=0.4]{img/folders.png}{Imagen de una estructura de carpetas.}
    \end{column}
    \begin{column}{0.5\textwidth}
      Así, desde el archivo ``sitio.html'' podremos hacer
      referencia las imágenes como
      \begin{itemize}
        \item \url{fondos/principal.png}
        \item \url{fotos/personales/juan.jpg}
      \end{itemize}
    \end{column}
  \end{columns}
  \jump
  El directorio en donde se encuentra ``sitio web'' es irrelevante en dichas
  rutas, y podría ser \colored{``\url{C:\\Usuarios\\Juan}''} como
  \colored{``\url{/var/www/site}''}.
\end{frame}

%%%%%%%%%%%%%%%%%%%%%%%%%%%%%%%%%%%%%%%%%%%%%%%%%%%%%

\begin{frame}{Rutas relativas: Carpetas especiales}
 Una cosa muy común es querer tener una ruta relativa a partir de un archivo
 de referencia.
 \jump
 Es decir, la ruta parte desde la carpeta en donde se encuentra dicho archivo en adelante.
 \jump
 Pero podemos referenciar archivos que estén en una carpeta más arriba que nuestro archivo
 de referencia mediante el uso del directorio ``..''
 \jump
 ``..'' (sin comillas) indica que la ruta debe ir a la carpeta superior, y desde
 allí continúa.
\end{frame}

%%%%%%%%%%%%%%%%%%%%%%%%%%%%%%%%%%%%%%%%%%%%%%%%%%%%%

\begin{frame}{Rutas relativas: Ejemplo Carpetas Especiales}
  \begin{columns}
    \begin{column}{0.5\textwidth}
      \image[scale=0.4]{img/folders_special.png}{Imagen con estructura de carpetas que requieren de dos puntos para subir un nivel}
    \end{column}
    \begin{column}{0.5\textwidth}
      En este caso, para referenciar a las imágenes desde el archivo ``inicio.html''
      de forma relativa, se debe indicar que, desde la carpeta en donde se encuentra
      este archivo, hay que ``subir'' dos veces, para luego entrar a la
      carpeta ``img'' y desde allí se accede a las imágenes.
      \begin{itemize}
        \item \url{../../img/foto_a.png}
        \item \url{../../img/foto_b.png}
        \item \url{../../img/foto_c.png}
      \end{itemize}
    \end{column}
  \end{columns}
\end{frame}

%%%%%%%%%%%%%%%%%%%%%%%%%%%%%%%%%%%%%%%%%%%%%%%%%%%%%
%%%%%%%%%%%%%%%%%%%%%%%%%%%%%%%%%%%%%%%%%%%%%%%%%%%%%

\subsection{URIs.}
\toc[currentsection, currentsubsection]

%%%%%%%%%%%%%%%%%%%%%%%%%%%%%%%%%%%%%%%%%%%%%%%%%%%%%

\begin{frame}[fragile]{URI}
	Una URI (Uniform Resource Identifier) es una secuencia de caracteres que
  identifica los recursos de una red de forma unívoca.
  \jump
  Es decir, extiende el concepto de ruta, para un archivo que se encuentra en
  una red, por ejemplo, en internet, pero también sirve para identificar archivos de
  la máquina local.
	\jump
	Se componen de varias partes:
	\begin{itemize}
		\item Esquema: (http, mailto, file)
		\item Autoridad (//www.example.com)
		\item Ruta (organizado de forma jerárquica)
		\item Consulta (?clave=valor)
		\item Fragmento (\#if)
	\end{itemize}
\end{frame}

%%%%%%%%%%%%%%%%%%%%%%%%%%%%%%%%%%%%%%%%%%%%%%%%%%%%%

\begin{frame}[fragile]{URI Ejemplos}
	Algunos ejemplos de URIs son:
  \begin{itemize}
    \item \colored{\url{https://bing.com}}
    \item \colored{\url{https://listado.mercadolibre.com.ar/notebook-i7}}
    \item \colored{\url{https://www.google.com.ar/maps/place/Universidad+Nacional+de+Quilmes}}
    \item \colored{\url{https://source.unsplash.com/random/800x600}}
	\end{itemize}
\end{frame}

%%%%%%%%%%%%%%%%%%%%%%%%%%%%%%%%%%%%%%%%%%%%%%%%%%%%%

\begin{frame}[fragile]{URI: Utilidad}
  Las URIs son un estándar, independiente del sistema operativo (siempre usan
  barra simple) y que pone reglas para el caso de archivos que contienen espacios,
  etc.
  \jump
  Por tanto, son muy útiles para hacer referencia a archivos en elementos que
  tienen que funcionar independientemente del sistema operativo, como un
  documento, sitios web, o programas.
  \jump
  Vamos a hacer uso intensivo de URIs más adelante, y lo van a usar durante
  toda la carrera.
\end{frame}

%%%%%%%%%%%%%%%%%%%%%%%%%%%%%%%%%%%%%%%%%%%%%%%%%%%%%
%%%%%%%%%%%%%%%%%%%%%%%%%%%%%%%%%%%%%%%%%%%%%%%%%%%%%

\section{Programas.}
\toc[currentsection, currentsubsection]

%%%%%%%%%%%%%%%%%%%%%%%%%%%%%%%%%%%%%%%%%%%%%%%%%%%%%

\begin{frame}{¿Qué es un programa informático?}
  A los efectos prácticos, un programa no es más que una cierta combinación de
  ceros y unos, que es interpretada de una forma particular por la computadora.
  \bolder{Es decir, un archivo ejecutable.}
  \jump
  Así, para saber programar, basta con saber que efectos producirán en la
  computadora las diferentes combinaciones de ceros y unos.
  \jump
  Las primeras computadoras se programaban de esta forma, pero a los efectos
  prácticos este método es engorroso, lento y propenso a errores. Por esos
  motivos hoy ya no se utiliza este método, sino que en general se usan
  compiladores.
\end{frame}

%%%%%%%%%%%%%%%%%%%%%%%%%%%%%%%%%%%%%%%%%%%%%%%%%%%%%

\begin{frame}{Compiladores}
  Los \bolder{compiladores} son programas ya hechos que tienen por función
  leer un archivo que contiene texto escrito en algún lenguaje de programación
  (una secuencia de unos y ceros) y transformarlo en un programa ejecutable
  por la computadora (otra secuencia de unos y ceros).
  \bolder{Es decir, pasa de un archivo de texto plano a un archivo ejecutable.}
  \jump
  Esto tiene la ventaja de que permite al programador expresar mejor que
  desea realizar en la computadora, utilizando un lenguaje lo más parecido
  posible al lenguaje natural.
  \jump
  \bolder{El proceso no es reversible}
  \jump
  La clase que viene veremos más acerca de lenguajes y de como la computadora
  entiende lo que le decimos.
\end{frame}

%%%%%%%%%%%%%%%%%%%%%%%%%%%%%%%%%%%%%%%%%%%%%%%%%%%%%

\begin{frame}{Código Fuente}
  El \bolder{código fuente} consiste en los archivos de texto que escriben los
  programadores indicándole a la máquina cosas para hacer (en las unidades que
  vienen vamos a ver más detalles sobre esto).
  \jump
  El código consiste en texto escrito en algún lenguaje específico, que el
  programador puede entender, pero la computadora no (al menos no como un
  programa, sino solo como texto)
\end{frame}

%%%%%%%%%%%%%%%%%%%%%%%%%%%%%%%%%%%%%%%%%%%%%%%%%%%%%

\begin{frame}{Código Objeto}
  El \bolder{código objeto} consiste en archivos binarios ejecutables. Es
  decir en programas (software).
  \jump
  El código objeto son secuencias de unos y ceros, inentendibles para un
  programador, pero entendibles por la computadora.
\end{frame}

%%%%%%%%%%%%%%%%%%%%%%%%%%%%%%%%%%%%%%%%%%%%%%%%%%%%%

\begin{frame}{Compilación}
  El proceso de \bolder{compilación}, entonces, es lo que lleva a cabo el
  \bolder{compilador}. Consiste en pasar un archivo que contiene
  \bolder{código fuente} en un \bolder{lenguaje} que el compilador entiende,
  a un archivo con \bolder{código objeto} que la computadora puede interpretar
  como un programa.
  \jump
  \image[scale=0.25]{img/compilacion.png}{Diagrama con el flujo de un programa compilandose.}
\end{frame}

%%%%%%%%%%%%%%%%%%%%%%%%%%%%%%%%%%%%%%%%%%%%%%%%%%%%%

\begin{frame}{Cuidado: Simplificación}
  La realidad es que el proceso es un poco, bastante, más complejo. Sin embargo,
  esto es suficiente para lo que nos interesa llevarnos de la materia. Estos
  temas los van a ver en más profundidad a lo largo de la carrera en diversas
  materias.
  \jump
  En esta materia, nos vamos a centrar en entender un poco más en que consiste
  el código fuente, y como los programadores escriben programas. Pero eso será
  en futuras clases.
\end{frame}