\section{Markdown.}
\subsection{El lenguaje.}
\toc[currentsection,currentsubsection]

%%%%%%%%%%%%%%%%%%%%%%%%%%%%%%%%%%%%%%%%%%%%%%%%%%%%%

\begin{frame}{Markdown}
  Es un lenguaje de marcado \bolder{diseñado para maximizar la facilidad de lectura},
  tanto en su entrada como en su salida.
  \jump
  Se inspira fuertemente en convenciones usadas al momento de enviar correos
  electrónicos en texto plano.
  \jump
  Muchos sistemas de blogs permiten escribir artículos en código Markdown.
  \jump
  En programación es ampliamente utilizado para escribir documentación de los
  programas.
  \jump
  \bolder{Su salida suele ser un documento HTML.}
\end{frame}

%%%%%%%%%%%%%%%%%%%%%%%%%%%%%%%%%%%%%%%%%%%%%%%%%%%%%

\begin{frame}{Markdown: Editores y Visualizadores}
  Se puede utilizar cualquier editor de texto para editar el archivo.
  \jump
  Se debe contar con un visualizador específico para Markdown para poder ver el
  resultado final.
  \jump
  Como su salida es un HTML, un visualizador de HTML también puede servir.
\end{frame}

%%%%%%%%%%%%%%%%%%%%%%%%%%%%%%%%%%%%%%%%%%%%%%%%%%%%%

\begin{frame}{Markdown: Editores}
  Hay editores de texto específicos para Markdown, que permiten ver tanto la
  entrada como la salida al mismo tiempo.
  \jump
  Hay incluso editores de markdown online, basta contar con conexión a internet.
  Vamos a estar utilizando el siguiente:
  \jump
  \colored{\href{http://dillinger.io/}{http://dillinger.io/}}
  \jump
  Pero pueden utilizar también:
  \jump
  \bolder{Atom, Sublime Text, Visual Studio Code, etc.}
  \jump
  (Algunos pueden requerir complementos para visualizar la salida)
\end{frame}

%%%%%%%%%%%%%%%%%%%%%%%%%%%%%%%%%%%%%%%%%%%%%%%%%%%%%

\begin{frame}{Markdown: ¿Por qué lo usamos?}
  Para nosotros va a ser de extrema importancia, pues va a ser el
  formato con el que vamos a enviar mails y documentar nuestros
  programas (No solo en esta materia sino en toda la carrera).
  \jump
  Es un estándar en la industria, por lo que aprenderlo es fundamental
  para cualquiera que quiera dedicarse a la programación.
\end{frame}

%%%%%%%%%%%%%%%%%%%%%%%%%%%%%%%%%%%%%%%%%%%%%%%%%%%%%

\begin{frame}{Markdown: Ejemplos de markdown en la industria}
  El sitio web \bolder{GitHub} que almacena código fuente de muchos programas
  de \bolder{software libre} (\textit{luego veremos de que se trata eso}) utiliza
  Markdown como lenguaje para documentar los proyectos.
  \jump
  Algunos proyectos en GitHub:
  \begin{itemize}
    \item \href{https://github.com/apache/httpd}{Apache HTTPD.}
    \item \href{https://github.com/cereda/arara}{Arara.}
    \item \href{https://github.com/scala/scala}{Scala.}
  \end{itemize}
\end{frame}

%%%%%%%%%%%%%%%%%%%%%%%%%%%%%%%%%%%%%%%%%%%%%%%%%%%%%

\begin{frame}{Markdown como generador de sitios web}
  Markdown está muy ligado a otro lenguaje de marcado, el HTML. De hecho, la
  mayoría de los visualizadores de Markdown transforman el mismo en un documento
  HTML, documento que, como veremos en próximas clases, sirve para definir
  sitios web.
  \jump
  Por ese motivo, muchos visualizadores dan la opción de obtener el archivo HTML
  a partir de un archivo Markdown. Eso, si bien puede resultar útil en algunos
  escasos escenarios, no es en absoluto la intención de Markdown.
  \jump
  Es decir, \bolder{Markdown no es un lenguaje pensado para armar y describir
  sitios web}, ni siquiera bosquejos del mismo. Por tanto no debe ser utilizado
  para ello.
\end{frame}

%%%%%%%%%%%%%%%%%%%%%%%%%%%%%%%%%%%%%%%%%%%%%%%%%%%%%
%%%%%%%%%%%%%%%%%%%%%%%%%%%%%%%%%%%%%%%%%%%%%%%%%%%%%

\subsection{Sintaxis y ejemplos.}
\toc[currentsection,currentsubsection]

%%%%%%%%%%%%%%%%%%%%%%%%%%%%%%%%%%%%%%%%%%%%%%%%%%%%%

\begin{frame}{Markdown: Párrafos}
  \begin{itemize}
    \item Todo texto escrito, sin marcas aparece tal cual.
    \item Todo el texto que no tiene saltos de línea conforma un párrafo.
      Si hay un salto de línea, este aparece en el resultado, pero no se termina
      el párrafo.
    \item Si se aplican dos o más saltos de línea consecutivos, entonces se da
      comienzo a un nuevo párrafo (No importa si hay 20 saltos de línea
      consecutivos, el resultado no mostrará 20, sino solo el inicio de un nuevo
      párrafo)
  \end{itemize}
\end{frame}

%%%%%%%%%%%%%%%%%%%%%%%%%%%%%%%%%%%%%%%%%%%%%%%%%%%%%

\begin{frame}{Markdown: Párrafos. Visualización}
  \image[scale=0.4]{img/markdown_parrafos.png}{Resultado de documento Markdown con párrafos.}
\end{frame}

%%%%%%%%%%%%%%%%%%%%%%%%%%%%%%%%%%%%%%%%%%%%%%%%%%%%%

\begin{frame}{Markdown: Títulos}
  El texto subrayado con signos de igual (=) es un título, el resaltado
  con signos de menos (-) es un subtítulo.
  \jump
  Las almohadillas (\#) también representan título que van de mayor a menor
  importancia dependiendo de la cantidad de almohadillas consecutivas (máximo 6)
\end{frame}

%%%%%%%%%%%%%%%%%%%%%%%%%%%%%%%%%%%%%%%%%%%%%%%%%%%%%

\begin{frame}{Markdown: Títulos. Visualización}
	\image[scale=0.4]{img/markdown_titulos.png}{Resultado de documento Markdown con titulos.}
\end{frame}

%%%%%%%%%%%%%%%%%%%%%%%%%%%%%%%%%%%%%%%%%%%%%%%%%%%%%

\begin{frame}{Markdown: Énfasis}
  Colocar texto entre asteriscos (*) o guiones bajos (\_) permite dar énfasis a
  palabras o frases.
  \jump
  \centerline{\colored{*Un asterisco da por resultado italica*.}}
  \centerline{\colored{\_Un guion bajo tambien es italica\_.}}
  \centerline{\colored{**Dos asteriscos dan negrita**.}}
  \centerline{\colored{\_\_Al igual que dos guiones bajos\_\_.}}
\end{frame}

%%%%%%%%%%%%%%%%%%%%%%%%%%%%%%%%%%%%%%%%%%%%%%%%%%%%%

\begin{frame}{Markdown: Énfasis. Visualización}
	\image[scale=0.4]{img/markdown_enfasis.png}{Resultado de documento Markdown con enfasis.}
\end{frame}

%%%%%%%%%%%%%%%%%%%%%%%%%%%%%%%%%%%%%%%%%%%%%%%%%%%%%

\begin{frame}{Markdown: Citas}
  En Markdown se puede citar texto utilizando un formato similar al que usan los
  sistemas correo electrónico, con el carácter de mayor (>).
\end{frame}

%%%%%%%%%%%%%%%%%%%%%%%%%%%%%%%%%%%%%%%%%%%%%%%%%%%%%

\begin{frame}{Markdown: Citas. Visualización}
	\image[scale=0.4]{img/markdown_citas.png}{Resultado de documento Markdown con citas.}
\end{frame}

%%%%%%%%%%%%%%%%%%%%%%%%%%%%%%%%%%%%%%%%%%%%%%%%%%%%%

\begin{frame}{Markdown: Listas}
  Se pueden generar listas no ordenadas comenzando la línea con asterisco (*),
  signo de suma (+) o de resta (-) siendo todos equivalentes.
  \jump
  Se pueden generar listas numeradas usando el número para el inicio de la línea
  seguido de un punto (1.).
\end{frame}

%%%%%%%%%%%%%%%%%%%%%%%%%%%%%%%%%%%%%%%%%%%%%%%%%%%%%

\begin{frame}{Markdown: Listas. Visualización}
	\image[scale=0.4]{img/markdown_listas.png}{Resultado de documento Markdown con listas.}
\end{frame}

%%%%%%%%%%%%%%%%%%%%%%%%%%%%%%%%%%%%%%%%%%%%%%%%%%%%%

\begin{frame}{Markdown: Separadores}
  Se pueden ingresar líneas de separación horizontales usando al menos tres signos
  de menos consecutivos (---) o tres asteriscos (***) y luego seguirlo de un salto
  de línea.
  \jump
  También es posible agregar varios signos de menos o asteriscos más para
  representar la misma separación.
\end{frame}

%%%%%%%%%%%%%%%%%%%%%%%%%%%%%%%%%%%%%%%%%%%%%%%%%%%%%

\begin{frame}{Markdown: Separadores. Visualización}
	\image[scale=0.4]{img/markdown_separadores.png}{Resultado de documento Markdown con separadores.}
\end{frame}

%%%%%%%%%%%%%%%%%%%%%%%%%%%%%%%%%%%%%%%%%%%%%%%%%%%%%

\begin{frame}{Markdown: Enlaces}
  Se puede agregar enlaces a un sitio web (links) colocando el texto a mostrar en
  el enlace entre corchetes seguido por la dirección a la cual se va al
  presionarlo entre paréntesis.
  \jump
  Por ejemplo:
  \centerline{\colored{[Ir a Google](http://google.com)}}
\end{frame}

%%%%%%%%%%%%%%%%%%%%%%%%%%%%%%%%%%%%%%%%%%%%%%%%%%%%%

\begin{frame}{Markdown: Enlaces. Visualización}
	\image[scale=0.4]{img/markdown_enlaces.png}{Resultado de documento Markdown con enlaces.}
\end{frame}

%%%%%%%%%%%%%%%%%%%%%%%%%%%%%%%%%%%%%%%%%%%%%%%%%%%%%

\begin{frame}{Markdown: Imágenes}
  Se pueden insertar imágenes de la misma forma que un link, pero agregando el
  signo de exclamación antes de los corchetes. En este caso, los corchetes
  contienen la descripción de la imagen (No deberían quedar vacíos).
  \jump
  Por ejemplo:
  \centerline{\colored{![Descripción de la imágen](http://link\_a\_una\_imagen.jpg)}}
\end{frame}

%%%%%%%%%%%%%%%%%%%%%%%%%%%%%%%%%%%%%%%%%%%%%%%%%%%%%

\begin{frame}{Markdown: Imágenes. Visualización}
	\image[scale=0.4]{img/markdown_imagenes.png}{Resultado de documento Markdown con imágenes.}
\end{frame}

%%%%%%%%%%%%%%%%%%%%%%%%%%%%%%%%%%%%%%%%%%%%%%%%%%%%%

\begin{frame}{Markdown: Otros elementos}
  Hay muchos otros elementos, como citas, inclusión de código de ejemplos,
  etc. No es el propósito de este curso ni de esta diapositiva volverlos
  expertos en Markdown.
  \jump
  Si les interesa, siempre se puede encontrar mucha más información en
  internet. Basta buscar ``sintaxis de markdown'' o algo parecido.
\end{frame}
